\documentclass{article}
% Change "article" to "report" to get rid of page number on title page
\usepackage{amsmath,amsfonts,amsthm,amssymb}
\usepackage{setspace}
\usepackage{Tabbing}
\usepackage{fancyhdr}
\usepackage{lastpage}
\usepackage{extramarks}
\usepackage{chngpage}
\usepackage{indentfirst}
\usepackage{soul,color}
\usepackage{graphicx,float,wrapfig}
\usepackage{gauss}
\usepackage{dcolumn}
\newcolumntype{2}{D{.}{}{2.0}}


\usepackage{etoolbox}
\makeatletter
\patchcmd\g@matrix
 {\vbox\bgroup}
 {\vbox\bgroup\normalbaselines}% restore the standard baselineskip
 {}{}
\makeatother

% In case you need to adjust margins:
\topmargin=-0.45in      %
\evensidemargin=0in     %
\oddsidemargin=0in      %
\textwidth=7in        %
\textheight=9.0in       %
\headsep=0.25in         %

% Homework Specific Information
\newcommand{\hmwkTitle}{Homework 2,\ Congruence}
\newcommand{\hmwkDueDate}{Friday,\ September\ 11,\ 2015}
\newcommand{\hmwkClass}{Math\ 620}
\newcommand{\hmwkClassTime}{10:00}
\newcommand{\hmwkClassInstructor}{Boynton}
\newcommand{\hmwkAuthorName}{Kailee\ Gray}

\newtheorem{theorem}{Theorem}[section]
\newtheorem{lemma}[theorem]{Lemma}
\newtheorem{proposition}[theorem]{Proposition}
\newtheorem{corollary}[theorem]{Corollary}


\newenvironment{definition}[1][Definition]{\begin{trivlist}
\item[\hskip \labelsep {\bfseries #1}]}{\end{trivlist}}
\newenvironment{example}[1][Example]{\begin{trivlist}
\item[\hskip \labelsep {\bfseries #1}]}{\end{trivlist}}
\newenvironment{remark}[1][Remark]{\begin{trivlist}
\item[\hskip \labelsep {\bfseries #1}]}{\end{trivlist}}




% Setup the header and footer
\pagestyle{fancy}                                                       %
\lhead{\hmwkAuthorName}                                                 %
\chead{\hmwkClass\ (\hmwkClassInstructor\ \hmwkClassTime): \hmwkTitle}  %
\rhead{\firstxmark}                                                     %
\lfoot{\lastxmark}                                                      %
\cfoot{}                                                                %
\rfoot{Page\ \thepage\ of\ \pageref{LastPage}}                          %
\renewcommand\headrulewidth{0.4pt}                                      %
\renewcommand\footrulewidth{0.4pt}                                      %

% This is used to trace down (pin point) problems
% in latexing a document:
%\tracingall

%%%%%%%%%%%%%%%%%%%%%%%%%%%%%%%%%%%%%%%%%%%%%%%%%%%%%%%%%%%%%
% Some tools
\newcommand{\enterProblemHeader}[1]{\nobreak\extramarks{#1}{#1 continued on next page\ldots}\nobreak%
                                    \nobreak\extramarks{#1 (continued)}{#1 continued on next page\ldots}\nobreak}%
\newcommand{\exitProblemHeader}[1]{\nobreak\extramarks{#1 (continued)}{#1 continued on next page\ldots}\nobreak%
                                   \nobreak\extramarks{#1}{}\nobreak}%

\newlength{\labelLength}
\newcommand{\labelAnswer}[2]
  {\settowidth{\labelLength}{#1}%
   \addtolength{\labelLength}{0in}%
   \changetext{}{-\labelLength}{}{}{}%
   \noindent\fbox{\begin{minipage}[c]{\columnwidth}#2\end{minipage}}%
   \marginpar{\fbox{#1}}%

   % We put the blank space above in order to make sure this
   % \marginpar gets correctly placed.
   \changetext{}{+\labelLength}{}{}{}}%

\setcounter{secnumdepth}{0}
\newcommand{\homeworkProblemName}{}%
\newcounter{homeworkProblemCounter}%
\newenvironment{homeworkProblem}[1][Exercise \arabic{homeworkProblemCounter}]%
  {\stepcounter{homeworkProblemCounter}%
   \renewcommand{\homeworkProblemName}{#1}%
   \section{\homeworkProblemName}%
   \enterProblemHeader{\homeworkProblemName}}%
  {\exitProblemHeader{\homeworkProblemName}}%

\newcommand{\problemAnswer}[1]
  {\noindent\fbox{\begin{minipage}[c]{\columnwidth}#1\end{minipage}}}%

\newcommand{\problemLAnswer}[1]
    {\noindent\fbox{\begin{minipage}[c]{\columnwidth}#1\end{minipage}}}%

\newcommand{\homeworkSectionName}{}%
\newlength{\homeworkSectionLabelLength}{}%
\newenvironment{homeworkSection}[1]%
  {% We put this space here to make sure we're not connected to the above.
   % Otherwise the changetext can do funny things to the other margin

   \renewcommand{\homeworkSectionName}{#1}%
   \settowidth{\homeworkSectionLabelLength}{\homeworkSectionName}%
   \addtolength{\homeworkSectionLabelLength}{0 in}%
   \changetext{}{-\homeworkSectionLabelLength}{}{}{}%
   \subsection{\homeworkSectionName}%
   \enterProblemHeader{\homeworkProblemName\ [\homeworkSectionName]}}%
  {\enterProblemHeader{\homeworkProblemName}%

   % We put the blank space above in order to make sure this margin
   % change doesn't happen too soon (otherwise \sectionAnswer's can
   % get ugly about their \marginpar placement.
   \changetext{}{+\homeworkSectionLabelLength}{}{}{}}%

\newcommand{\sectionAnswer}[1]
  {\noindent\fbox{\begin{minipage}[c]{\columnwidth}#1\end{minipage}}}%
   \enterProblemHeader{\homeworkProblemName}\exitProblemHeader{\homeworkProblemName}%
  
 \newcommand{\R}{{\mathbb R}}
          \newcommand{\nil}{\varnothing}
          \newcommand{\N}{{\mathbb N}}
          \newcommand{\Z}{{\mathbb Z}}
        \newcommand{\MOD}{{ \ (\text{mod} \ }}


%%%%%%%%%%%%%%%%%%%%%%%%%%%%%%%%%%%%%%%%%%%%%%%%%%%%%%%%%%%%%
% Make title
\title{\vspace{2in}\textmd{\textbf{\hmwkClass:\ \hmwkTitle}}\\\normalsize\vspace{0.1in}\small{Due\ on\ \hmwkDueDate}\\\vspace{0.1in}\large{\textit{\hmwkClassInstructor\ \hmwkClassTime}}\vspace{3in}}
\date{}
\author{\textbf{\hmwkAuthorName}}
%%%%%%%%%%%%%%%%%%%%%%%%%%%%%%%%%%%%%%%%%%%%%%%%%%%%%%%%%%%%%

\begin{document}
\begin{spacing}{1.1}
\maketitle
\newpage
% Uncomment the \tableofcontents and \newpage lines to get a Contents page
% Uncomment the \setcounter line as well if you do NOT want subsections
%       listed in Contents
%\setcounter{tocdepth}{1}
%\tableofcontents
%\newpage

% When problems are long, it may be desirable to put a \newpage or a
% \clearpage before each homeworkProblem environment

\clearpage

\begin{homeworkProblem}
Exercise 1.3.4 from B\&B: Solve the conguence $20x\ \equiv\ 12\ ($mod$\ 72)$. \\
\\
\problemAnswer{
Note $\gcd(20,72)=4$. Note $4 \ | \ 12$ so there will be $4$ distinct solutions modulo $72$. If $20x\ \equiv\ 12\ (\text{mod}\ 72)$, $20x\ =\ 12 + 72k$ for some integer $k$. $20,\ 12,\ 72$ are all divisible by 4, so the previous equation is equivalent to $5x\ =\ 3 + 18k$. This yields the congruence
\begin{equation}
5x\ \equiv\ 3\ (\text{mod}\ 18).
\end{equation}
Since $\gcd(5,18)=1$ proposition 1.3.4 in B\&B implies there exists some integer $b$ such that $5b\ \equiv\ 1 ($mod$\ 18$. Apply the extended Euclidean Algorithm to find this $b$:
\begin{equation*}
18=3\cdot 5 + 3 \Leftrightarrow 3 = 18 - 3 \cdot 5,
\qquad
5 = 3 \cdot 1 + 2 \Leftrightarrow 2 = 5 - 3 \cdot 1,
\qquad
3 = 2 \cdot 1 + 1 \Leftrightarrow 1 = 3 - 2 \cdot 1.
\end{equation*}
Then, using back substitution we have
 \begin{equation*}
1 = 3 - (5 - 3 \cdot 1) \cdot 1 = 2\cdot 3 - 5 = 2\cdot (18- 3 \cdot 5) - 5 = 2\cdot 18 -6\cdot 5-5=2\cdot18-7\cdot5
\end{equation*}.
Thus, $b=-7$ which is equivalent to $11$ mod $18$. So, multiply both sides of equation (1) by $11$ to obtain
\begin{equation*}
11\cdot 5x\ \equiv\ 11\cdot3\ (\text{mod}\ 18) \Leftrightarrow 55x\ \equiv\ 33\ (\text{mod}\ 18) \Leftrightarrow 1\cdot x\ \equiv\ 15\ (\text{mod}\ 18) \Leftrightarrow x\ \equiv\ 15\ (\text{mod}\ 18) 
\end{equation*}
Hence, the solutions of the given congruence are $15,\ 33,\ 51,\ 69$ mod $72$.

}
\end{homeworkProblem}

\begin{homeworkProblem}
Exercise 1.3.12 of B\&B: Show that $4\cdot(n^2+1)$ is never divisible by $11$.\\
\\
 \problemLAnswer{
\begin{proof}
	If there was an integer $n$ such that $11 \ | \ 4\cdot(n^2+1)$, then \ $4\cdot(n^2+1) \equiv 0 \ (\text{mod}\ 11)$. Suppose there is such an $n$. If $4\cdot(n^2+1) \equiv 0 \ (\text{mod}\ 11)$, then $4n^2 + 4 \equiv \ 0 \ (\text{mod}\ 11) $ and $\ 4n^2 \equiv \ -4 \ (\text{mod}\ 11)$. Because $gcd(4,11)=1$, $\ 4n^2 \equiv \  -4 \ ( \text{mod}\ 11)$ is equivalent to $\ n^2 \equiv \ -1\  (\text{mod}\ 11)$. This is equivalent to $\ n^2 \equiv \ 10\ (\text{mod}\ 11)$. By the division algorithm, all integers can be written as $k + 11\cdot l, \ k \in \mathbb{Z}_{11}\  \text{and}\  l \in \mathbb{Z}$; thus it suffices to check all $n \in \mathbb{Z}_{11}$ to see if such an $n$ exists:
	\begin{equation*}
	0^2 \equiv 0 \ (\text{mod}\ 11),
	\quad
			1^2 \equiv 1 \ (\text{mod}\ 11),
	\quad
			2^2 \equiv 4 \ (\text{mod}\ 11),
	\quad
			3^2 \equiv 9 \ (\text{mod}\ 11),	
	\quad
			4^2 \equiv 5 \ (\text{mod}\ 11),
				\quad
			5^2 \equiv 4 \ (\text{mod}\ 11),
	\end{equation*}
\begin{equation*}
			5^2 \equiv 4 \ (\text{mod}\ 11),
	\quad
			6^2 \equiv 3 \ (\text{mod}\ 11),
				\qquad
			7^2 \equiv 5 \ (\text{mod}\ 11),
	\quad
			8^2 \equiv 9 \ (\text{mod}\ 11),
\quad
			9^2 \equiv 4 \ (\text{mod}\ 11),
			\quad
10^2 \equiv 1 \ (\text{mod}\ 11).
\end{equation*}

Since no $n\in \mathbb{Z}_{11}$ satisfies the congruence $4\cdot(n^2+1) \equiv 0 \ (\text{mod}\ 11)$ we know no such integer $n$ exists. Hence, \ $4\cdot(n^2+1)$ is never divisible by $11$.
\end{proof}
}
\end{homeworkProblem}
\newpage
\begin{homeworkProblem}
Exercise 1.3.14 of B \& B: Find the units digit of $3^{29}+11^{12}+15$.\\

\problemLAnswer{
\begin{proof}
	To find the units digit we will reduce $3^{29}+11^{12}+15$ mod $10$. 
	Notice,

\begin{equation*}
\left( 3^{29}+11^{12}+15 \right)\ \text{mod} \ 10 = 3^{29} \ \text{mod} \ 10 + 11^{12}\  \text{mod} \ 10 + 15  \ \text{mod} \ 10.
\end{equation*}

Now we will reduce each of these integers mod 10:

\begin{equation}
	3^{29} \ \text{mod} \ 10=(3^4)^7 \cdot 3\  \text{mod}\  10 = (81)^7 \cdot 3\  \text{mod}\  10 = (1)^7 \cdot 3\  \text{mod}\ 10 = 1 \cdot 3\  \text{mod}\ 10 = 3\  \text{mod}\ 10,
\end{equation}

\begin{equation}
	11^{12} \ \text{mod} \ 10 = (1)^{12} \  \text{mod}\  10 = 1 \  \text{mod}\  10,
	\end{equation}
	and
	\begin{equation}
	15 \ \text{mod} \ 10 = 5 \  \text{mod}\  10. 
	\end{equation}
Thus,

\begin{equation*}
\left( 3^{29}+11^{12}+15 \right)\ \text{mod} \ 10 = 3 \ \text{mod} \ 10 + 1\  \text{mod} \ 10 + 5  \ \text{mod} \ 10 = \left(3 + 1 + 5 \right) \ \text{mod}\ 10 = 9 \ \text{mod}\ 10.
\end{equation*}
Hence the units digit of $3^{29}+11^{12}+15$ is $9$.
\end{proof}
}

\end{homeworkProblem}


\begin{homeworkProblem}
Exercise 1.3.20 of B \& B: Solve the following system of congruences:
\begin{equation}
2x \equiv \ 5 \ (\text{mod} \ 7),
\qquad
\qquad
3x \equiv 4 \ (\text{mod} \ 8)	
\end{equation}

\problemLAnswer{
First we will solve each of the congruences in equation 5 for $x$. By trial and error, $-3$ is found to be an inverse of $2$ mod $7$ and $-5$ is found to be an inverse of $3$ mod $8$. Applying these inverses we have:
\begin{equation*}
-3\cdot2x \equiv \ -3 \cdot 5 \ (\text{mod} \ 7),
\qquad
-6x \equiv \ -15 \ (\text{mod} \ 7),
\qquad
1\cdot x \equiv \ 6 \ (\text{mod} \ 7),
\qquad
x \equiv \ 6 \ (\text{mod} \ 7)
\end{equation*}
and
\begin{equation*}
-5 \cdot 3x \equiv -5 \cdot 4 \ (\text{mod} \ 8)
\qquad
-15x \equiv -20 \ (\text{mod} \ 8)
\qquad	
1\cdot x \equiv 4 \ (\text{mod} \ 8)
\qquad	
 x \equiv 4 \ (\text{mod} \ 8).
\end{equation*}
Now, using the construction within the proof of the Chinese Remainder Theorem, we will solve the system of equations, $x \equiv \ 6 \ (\text{mod} \ 7)$,  $x \equiv 4 \ (\text{mod} \ 8)$ which we showed is equivalent to the system given in (5).
Since $\gcd(7,8)=1$, theorem 1.3.6 implies the given system has a solution modulo $7\cdot 8$. The congruence $x \equiv \ 6 \ (\text{mod} \ 7)$ gives us the equation $x=6 + 7k$ for some integer $k$. Then, substituting we obtain $6 + 7k \equiv 4 \ (\text{mod}\ 8)$, or equivalently, $7k \equiv -2 \ (\text{mod}\ 8)$. Multiplying by $7$, Since $7\cdot 7 \equiv 1 \ (\text{mod}\ 8)$, gives us $k \equiv -14 \ (\text{mod}\ 8)$ or $k \equiv 2 \ ( \text{mod}\ 8)$. This yields the particular solution $x=6+7\cdot2=20$. Thus, we write the solution to the given system of equations, $x \equiv 20 \ (\text{mod}\ 56)$.

}

\end{homeworkProblem}
\newpage
\begin{homeworkProblem}
Exercise 1.3.24 of B\&B: Show that the remainder of an integer $n$ when divided by $9$ is the same as the remainder of the sum of its digits when divided by $9$.\\

\problemLAnswer{
\begin{proof}
We will show for any integer $n$ written in decimal form as $n=a_ka_{k-1}...a_1a_0$ satisfies the following equation:
\begin{equation*}
n\  \equiv \ \left( a_k+a_{k-1}+ \ \dots \ +a_1+a_0	\right) \ \text{mod} \ 9.
\end{equation*}
If $n$ has decimal digits $a_ka_{k-1}...a_1a_0$ we can write $n$ in expanded form: 
\begin{equation*}
	n = 10^k\cdot a_k + 10^{k-1}\cdot a_{k-1}+ \ \dots \  +10^1 \cdot a_1+ 10^0 \cdot a_0.
\end{equation*}

 If this equality holds, it must also be valid mod 9: 
  \begin{equation*}
 	n\ \equiv  \left(10^k\cdot a_k + 10^{k-1}\cdot a_{k-1}+...+10^1 \cdot a_1+ 10^0 \cdot a_0 \right)\ \text{mod}\ 9.
 \end{equation*}
 Then, since $10 \equiv 1$ (mod $9)$, we can write
 \begin{equation*}
  	n\ \equiv  \left(1^k\cdot a_k + 1^{k-1}\cdot a_{k-1}+...+1^1 \cdot a_1+ 1^0 \cdot a_0 \right)\ \text{mod}\ 9.
 \end{equation*}
 Any power of $1$ is $1$, so we have
  	\begin{equation*}
  	n\ \equiv  \left(1\cdot a_k + 1\cdot a_{k-1}+...+1 \cdot a_1+ 1 \cdot a_0 \right)\ \text{mod}\ 9.
 \end{equation*}
 
 Because $1$ is the multiplicative identity, we have
  	\begin{equation*}
  	n\ \equiv  \left( a_k + a_{k-1}+...+a_1+ a_0 \right)\ \text{mod}\ 9.
 \end{equation*}
 Therefore, when divided by $9$, the remainder of $n$ is be the same as the remainder of the sum of its digits.
\end{proof}
}


\end{homeworkProblem}
\newpage

\begin{homeworkProblem}
Exercise 1.3.26 of B\&B: let $p$ be a prime number and let $a, b$ be any integers. Prove that $(a+b)^p \ \equiv \ a^p + b^p \ (\text{mod}\ p)$\\

\problemLAnswer{
\begin{proof}
	Let $p$ be a prime number and let $a, b$ be any integers. Using the binomial formula, 
\begin{equation*}
(a+b)^p = \sum_{k=0}^p \binom{p}{k}a^{p-k} \cdot b^k
\end{equation*}

Expanding this binomial we have
\begin{equation*}
	(a+b)^p =  \binom{p}{0}a^{p} \cdot b^0+\binom{p}{1}a^{p-1} \cdot b^1+\binom{p}{2}a^{p-2} \cdot b^2+ \ \dots \ + \binom{p}{p-2}a^{2} \cdot b^{p-2}+\binom{p}{p-1}a^{1} \cdot b^{p-1}+\binom{p}{p}a^{0} \cdot b^{p}.
\end{equation*} 	
Notice
\begin{equation*}
	\binom{p}{0}a^{p}\cdot b^0=a^p 
	\qquad
	\qquad
	\text{and}
\binom{p}{p}a^{0} \cdot b^{p}=b^p
\end{equation*}

 Thus our goal is to show that for all $1 \leq k \leq p-1$, 
 \begin{equation*}
 	\binom{p}{k}a^{p-k} \cdot b^k \ \equiv \ 0 \ (\text{mod}\ p).
 \end{equation*}
Note 
\begin{equation*}
	\binom{p}{k}=\frac{p!}{k!(p-k)!}=\frac{p\cdot (p-1)!}{k!(p-k)!}  = p\cdot\frac{(p-1)!}{k!(p-k)!}.
\end{equation*}

The coefficients $\frac{p!}{k!(p-k)!}$ are known to be integers from the binomial theorem. Also, since $p$ is prime, $\frac{p!}{k!(p-k)!}$ has $p$ as a factor  because $p$ is a divisor of the numerator but not the denominator. Thus  $\frac{(p-1)!}{k!(p-k)!}$ is an integer and $p \ | \ \binom{p}{k}$ for $1 \leq k \leq p-1$. Since $\binom{p}{0}=\binom{p}{p}=1$, the coefficients on $a^p$ and $b^p$ are not divisible by $p$ whereas when $1 \leq k \leq p-1$,  $\binom{p}{k}\ \equiv \ 0 \ (\text{mod}\ p)$. This implies $\binom{p}{k}a^{p-k} \cdot b^k \ \equiv \ 0 \ (\text{mod}\ p)$ when $1 \leq k \leq p - 1$. Thus,

\begin{equation*}
	(a+b)^p \ \equiv \   \binom{p}{0}a^{p} \cdot b^0+ 0 + 0 + \ \dots \ + 0 + 0 +\binom{p}{p}a^{0} \cdot b^{p} \ (\text{mod}\ p)
\end{equation*} 

\begin{equation*}
	\equiv \  a^{p} + b^{p} \ (\text{mod}\ p).
\end{equation*}

\end{proof}
}
\end{homeworkProblem}


\begin{homeworkProblem}
Exercise 1.4.1(b): Make addition and multiplication tables for the set $\mathbb{Z}_4$.
\\

\problemLAnswer{
\vskip .25 in
\begin{center}


\renewcommand\arraystretch{1.3}
\setlength\doublerulesep{0pt}
\begin{tabular}{r||*{4}{2|}}
+ & 0 & 1 & 2 & 3 \\
\hline\hline
0 & 0 & 1 & 2 & 3 \\ 
\hline
1 & 1 & 2 & 3 & 0 \\ 
\hline
2 & 2 & 3 & 0 & 1 \\ 
\hline
3 & 3 & 0 & 1 & 2 \\ 
\hline
\end{tabular}
\hspace{.25 in}
\renewcommand\arraystretch{1.3}
\setlength\doublerulesep{0pt}
\begin{tabular}{r||*{4}{2|}}
$\cdot$ & 0 & 1 & 2 & 3 \\
\hline\hline
0 & 0 & 0 & 0 & 0 \\ 
\hline
1 & 0 & 1 & 2 & 3 \\ 
\hline
2 & 0 & 2 & 0 & 2 \\ 
\hline
3 & 0 & 3 & 2 & 1 \\ 
\hline
\end{tabular}
	
\end{center}
\vskip .25 in
}
\end{homeworkProblem}

\begin{homeworkProblem}

Exercise 1.4.2(a) in B \& B: Make multiplication table for $\mathbb{Z}_6$.
\\
\\
\problemLAnswer{
\vskip .25 in
\begin{center}
\renewcommand\arraystretch{1.3}
\setlength\doublerulesep{0pt}
\begin{tabular}{r||*{6}{2|}}
$\cdot$ & 0 & 1 & 2 & 3 & 4 & 5 \\
\hline\hline
0 & 0 & 0 & 0 & 0 & 0&0\\ 
\hline
1 & 0 & 1 & 2 & 3 & 4 & 5\\ 
\hline
2 & 0 & 2 & 4 & 0 & 2 & 4\\ 
\hline
3 & 0 & 3 & 0 & 3 & 0 & 3\\ 
\hline
4 & 0 & 4 & 2 & 0 & 4 & 2\\ 
\hline
5 & 0 & 5 & 4 & 3 & 2 & 1\\ 
\hline
\end{tabular}
	
\end{center}
\vskip .25 in
}

	
\end{homeworkProblem}

\begin{homeworkProblem}

Exercise 1.4.3(b) in B \& B: Find the multiplicative inverses $[38]$ in $\mathbb{Z}_{83}$.
\\

\problemLAnswer{
Since $83$ is prime, $[38]$ has multiplicative inverses (by corollary 1.4.6). To find $[38]_{83}^{-1}$, we can use the matrix form of the Euclidean algorithm:
\[
  \begin{bmatrix} 1 & 0 & 83 \\ 0 & 1 & 38   \end{bmatrix}
  \begin{matrix}  \xrightarrow{R_1-2R_2} \\ ~ \end{matrix}
  \begin{bmatrix} 1 & -2 & 7 \\ 0 & 1 & 38  \end{bmatrix}
  \begin{matrix} ~ \\ \xrightarrow{R_2+-5R_1} \end{matrix}
  \begin{bmatrix} 1 & -2 & 7 \\ -5 & 11 & 3  \end{bmatrix}
   \begin{matrix}  \xrightarrow{R_1-2R_2} \\ ~ \end{matrix}
  \begin{bmatrix} 11 & -24 & 1 \\ -5 & 11 & 3 \end{bmatrix}
  \begin{matrix} ~ \\ \xrightarrow{R_2+-3R_1} \end{matrix}
  \begin{bmatrix} 11 & -24 & 1 \\ -38 & 83 & 0 \end{bmatrix}
  \].
  
Thus, $11 \cdot 83 + -24 \cdot 38 = 1$, which shows that $[38]_{83}^{-1}=[-24]_{83}=[59]_{83}$.
}

	
\end{homeworkProblem}

\begin{homeworkProblem}

Exercise 1.4.9(a) in B \& B: Let $\gcd(a,n)=1$. The smallest positive integer $k$ such that $a^k \ \equiv \ 1 \ (\text{mod}\ n)$ is called the multiplicative order of $[a]$ in $\mathbb{Z}_n^\times$. Find the multiplicative orders of $[2]$ and $[5]$ in f.
\\

\problemLAnswer{
First, note that $\gcd(5,16)=1$, so we will find the multiplicative order of $[5]$ in $\mathbb{Z}_{16}^\times$ using theorem 1.4.11. Since $\varphi(16)=8$, we know that $5^8 \ \equiv \ 1 \ (\text{mod}\ 16)$. However, by exercise 1.4.10, the multiplicative order of any element of $\mathbb{Z}_{16}^\times$ must divide $\varphi(16)=8$, so we must test $1, 2, 4, 8$:
\begin{equation*}
	5^1 \ \equiv \ 5 \ (\text{mod}\ 16), \qquad 5^2 \ \equiv \ 9 \ (\text{mod}\ 16), \qquad 5^4 \ \equiv \ (5^2)^2 \ \equiv \ 9^2 \ \equiv \ 81 \ \equiv \ 1 \ (\text{mod}\ 16).
\end{equation*}
Hence the multiplicative order of $[5]$ in $\mathbb{Z}_{16}^\times$ is $4$. \\
\\
Next, find the multiplicative order of $[7]$ in $\mathbb{Z}_{16}^\times$ by testing $1, 2, 4, 8$:

\begin{equation*}
	7^1 \ \equiv \ 7 \ (\text{mod}\ 16), \qquad 7^2 \ \equiv \ 49 \ \equiv \ 1 \ (\text{mod}\ 16).
\end{equation*}
Hence the multiplicative order of $[7]$ in $\mathbb{Z}_{16}^\times$ is $2$. 
}	
\end{homeworkProblem}
\newpage

\begin{homeworkProblem}

Exercise 1.4.14 from B \& B: If $p$ is a prime number, show that $[0]$ and $[1]$ are the only idempotent elements in $\mathbb{Z}_p$.
\\

\problemLAnswer{
\begin{proof}
Note that $[0]$ is trivially idempotent since $0$ times any integer in $\mathbb{Z}_p$ must be zero, so $[0]^2=[0]$. Suppose there exists some $a \in \mathbb{Z}_p$ such that $ \lbrack a \rbrack ^2=[a]$ but $a > 1$. Since $ \lbrack a \rbrack ^2=[a]$, $a^2 \ \equiv \ a \ (\text{mod} \ p)$. Because $p$ is prime, all nonzero elements in $\mathbb{Z}_p$ have a multiplicative inverse. Thus, there exists some integer $a^{-1}$ such that $a^{-1}\cdot a \ \equiv \ 1 \ (\text{mod} \ p)$. Multiply both sides of the congruence $a^2 \ \equiv \ a \ (\text{mod} \ p)$ by $a^{-1}$ to obtain $a^{-1} \cdot a^2 \ \equiv \ a^{-1} \cdot a \ (\text{mod} \ p)$. Equivalently, $a^{-1} \cdot a \cdot a \ \equiv \ 1 \ (\text{mod} \ p)$ and $1\cdot a \ \equiv \ 1 \ (\text{mod} \ p)$. Thus, $ a \ \equiv \ 1 \ (\text{mod} \ p)$ which contradicts our assumption that $a$ is in $\mathbb{Z}_p$ but $a>1$. Thus, $[0]$ and $[1]$ are the only idempotent elements in $\mathbb{Z}_p$.
\end{proof}
}	
\end{homeworkProblem}

\begin{homeworkProblem}

Exercise 1.4.15 from B \& B: If $n$ is not a prime power, show that $\mathbb{Z}_n$ has an idempotent element different from $[0]$ and $[1]$.
\\

\problemLAnswer{
\begin{proof}
Assume $n$ is not a prime power. Thus, $n$ must have more than one prime factor so there must exist integers $b$ and $c$ such that $b \ | \ n$, $c \ | \ n$, $n = bc$, and $\gcd(b, c) = 1$. Because $\gcd(b,c)=1$, the Chinese Remainder Theorem implies a solution, $x$, exists mod $bc$ to the following system of congruences:
\begin{equation*}
	x \equiv \ 1 \ (\text{mod}\ b), \qquad x \equiv \ 0 \ (\text{mod}\ c).
\end{equation*}
\textbf{Claim 1:} $x \not \equiv 0 \MOD bc)$
\begin{proof}
	 If $x  \equiv 0 \MOD bc)$, $bc \ | \ x$ implies $b \ | \ x$, but $x \equiv \ 1 \ (\text{mod}\ b)$.
\end{proof}
\textbf{Claim 2:} $x \not \equiv 1 \MOD bc)$
\begin{proof}
	 If $x  \equiv 1 \MOD bc)$, $bc \ | \ (x-1)$ implies $c \ | \ (x-1)$, but $x \equiv \ 0 \ (\text{mod}\ c)$.
\\
\end{proof}

Notice if $x \equiv \ 1 \ (\text{mod}\ b)$ and $x \equiv \ 0 \ (\text{mod}\ c)$, $x=1 + bk$ and $x=cm$ for some integers $k, m$. If we multiply both sides of the equation $x=1 + bk$ by $x$ we obtain $x\cdot x= (1 + bk)x$. This implies $x^2= 1\cdot x + (bk)\cdot x$. Since $x=cm$ t, $x^2= x + (bk)\cdot cm$. Thus, $x^2 \ \equiv \ x \ (\text{mod}\ bc)$. 

\end{proof}
}	
\end{homeworkProblem}
\newpage

\begin{homeworkProblem}

Exercise 1.4.16 from B \& B: An element $[a]$ of $\mathbb{Z}_n$ is said to be nilpotent if $[a]^k=[0]$ for some $k$. Show that $\mathbb{Z}_n$ has no nonzero nilpotent elements if and only if $n$ has no factor that is a square (except 1).
\\

\problemLAnswer{
\begin{proof}
First, assume $n$ has no factor that is a square. Then, write the prime factorization of $n$:
\begin{equation*}
n=\prod_{i=1}^{m}p_i^{\alpha_i}, \qquad \text{ where } p_i \text{ are prime, } \alpha_i, \text{ are in } \Z^+.
\end{equation*}
But, $n$ has no square factors, so we can conclude that all $\alpha_i=1$:
\begin{equation*}
n=\prod_{i=1}^{m}p_i \qquad \text{ where } p_i \text{ are prime.} \end{equation*}
We want to show that $\Z_n$ has no nonzero nilpotent elements. Suppose that $\Z_n$ has some nonzero nilpotent element, $[a]$, $[a] \neq [0]$. Then, $a^k\ \equiv \ 0 \ (\text{mod} \ n)$. This implies $n \ | \ a^k$. Equivalently, $\prod_{i=1}^{m}p_i \ | \ a^k$. Then, there exists some integer $b$ such that $a^k=b\cdot (\prod_{i=1}^{m}p_i)$. So, for any $1\leq j \leq m$, 
\begin{equation*}
	a^k=p_j \cdot b \cdot  \prod_{i=1}^{j-1}p_i\cdot  \prod_{i=j+1}^{m}p_i.
\end{equation*}

Thus for all $1\leq j \leq m$, $p_j \ | \ a^k$. Equivalently, for all $1\leq j \leq m$, $p_j \ | \ a\cdot a^{k-1}$. So by corollary 1.2.6,  we can inductively conclude that for all $1\leq j \leq m$, $p_j \ | \ a$. Hence, $\prod_{i=1}^m p_i \ | \ a$. This contradicts our assumption that $[a] \neq [0]$. Therefore, $\Z _n$ has no nonzero nilpotent elements. \\
\\


Next, assume $\mathbb{Z}_n$ has no nonzero nilpotent elements. Then, there are no $[a] \neq [0]$ such that $a^k \equiv 0 \MOD n)$. Suppose $n$ has some square factor, $s^2\neq 1$, so $n=t\cdot s^2$. Note $[st] \in \Z_n$ and $[st]\neq [0]$ since $n \nmid \ st$; but $(ts)^2 \equiv 0 \MOD n)$. Thus, if $\Z_n$ has no nonzero nilpotent elements, $n$ has no square factors.


\end{proof}
}	
\end{homeworkProblem}

\begin{homeworkProblem}
Exercise 1.4.17 from B \& B: Compute $\varphi(27), \varphi(81), \varphi(p^\alpha)$\\
\\
\sectionAnswer{
Using the formula in proposition 1.4.8, since $27=3^3$, $\varphi(27)=27\left(1-\frac{1}{3}\right)=18$.\\
Similarly, $81 = 3^4$, so $\varphi(81) = 81\left(1-\frac{1}{3}\right)=54$.
\\
Finally, $\varphi(p^\alpha)=p^\alpha\left(1-\frac{1}{p}\right)=p^\alpha\left( \frac{p-1}{p}\right)=p^{\alpha -1}(p-1)=p^\alpha-p^{\alpha -1}$.
}
\newpage

Give a proof that the formula for $\varphi(n)$ is valid when $n = p^\alpha$.
\\
\\
\sectionAnswer{

To calculate $\varphi(p^\alpha)$, we need to count the number of integers from the set $\Z_{p^\alpha}$ that are relatively prime to $p^{\alpha}$. Note $|\Z_{p^\alpha}|=p^\alpha$. To find $\varphi(p^\alpha)$, we will first find all integers in $\Z_{p^\alpha}$ that are not relatively prime to $p^{\alpha}$. Consider $m \in \Z_{p^\alpha}$ such that $\gcd(m,p^\alpha)\neq 1$. To count how many $m$ exist, we will prove the following lemma.


\begin{lemma}
The following statements are equivalent when $p$ is prime, $\alpha \in \mathbb{Z}^+$, $m \in \mathbb{Z}^+$ and $1 \leq k < p^\alpha$: 
\begin{center}
	i. $\gcd(k, p)=1$ \\
ii. $\gcd(k, p^\alpha)=1$ \\
iii.  $k$ is not a multiple of $p$ \\
\end{center}
\end{lemma}

\begin{proof}
	To show these statements are equivalent, it suffices to prove the following conditional statements:
\begin{center}
(1) If $\gcd(k, p)=1$ , then $\gcd(k, p^\alpha)=1$. \\
(2) If $\gcd(k, p^\alpha)=1$, then $k$ is not a multiple of $p$.\\
(3) If $k$ is not a multiple of $p$, then $\gcd(k, p)=1$.
\end{center}
(1). Assume $\gcd(k, p)=1$. Note if $\alpha = 2$, proposition 1.2.3 (d) implies $\gcd(k, p^2)=1$. Now, assume when $\alpha = n$, $\gcd(k, p^n)=1$. If $\gcd(k, p^n)=1$ and $\gcd(k, p)=1$, proposition 1.2.3 (d) implies $\gcd(k, p^{n+1})=1$. Thus, by the principle of induction, $\gcd(k, p^\alpha)=1$ for any $\alpha \in \Z^+$. 
\\
\\
(2). Assume $\gcd(k, p^\alpha)=1$. Then, by theorem 1.1.6 there exists integers $x, y$ such that $kx + p^\alpha y = 1$. Equivalently, $kx + p(p^{\alpha - 1}y)=1$. Again, by theorem 1.1.6, this implies $\gcd(k, p) = 1$. Then, $p \nmid k$ so $k$ is not a multiple of $p$. 
\\
\\
(3). Assume $k$ is not a multiple of $p$. Then, $p \nmid k$. The only divisors of $p$ are $1$ and $p$ so if $p \nmid k$, $\gcd(k, p)=1.$
\\
\\
\\
By lemma 0.1, we know the following statements are equivalent:
\begin{center}
	i. $\gcd(k, p)\neq 1$ \\
ii. $\gcd(k, p^\alpha) \neq 1$ \\
iii.  $k$ is a multiple of $p$. \\
\end{center}
Thus, the only $m \in \Z_{p^\alpha}$ such that $\gcd(m,p^\alpha)\neq 1$, are multiples of $p$: $0, p, 2p, 3p, ..., p^{\alpha -1}p$. So every $p^{\text{th}}$ integer in $\{0, 1, 2, ..., p^\alpha - 1\}$ is a multiple of $p$. Thus, there are $p^\alpha/p=p^{\alpha - 1} m \in \Z_{p^\alpha}$ such that $\gcd(m,p^\alpha)\neq 1$. \\
\\
Therefore, $\varphi(n)=p^\alpha -p^{\alpha - 1}$. 
\end{proof}




}
	
\end{homeworkProblem}

\begin{homeworkProblem}
Exercise 1.4.24 from B \& B: Show that if $p$ is a prime number, then the congruence $x^2 \ \equiv \ 1 \  (\text{mod}\ p)$ has only the solutions $x \ \equiv \ 1 \  (\text{mod}\ p)$ and $x \ \equiv \ -1 \  (\text{mod}\ p)$.\\
\\
\sectionAnswer{
\begin{proof}
	Let $p$ be a prime number and let $x \in \mathbb{Z}_p$ such that $x^2 \ \equiv \ 1 \  (\text{mod}\ p)$. Note that $x \ \equiv \ 1$ and $x \ \equiv \ -1$ satisfy the given congruence since $1^2 \ \equiv 1 \ (\text{mod}\ p)$ and $(-1)^2 \ \equiv 1 \ (\text{mod}\ p)$. Now, suppose there exists some other integer $a$ such that $a^2 \ \equiv \ 1 \  (\text{mod}\ p)$ but $a \not\equiv  1, -1$. Then, $a^2 - 1 \ \equiv \ 0 \ (\text{mod}\ p )$, or equivalently, $(a - 1)(a+1) \ \equiv \ 0 \ (\text{mod}\ p )$. Thus, $p \ | \ (a-1)(a+1)$.  By corollary 1.2.6 in B \& B, if $p \ | \ (a-1)(a+1)$, then $p \ | \ (a-1)$ or $p \ | \ (a+1)$. if $p \ | \ (a-1)$, then $a - 1 \ \equiv 0 \ (\text{mod}\ p)$ which implies $a \ \equiv 1 \ (\text{mod}\ p)$ which contradicts our assumption that $a \not\equiv 1$. If $p \ | \ (a+1)$, $a + 1 \ \equiv 0 \ (\text{mod}\ p)$ which implies $a \ \equiv -1 \ (\text{mod}\ p)$ which contradicts our assumption that $a \not\equiv -1$. Thus, the congruence $x^2 \ \equiv \ 1 \  (\text{mod}\ p)$ has only the solutions $x \ \equiv \ 1 \  (\text{mod}\ p)$ and $x \ \equiv \ -1 \  (\text{mod}\ p)$. 	\end{proof}
}

\end{homeworkProblem}


\begin{homeworkProblem}
Exercise 1.4.27 from B \& B: Show that if $p$ is a prime number, then $(p-1)! \ \equiv \ -1 \  (\text{mod}\ p)$.\\
\\
\sectionAnswer{
\begin{proof}
	Let $p$ be a prime number. Note that when $p=2$ and $p=3$, $(2-1)! \ \equiv \ 1! \ \equiv \ -1\  (\text{mod}\ 2)$ and $(3-1)! \ \equiv \ 2! \ \equiv \ \equiv \ 2 \ \equiv \ -1\  (\text{mod}\ 3)$. We have shown the given congruence holds for $p=2, 3$, so we will consider only $p>3$ and so that $p$ is odd. Consider $[a]_p$. Since every integer $1 \leq a \leq p-1$ is relatively prime to $p$, all $[a]_p$ have unique multiplicative inverses in $\mathbb{Z}_p$. Thus, $(p-1)!$ is the product of all elements in $\mathbb{Z}_p^\times $. So for all $[a]_p$ we can find a unique $[a]_p^{-1}$. Note that the only cases when $[a]_p^{-1}=[a]_p$ can be found by applying exercise 24 in section 1.4 of B \& B: if the only solutions to $a^2 \ \equiv \ 1 \ (\text{mod}\ p)$ are $\pm 1$, then $a^{-1}a^2 \ \equiv \ a^{-1}1 \ (\text{mod}\ p)$ implies $\pm 1$ are the only solutions to $a \ \equiv \ a^{-1} \ (\text{mod}\ p)$. Thus for all $a \neq 1, p-1$ in $\mathbb{Z}_p$ there exists a unique $a^{-1}$ in $\mathbb{Z}_p$ with $a \neq a^{-1}$. Then all $2 \leq a \leq p-2$ must have a multiplicative inverse $2 \leq a^{-1} \leq p-2$. Consider the product $2\cdot 3 \cdot 4 \cdots (p-3)(p-2)$ then rearrange and group this product so that each element is multiplied by its multiplicative inverse so that $2\cdot 3 \cdot 4 \cdots (p-3)(p-2) \equiv \ 1 \ (\text{mod}\ p)$. Then, multiply both sides by $p-1$ to obtain $2\cdot 3 \cdot 4 \cdots (p-3)(p-2)(p-1) \equiv \ 1(p-1) \ (\text{mod}\ p)$, or equivalently, $1\cdot 2\cdot 3 \cdot 4 \cdots (p-3)(p-2)(p-1) \equiv \ -1 \ (\text{mod}\ p)$. Thus, $(p-1)! \ \equiv \ -1 \  (\text{mod}\ p)$ when $p$ is prime.

	\end{proof}
}
\end{homeworkProblem}

\end{spacing}
\end{document}

%%%%%%%%%%%%%%%%%%%%%%%%%%%%%%%%%%%%%%%%%%%%%%%%%%%%%%%%%%%%%

%----------------------------------------------------------------------%
% The following is copyright and licensing information for
% redistribution of this LaTeX source code; it also includes a liability
% statement. If this source code is not being redistributed to others,
% it may be omitted. It has no effect on the function of the above code.
%----------------------------------------------------------------------%
% Copyright (c) 2007, 2008, 2009, 2010, 2011 by Theodore P. Pavlic
%
% Unless otherwise expressly stated, this work is licensed under the
% Creative Commons Attribution-Noncommercial 3.0 United States License. To
% view a copy of this license, visit
% http://creativecommons.org/licenses/by-nc/3.0/us/ or send a letter to
% Creative Commons, 171 Second Street, Suite 300, San Francisco,
% California, 94105, USA.
%
% THE SOFTWARE IS PROVIDED "AS IS", WITHOUT WARRANTY OF ANY KIND, EXPRESS
% OR IMPLIED, INCLUDING BUT NOT LIMITED TO THE WARRANTIES OF
% MERCHANTABILITY, FITNESS FOR A PARTICULAR PURPOSE AND NONINFRINGEMENT.
% IN NO EVENT SHALL THE AUTHORS OR COPYRIGHT HOLDERS BE LIABLE FOR ANY
% CLAIM, DAMAGES OR OTHER LIABILITY, WHETHER IN AN ACTION OF CONTRACT,
% TORT OR OTHERWISE, ARISING FROM, OUT OF OR IN CONNECTION WITH THE
% SOFTWARE OR THE USE OR OTHER DEALINGS IN THE SOFTWARE.
%----------------------------------------------------------------------%
