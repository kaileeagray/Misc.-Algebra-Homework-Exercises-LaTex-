\documentclass[12 pt]{article}
% Change "article" to "report" to get rid of page number on title page
\usepackage{amsmath,amsfonts,amsthm,amssymb}
\usepackage{setspace}
\usepackage{Tabbing}
\usepackage{fancyhdr}
\usepackage{lastpage}
\usepackage{extramarks}
\usepackage{chngpage}
\usepackage{indentfirst}
\usepackage{soul,color}
\usepackage{graphicx,float,wrapfig}
\usepackage{gauss}
\usepackage{dcolumn}
\newcolumntype{2}{D{.}{}{2.0}}
\usepackage{multicol}
\usepackage{Tabbing}
\usepackage{fancyhdr}
\usepackage{lastpage}
\usepackage{extramarks}
\usepackage{enumerate}
\usepackage{mathtools}
\usepackage{tikz}
\usetikzlibrary{positioning}
\usepackage{float}
\usepackage{wrapfig}



\graphicspath{ {/home/user/Documents/} }

\makeatletter
\renewcommand\section{\@startsection{section}{1}{\z@}%
                                  {-3.5ex \@plus -1ex \@minus -.2ex}%
                                  {2.3ex \@plus.2ex}%
                                  {\normalfont\bfseries}
                                }
\makeatother

\usepackage{etoolbox}
\makeatletter
\patchcmd\g@matrix
 {\vbox\bgroup}
 {\vbox\bgroup\normalbaselines}% restore the standard baselineskip
 {}{}
\makeatother

% In case you need to adjust margins:
\topmargin=-0.75in      %
\evensidemargin=0in     %
\oddsidemargin=0in      %
\textwidth=6.5in        %
\textheight=9.0in       %
\headsep=0.25in         %

% Homework Specific Information
\newcommand{\hmwkTitle}{$\S3.8$ Cosets, Normal Subgroups, and Factor Groups}
\newcommand{\hmwkDueDate}{Monday,\ November\ 16,\ 2015}
\newcommand{\hmwkClass}{Math\ 620}
\newcommand{\hmwkClassTime}{10:00}
\newcommand{\hmwkClassInstructor}{Boynton}
\newcommand{\hmwkAuthorName}{Kailee\ Gray}

\newtheorem{theorem}{Theorem}[section]
\newtheorem{lemma}[theorem]{Lemma}
\newtheorem{proposition}[theorem]{Proposition}
\newtheorem{corollary}[theorem]{Corollary}


\newenvironment{definition}[1][Definition]{\begin{trivlist}
\item[\hskip \labelsep {\bfseries #1}]}{\end{trivlist}}
\newenvironment{example}[1][Example]{\begin{trivlist}
\item[\hskip \labelsep {\bfseries #1}]}{\end{trivlist}}
\newenvironment{remark}[1][Remark]{\begin{trivlist}
\item[\hskip \labelsep {\bfseries #1}]}{\end{trivlist}}




% Setup the header and footer
\pagestyle{plain}                                                       %
\lhead{\hmwkAuthorName}                                                 %
\chead{\hmwkClass\ (\hmwkClassInstructor\ \hmwkClassTime): \hmwkTitle}  %
\rhead{\firstxmark}                                                     %
\lfoot{\lastxmark}                                                      %
\cfoot{}                                                                %
\rfoot{Page\ \thepage\ of\ \pageref{LastPage}}                          %
\renewcommand\headrulewidth{0.4pt}                                      %
\renewcommand\footrulewidth{0.4pt}                                      %

% This is used to trace down (pin point) problems
% in latexing a document:
%\tracingall

%%%%%%%%%%%%%%%%%%%%%%%%%%%%%%%%%%%%%%%%%%%%%%%%%%%%%%%%%%%%%
% Some tools
\newcommand{\enterProblemHeader}[1]{\nobreak\extramarks{#1}{#1 continued on next page\ldots}\nobreak%
                                    \nobreak\extramarks{#1 (continued)}{#1 continued on next page\ldots}\nobreak}%
\newcommand{\exitProblemHeader}[1]{\nobreak\extramarks{#1 (continued)}{#1 continued on next page\ldots}\nobreak%
                                   \nobreak\extramarks{#1}{}\nobreak}%

\newlength{\labelLength}
\newcommand{\labelAnswer}[2]
  {\settowidth{\labelLength}{#1}%
   \addtolength{\labelLength}{0in}%
   \changetext{}{-\labelLength}{}{}{}%
   \noindent\fbox{\begin{minipage}[c]{\columnwidth}#2\end{minipage}}%
   \marginpar{\fbox{#1}}%

   % We put the blank space above in order to make sure this
   % \marginpar gets correctly placed.
   \changetext{}{+\labelLength}{}{}{}}%

\setcounter{secnumdepth}{0}
\newcommand{\homeworkProblemName}{}%
\newcounter{homeworkProblemCounter}%
\newenvironment{homeworkProblem}[1][\arabic{homeworkProblemCounter}]%
  {\stepcounter{homeworkProblemCounter}%
   \renewcommand{\homeworkProblemName}{#1}%
   \section{\homeworkProblemName}%
   \noindent 
   
   \enterProblemHeader{\homeworkProblemName}}%
  {\exitProblemHeader{\homeworkProblemName}}%

\newcommand{\problemAnswer}[1]
  {\noindent\begin{minipage}[c]{\columnwidth}#1\end{minipage}}%

\newcommand{\problemLAnswer}[1]
    {\noindent\begin{minipage}[c]{\columnwidth}#1\end{minipage}}%

\newcommand{\homeworkSectionName}{}%
\newlength{\homeworkSectionLabelLength}{}%
\newenvironment{homeworkSection}[1]%
  {% We put this space here to make sure we're not connected to the above.
   % Otherwise the changetext can do funny things to the other margin

   \renewcommand{\homeworkSectionName}{#1}%
   \settowidth{\homeworkSectionLabelLength}{\homeworkSectionName}%
   \addtolength{\homeworkSectionLabelLength}{0 in}%
   \changetext{}{-\homeworkSectionLabelLength}{}{}{}%
   \subsection{\homeworkSectionName}%
   \enterProblemHeader{\homeworkProblemName\ [\homeworkSectionName]}}%
  {\enterProblemHeader{\homeworkProblemName}%

   % We put the blank space above in order to make sure this margin
   % change doesn't happen too soon (otherwise \sectionAnswer's can
   % get ugly about their \marginpar placement.
   \changetext{}{+\homeworkSectionLabelLength}{}{}{}}%

\newcommand{\sectionAnswer}[1]
  {\noindent\begin{minipage}[c]{\columnwidth}#1\end{minipage}}%
   \enterProblemHeader{\homeworkProblemName}\exitProblemHeader{\homeworkProblemName}%
  
 \newcommand{\R}{{\mathbb R}}
          \newcommand{\nil}{\varnothing}
          \newcommand{\N}{{\mathbb N}}
          \newcommand{\Z}{{\mathbb Z}}
        \newcommand{\MOD}{{ \ (\text{mod} \ }}

 \newcommand{\C}{{\mathbb C}}
  \newcommand{\Q}{{\mathbb Q}}
  
%%%%%%%%%%%%%%%%%%%%%%%%%%%%%%%%%%%%%%%%%%%%%%%%%%%%%%%%%%%%%
% Make title
\title{\vspace{2in}\textmd{\textbf{\hmwkClass:\ \hmwkTitle}}\\\normalsize\vspace{0.1in}\small{Due\ on\ \hmwkDueDate}\\\vspace{0.1in}\large{\textit{\hmwkClassInstructor\ \hmwkClassTime}}\vspace{3in}}
\date{}
\author{\textbf{\hmwkAuthorName}}
%%%%%%%%%%%%%%%%%%%%%%%%%%%%%%%%%%%%%%%%%%%%%%%%%%%%%%%%%%%%%

\begin{document}
\begin{spacing}{1.75}
\maketitle

 
% Uncomment the \tableofcontents and   lines to get a Contents page
% Uncomment the \setcounter line as well if you do NOT want subsections
%       listed in Contents
%\setcounter{tocdepth}{1}
%\tableofcontents
% 

% When problems are long, it may be desirable to put a   or a
% \clearpage before each homeworkProblem environment


\begin{flushleft}

\begin{homeworkProblem} [Exercise 3.8.4: For each of the subgroups $H=\{e, r^2\}$ and $K=\{e, s\}$ of $D_4$, list all left and right cosets. ]
\problemAnswer{ 
($H=\{e, r^2\}$): Note $[D_4:H]=|D_4|\slash |H|=4$ and $r^2H=H$. Then,
\begin{eqnarray*}
rH&=&rr^2H=r^3H \\	
sH & = & sr^2H \\
srH & = & srr^2H = sr^3H \\
\text{Therefore, } \mathcal{L}(H) & = & \{ H, rH, sH, srH \} \\
Hr^2 & = & H \\
Hr&=&Hr^2r=Hr^3 \\	
Hs & = & Hr^2s=Hsr^2 \\
Hsr & = & Hr^2sr = Hsr^3 \\
\text{Therefore, } \mathcal{R}(H) & = & \{ H, Hr, Hs, Hsr \} 
\end{eqnarray*}
($K=\{e, s\}$):  Note $[D_4:K]=|D_4|\slash |K|=4$ and $sK=K$. Then,
\begin{eqnarray*}
rK&=&rsK=sr^3K \\	
r^2K & = & r^2sK=sr^2K \\
srK & = & srsK = r^3K \\
\text{Therefore, } \mathcal{L}(H) & = & \{ K, rK, r^2K, srK \} \\
Ks & = & K \\
Kr&=&Ksr \\	
Ksr^2 & = & Kssr^2=Kr^2 \\
Ksr^3 & = & Kssr^3 = Kr^3 \\
\text{Therefore, } \mathcal{R}(H) & = & \{ K, Kr, Ksr^2, Ksr^3 \} 
\end{eqnarray*}}
\end{homeworkProblem}
\begin{homeworkProblem}[Exercise 3.8.6: Prove that if $N$ is a normal subgroup of $G$, and $H$ is any subgroup of $G$, then $H \cap N$ is a normal subgroup of $H$. ]
\problemAnswer{
\begin{proof}
Assume $N$ is a normal subgroup of $G$. Let $H$ is any subgroup of $G$. By exercise 17 in section 3.2, $H \cap N$ is a subgroup of $G$. So $H \cap N \neq \O$. Also $H \cap N \subseteq H$. Let $a, b \in H \cap N$. Since $H \cap N$ is a subgroup of $G$, corollary 3.2.3 implies $ab^{-1} \in H \cap N$ for all $a, b \in H \cap N$. Thus, $H \cap N$ is a subgroup of $H$. Next, consider any $a \in H$; then, $a \in G$. Let $x \in a(H \cap N)a^{-1}$. So, $x=ana^{-1}$ for some $n \in H \cap N$. Notice $a(H \cap N)a^{-1}=aHa^{-1}\cap aNa^{-1}$:
\[
x=ana^{-1} \Leftrightarrow n \in H \text{ and } n \in N \Leftrightarrow ana^{-1} \in aHa^{-1} \text{ and } ana^{-1} \in aNa^{-1} \Leftrightarrow x \in aHa^{-1}\cap aNa^{-1}.
\]
 Since $N$ is a normal subgroup of $G$, $aNa^{-1}\subset N$ which implies $x \in N$. Also, since $a \in H$, $aHa^{-1} \subseteq H$ so $x \in aHa^{-1}$ implies $x \in H$. Thus, $x \in H \cap N$. Hence $a(H \cap N)a^{-1} \subseteq H \cap N$ for all $a \in H$. Thus, $H \cap N$ is a normal subgroup of H by Theorem 13 of Boynton.	
\end{proof}
}
\end{homeworkProblem}

\begin{homeworkProblem}[Exercise 3.8.9: Let $G$ be a finite group, and let $n$ be a divisor of $|G|$. Show if $H$ is the only subgroup of $G$ of order $n$, then $H$ must be normal in $G$.]
 
 \begin{lemma}
 For any subgroup $H$ of $G$ and $a \in G$, $aHa^{-1}$ is a subgroup of $G$.	
 \end{lemma}
\begin{proof}
Let $H \leq G$ and $a \in G$. Since $H \leq G$, $H$ contains the identity element of $G$ so $aea^{-1} \in aHa^{-1}$ and $aea^{-1}=aa^{-1}=e$. Thus, $aHa^{-1}$ contains the identity element and is nonempty. Since $a \in G$ and $H \subseteq G$, $aHa^{-1}\subseteq G$. Consider $ah_1a^{-1}, ah_2a^{-1} \in H$. Then,
\[
	(ah_1a^{-1})(ah_2a^{-1})^{-1}  = ah_1a^{-1}(h_2a^{-1})^{-1}a^{-1} =ah_1a^{-1}ah_2^{-1}a^{-1}  =ah_1h_2^{-1}a^{-1}. \text{ Since } h_1h_2^{-1} \in H, 
\]
$(ah_1a^{-1})(ah_2a^{-1})^{-1} \in aHa^{-1}$, corollary 3.2.3 implies $aHa^{-1}\leq G$. 
 \end{proof} 
 
 \begin{proof}
 Let $G$ be a finite group, and let $n$ be a divisor of $|G|$. Assume $H$ is the only subgroup of $G$ of order $n$. By lemma 0.1 above, $aHa^{-1}\leq G$. Since $H, aHa^{-1} \leq G$ and $G$ has finite order, $H, aHa^{-1}$ must have finite order.   
 Define $\phi_a: H \rightarrow aHa^{-1}$ where $a$ is any fixed element of $G$ and $\phi_a(x)=axa^{-1}$. \\
 \textbf{(well-defined)} Since $H \leq G$, for any $x \in H$, $axa^{-1} \in aHa^{-1}$ and if $x_1=x_2$ then $ax_1=ax_2$ and so $ax_1a^{-1}=ax_2a^{-1}$.\\
 \textbf{(onto)} Consider any $aha^{-1} \in aHa^{-1}$. Then, $h \in H$, so $\phi_a(h)=aha^{-1}$.\\
 \textbf{(1-1)} Suppose $ax_1a^{-1}=ax_2a^{-1}$. Then, left and right cancellation imply $x_1=x_2$.\\
 Thus, $\phi_a$ is a bijection between two sets of finite order. Therefore, $|H|=|aHa^{-1}|$. Since $H$ is the only subgroup of order $n$ and $|aHa^{-1}|=n$ it must be the case that $aHa^{-1}=H$. Therefore by theorem 13(3) of Boynton, $H$ must be normal in $G$. 
  \end{proof} 

\end{homeworkProblem}
 
\begin{homeworkProblem}[Exercise 3.8.10: Let $N$ be a normal subgroup of index $m$ in $G$. Show that $a^m \in N$ for all $a \in G$.]
\problemAnswer{
\begin{proof}
Let $N$ be a normal subgroup of index $m$ in $G$. By theorem 16 of Boynton, $|G \slash N|=m$. Using the identity element of $G \slash N$ given in the proof of theorem 16 of Boynton, since $aN \in G \slash N$, we have $(aN)^m=e_{G \slash N}=eN=N$. By corollary 17, part a, of Boynton, $(aN)^m=a^mN$. Thus, $a^mN=N$ which implies $a^m \in N$ by corollary 4 of Boynton. 
\end{proof}}
\end{homeworkProblem}
\begin{homeworkProblem}[Exercise 3.8.11: Let $N$ be a normal subgroup of $G$. Show that the order of any coset $aN$ in $G\slash N$ is a divisor of $o(a)$, when $o(a)$ is finite. ] 
\begin{proof}
 Let $N$ be a normal subgroup of $G$. Consider any coset $aN$ in $G\slash N$ with $o(a)$ finite. Suppose $o(a)=n$. By example 3.8.5 in Beachy, the order of $aN$ is the smallest positive integer $k$ such that $a^k \in N$. Notice $(aN)^n=a^nN=e_{G\slash N}$. Since $G \slash N$ is a group, the order of $aN$ must divide $n=o(a)$. 
 \end{proof}
\end{homeworkProblem}

\begin{homeworkProblem}[Exercise 3.8.12: Let $H$ and $K$ be normal subgroups of $G$ such that $H \cap K = \left< e\right>$. Show that $hk=kh$ for all $h \in H$ and $k \in K$. ]
\begin{proof}
Let $H$ and $K$ be normal subgroups of $G$ such that $H \cap K = \left< e\right>$. Consider $h \in H$ and $k \in K$. Since $H, K$ are subgroups, $h^{-1} \in H$ and $k^{-1} \in K$. Also, $H$ and $K$ are normal, so for all $a \in G$, $aHa^{-1} \subset H$ and $aKa^{-1} \subset K$. Then,  $H,K \subset G$, so $kh^{-1}k^{-1}, khk^{-1} \in H$ and $hk^{-1}h^{-1}, hkh^{-1} \in K$. By closure of $H, K$ $hkh^{-1}, k^{-1} \in K$ implies $hkh^{-1}k^{-1} \in K$ and $kh^{-1}k^{-1}, h\in H$ implies $hkh^{-1}k^{-1} \in H$. Therefore $hkh^{-1}k^{-1} \in H \cap K$. Since $H \cap K =\left< e\right>, hkh^{-1}k^{-1}=e$ and $hkh^{-1}=k$. Thus, $hk=kh$ for any $h \in H, k \in K$.
\end{proof}

\end{homeworkProblem}

\begin{homeworkProblem}[Exercise 3.8.13: Let $N$ be a normal subgroup of $G$. Prove that $G \slash N$ is abelian if and only if $N$ contains all elements of the form $aba^{-1}b^{-1}$ for $a, b \in G$. ]
\begin{proof}
$(\Rightarrow)$ Let $N$ be a normal subgroup in $G$. Assume $G \slash N$ is abelian. Then, for all $a, b \in G$, $aNbN=bNaN$ so $abN=baN$. Thus, $N=(ba)^{-1}abN=(ba)^{-1}NabN=$\\$abN(ba)^{-1}N=ab(ba)^{-1}N$. $N=ab(ba)^{-1}N$, so by corollary 4 of Boynton $ab(ba)^{-1} \in N$. Hence, $aba^{-1}b^{-1} \in N$ for all $a, b \in G$.
\\
$(\Leftarrow)$ Let $N$ be a normal subgroup of $G$. Assume $N$ contains all elements of the form $aba^{-1}b^{-1}$ for $a, b \in G$. Then, by corollary 4 of Boynton, $aba^{-1}b^{-1}=ab(ba)^{-1} \in N$ implies $ab(ba)^{-1}N=N$ and so $(ba)^{-1}N=(ab)^{-1}N$. By definition of inverse element given in proof of theorem 16 (Boynton), $(ba)^{-1}N=(baN)^{-1}=(abN)^{-1} =(ab)^{-1}N$. $G\slash N$ is a group so the inverse of each element in $G \slash N$ is unique, so $baN=abN$. Thus, $bNaN=aNbN$. 
 	\end{proof}
\end{homeworkProblem}


\begin{homeworkProblem}[Exercise 3.8.14: Let $N$ be a subgroup of the center of $G$. Show that if $G \slash N$ is a cyclic group, then $G$ must be abelian. ]
\begin{proof}
Let $N \leq Z(G)$. Then, $Z(G) \leq G$ implies $N \leq G$. Suppose $G \slash N$ is a cyclic group. Then, $G \slash N = \left< gN \right>$ for some $g \in G$. 
Let $a, b \in G$. Then, $aN=g^kN$ and $bN = g^hN$ for some $h, k \in \Z$. By theorem 3 part 1, $g^{-h}a, g^{-k}b \in N$. Thus, there exists $n_1, n_2 \in N$ such that $g^{-h}a=n_1$ and $g^{-k}b = n_2$. Since $n_1, n_2 \in Z(G)$ and by associativity in $G$,  \[
ab = g^kn_1g^hn_2=g^kg^hn_1n_2= g^{k+h}n_1n_2= g^hg^kn_2n_1=g^hn_2g^kn_1=ba.
\]   
Thus, $G$ is abelian.   
\end{proof}
\end{homeworkProblem}

\begin{homeworkProblem}[Exercise 3.8.17: Compute the factor group $(\Z_6 \times \Z_4)\slash \left<(2,2)\right>$.]
\problemAnswer{ $\text{Note } N=\left< (2, 2)\right>= \{ (2,2), (4,0), (0, 2), (2, 0), (4,2), (0,0) \}. \text{ So } |N|=6 \text{ which implies } $\\
$[\Z_6 \times \Z_4 : N]= \frac{24}{6}=4. \text{ Thus, } \Z_6 \times \Z_4 \slash N \text{ will contain 4 elements. These are listed below.}$
\begin{eqnarray*}
	N=\{ (2,2), (4,0), (0, 2), (2, 0), (4,2), (0,0) \}\\
	(1,1)+N = \{(3,3), (5,1), (1,3), (3,1), (5,3), (1,1))\}\\
	(1,0)+N=\{ (3,2), (5,0), (1, 2), (3, 0), (5,2), (1,0)  \}\\
	(0,1)+N=\{ (2,3), (4,1), (0, 3), (2, 1), (4,3), (0,1)  \}
\end{eqnarray*}
Since $|\Z_6 \times \Z_4 \slash N|=4$  but does not contain an element of order $4$, $\Z_6 \times \Z_4 \slash N \cong \Z_2 \times \Z_2$. }
\end{homeworkProblem}

\begin{homeworkProblem}[Exercise 3.8.18: Compute the factor group $(\Z_6 \times \Z_4)\slash \left<(3,2)\right>$.]
\problemAnswer{
\[\text{Note } N=\left< (3, 2)\right>= \{ (3,2), (0,0) \}. \text{ So } |N|=2 \text{ which implies } 
\]
\[
[\Z_6 \times \Z_4 : N]= \frac{24}{2}=12. \text{ Thus, } \Z_6 \times \Z_4 \slash N \text{ will contain 12 elements. These are listed below.}
\]
\begin{eqnarray*}
	N=(3,2)+N=\{  (3,2), (0,0) \} \qquad & \qquad	(1,1)+N = \{(4,3), (1,1)\}\\
	(1,0)+N=\{ (4,2), (1,0) \} \qquad & \qquad	(0,1)+N=\{ (3,3), (0,1) \}\\
	(2,0)+N = \{(5,2), (2,0) \} \qquad & \qquad (0,2)+N = \{ (3,0), (0,2) \}\\
	(2,2)+N = \{ (5,0), (2,2) \} \qquad & \qquad	(3,1)+N = \{(0,3), (3,1) \}\\
	(2,3)+N  = \{ (5,1),(2,3) \} \qquad & \qquad (1,2)+N = \{ (4,0), (1,2)\}\\
	(2,1)+N  = \{ (5,3),(2,1) \} \qquad & \qquad (1,3)+N = \{ (4,1), (1,3)\}
		\end{eqnarray*}
Notice $(2,3)+N$ has order $12$ which implies $(\Z_6 \times \Z_4)\slash \left<(3,2)\right>$ is cyclic. \\ Then, $(\Z_6 \times \Z_4)\slash \left<(3,2)\right> \cong \Z_{12}$.
}
\end{homeworkProblem}
\newpage
\begin{homeworkProblem}[Exercise 3.8.20: Show that $(\Z \times \Z) \slash \left< (1,1)\right>$ is an infinite cyclic group. ]

\begin{proof}
Define $\phi: \Z \times \Z \rightarrow \Z$ by $\phi((a,b))=a-b$. We will use $\phi$ and the fundamental homomorphism theorem to show $(\Z \times \Z) \slash \left< (1,1)\right> \cong \Z$. First, we will show $\phi$ is a group homomorphism. Addition in $\Z$ is closed, so $\phi$ satisfies WD1. $\phi$ also satisfies WD2: consider $(a_1, b_1), (a_2, b_2) \in \Z \times \Z$ such that $(a_1,b_1)=(a_2, b_2)$. Then, $a_1 = a_2$ and $b_1=b_2$. Subtracting the second equality from the first we obtain, $a_1 - b_1 = a_2 - b_2$ which implies $\phi((a_1, b_1))= \phi((a_2, b_2))$. Thus, $\phi$ is a function. Now consider any $(a_1, b_1), (a_2, b_2) \in \Z \times \Z$, notice $\phi$ is a group homomorphism: 
\[
\phi((a_1, b_1)+(a_2, b_2))=\phi((a_1+a_2, b_1+b_2))= a_1+a_2-(b_1 + b_2)=a_1 - b_1 + a_2 - b_2 = \phi((a_1, b_1)) + \phi((a_2, b_2)).
\]
Finally, we will show ker$\phi=\left< (1,1) \right>$. Suppose $x \in $ ker$\phi$. Then, $x \in \Z \times \Z$, so $x = (a, b)$ for some $a, b \in \Z$. If $x \in $ ker$\phi$, then $\phi(x)=e_{\Z}=0$. Thus, $\phi((a,b))=0$ so $a-b=0$. Thus, if $(a, b) \in $ ker$\phi$, $a=b$. Note that $\left< (1, 1 )\right> = \{(a,b) \in \Z \times \Z \ | \ a=b\}$. Thus, $x=(a,b) \in \left< (1,1) \right>$. Suppose $x \in \left< (1,1)\right>$. Then $x=(a,b) \in \Z \times \Z$ with $a = b$ so $a-b=0$. Therefore $\phi((a,b))=0$ which implies $x \in $ ker$\phi$. Hence, ker $\phi=\left< (1,1)\right>$. \\
Next, we will show $\phi$ is onto. Consider any $z \in \Z$. Then, $(z + 1, 1) \in \Z \times \Z$ and $\phi(z+1, 1)=z+1 - 1 = z$. Thus, $\phi$ is onto and so $\phi(\Z \times \Z)=\Z$.  \\
By, the fundamental homomorphism theorem, $(\Z \times \Z) \slash \left< (1,1)\right> \cong \phi(\Z \times \Z)=\Z$. $\Z=\left<1\right>$ is an infinite cyclic group. By proposition 3.4.3, $(\Z \times \Z) \slash \left< (1,1)\right>$ must be an infinite cyclic group.
\end{proof}

\end{homeworkProblem}

\begin{homeworkProblem}[Exercise 3.8.21: Show that $(\Z \times \Z) \slash \left< (2,2)\right>$ is not a cyclic group. ]

\begin{proof}
Define $\phi: \Z \times \Z \rightarrow \Z\times \Z_2$ by $\phi((a,b))=(a-b, b \MOD 2))$. We will use $\phi$ and the fundamental homomorphism theorem to show $(\Z \times \Z) \slash \left< (2,2)\right> \cong \Z\times \Z_2$. First, we will show $\phi$ is a group homomorphism. Addition in $\Z$ is closed and $b \MOD 2) \in \Z_2$, so $\phi$ satisfies WD1. $\phi$ also satisfies WD2: consider $(a_1, b_1), (a_2, b_2) \in \Z \times \Z_2$ such that $(a_1,b_1)=(a_2, b_2)$. Then, $a_1 = a_2$ and $b_1=b_2$. Subtracting the second equality from the first we obtain, $a_1 - b_1 = a_2 - b_2$.  If $b_1 = b_2$, $b_1 \MOD 2) = b_2 \MOD 2)$.  which implies $\phi((a_1, b_1))=(a_1 - b_1, b_1 \MOD 2))=(a_2-b_2, b_2 \MOD 2))=\phi((a_2, b_2))$. Thus, $\phi$ is a function. Now consider any $(a_1, b_1), (a_2, b_2) \in \Z \times \Z$, notice $\phi$ is a group homomorphism: $
\phi((a_1, b_1)+(a_2, b_2))=\phi((a_1+a_2, b_1+b_2))= (a_1+a_2-(b_1 + b_2), (b_1 + b_2)\MOD 2))=
$\\
$
(a_1 - b_1 + a_2 - b_2, b_1 \MOD 2) + b_2 \MOD 2 ))= (a_1- b_1, b_1 \MOD 2))+(a_2-b_2, b_2 \MOD 2))
$\\
$=\phi((a_1, b_1)) + \phi((a_2, b_2)).$ \\ Finally, we will show ker$\phi=\left< (2,2) \right>$. Suppose $(a,b) \in $ ker$\phi$. Then, $(a, b) \in \Z \times \Z$ such that $\phi((a,b))=(0, 0)$. Thus, $a-b=0$ and $b \MOD 2) = 0$. Thus, if $(a, b) \in $ ker$\phi$, $a=b$. And because $b \MOD 2)=0$, both $a,b$ must be multiples of $2$. Note that $\left< (2, 2)\right> = \{2(a,b) \in \Z \times \Z \ | \ a=b\}$. Thus, $(a,b) \in \left< (2,2) \right>$. Suppose $(a,b) \in \left< (2,2)\right>$. Then $(a,b) \in \Z \times \Z$ with $a = b$ so $a-b=0$. Also, $(a, b) \in \left< (2,2)\right>$ implies $b \MOD 2) = 0$. Therefore $\phi((a,b))=(0,0)$ which implies $(a,b) \in $ ker$\phi$. Hence, ker $\phi=\left< (2,2)\right>$. \\
Next, we will show $\phi$ is onto. Consider any $(a,b) \in \Z \times \Z_2$. Note $b=0,1$. First consider $(a,0)$. Notice $(a,0) \in \Z \times \Z$ and $\phi(a,0)=(a-0, 0 \MOD 2))=(a,0)$. Next, consider $(a,1)$. Then, $(a+1,1) \in \Z \times \Z$ and $\phi(a+1,1)=(a+1-1,1 \MOD 2))=(a,1)$. Thus, $\phi$ is onto and so $\phi(\Z \times \Z)= \Z \times \Z_2$. 
\\
By the fundamental homomorphism theorem, $(\Z \times \Z) \slash \left< (2,2)\right> \cong \phi(\Z \times \Z)= \Z \times \Z_2$. We will show $\Z \times \Z_2 $ is not a cyclic group. Suppose $\Z \times \Z_2$ is cyclic. Then there must be $(a,b) \in \Z \times \Z_2$ such that $\Z \times \Z_2 = \left< (a,b) \right>$. Since $(1,0) \in \Z \times \Z_2$, $(1,0)=k(a,b)$ so $ka=1$ and $kb=0$ implies $b=0$. Thus, the generator must have the form $(a,0)$. But, $(1,1)$ is also in $\Z \times \Z_2$. So, there must be some $k$ such that $k(a,0)=(1,1)$. No such $k$ exists. Thus, $\Z \times \Z_2$ is not cyclic. By proposition 3.4.3, $(\Z \times \Z) \slash \left< (2,2)\right>$ is not a cyclic group.
	
\end{proof}

\end{homeworkProblem}

\begin{homeworkProblem}[Exercise 3.8.24: Let $S$ be an infinite set. Let $H$ be the set of all elements $\sigma \in $ Sym$(S)$ such that $\sigma(x)=x$ for all but finitely many $x \in S$. Prove that the subgroup $H$ is normal in Sym$(S)$. ]
\begin{proof}
	Let $S$ be an infinite set. Let $H$ be the set of all elements $\sigma \in $ Sym$(S)$ such that $\sigma(x)=x$ for all but finitely many $x \in S$. Consider any $\sigma \in H$ and let $A= \{ a_1, a_2, \dots, a_n \}$ be the set of $ x\in S$ such that $\sigma(x) \neq x$. Note $A$ is finite since $\sigma \in H$. Next, consider any $\tau \in $ Sym$(S)$ and let $\tau(A)=\{\tau(a_1), \tau(a_2), \dots, \tau(a_n)\}$. \\
	\textbf{Claim: } $\tau \sigma \tau^{-1}(x)=x$ if and only if $x \not \in \{\tau(a_1), \tau(a_2), \dots, \tau(a_n)\}$.\\
	Suppose $x \in \{\tau(a_1), \tau(a_2), \dots, \tau(a_n)\}$. Then, $x= \tau(a_j)$ and $\tau \sigma \tau^{-1}(\tau(a_j))=\tau \sigma(a_j)$. Note $\sigma(a_j) = a_k$ and $k \neq j$ and so $\tau(a_k) \in \tau(A)$. Thus, $\tau \sigma \tau^{-1}(\tau(a_j))=\tau (a_k)$.   Since $\tau$ is one to one, $\tau(a_j)\neq \tau(a_k)$. Thus, if $x \in \{\tau(a_1), \tau(a_2), \dots, \tau(a_n)\}$, $\tau \sigma \tau^{-1}$ moves $x$ on the finite set $\tau(A)$.\\
	Suppose $x \not\in \{\tau(a_1), \tau(a_2), \dots, \tau(a_n)\}$. Suppose $\tau \sigma \tau^{-1}(x)= y$ with $x \neq y$. Then, $\sigma \tau^{-1}(x)= \tau^{-1}(y)$. Since $\tau $ is one to one $\tau^{-1}(x)\neq \tau^{-1}(y)$ which implies $\sigma$ moves $\tau^{-1}(x)$. Thus, $\tau^{-1}(x)\in A$ and so $\tau^{-1}(x)=a_j$. Thus, $x = \tau(a_j)$ implies $x \in \tau(A)$ which contradicts our assumption that $x \not\in \{\tau(a_1), \tau(a_2), \dots, \tau(a_n)\}$. Hence, $\tau \sigma \tau^{-1}$ must fix $x$ when $x \not\in \{\tau(a_1), \tau(a_2), \dots, \tau(a_n)\}$.\\
	Thus, $\tau \sigma \tau^{-1}$ fixes all $x$ except when $x \in \tau(A)$ which implies $\tau \sigma \tau^{-1} \in H$. Hence $H$ is normal in Sym$(S)$.
	
\end{proof}


\end{homeworkProblem}

\end{flushleft}
\end{spacing}
\end{document}

%%%%%%%%%%%%%%%%%%%%%%%%%%%%%%%%%%%%%%%%%%%%%%%%%%%%%%%%%%%%%

%----------------------------------------------------------------------%
% The following is copyright and licensing information for
% redistribution of this LaTeX source code; it also includes a liability
% statement. If this source code is not being redistributed to others,
% it may be omitted. It has no effect on the function of the above code.
%----------------------------------------------------------------------%
% Copyright (c) 2007, 2008, 2009, 2010, 2011 by Theodore P. Pavlic
%
% Unless otherwise expressly stated, this work is licensed under the
% Creative Commons Attribution-Noncommercial 3.0 United States License. To
% view a copy of this license, visit
% http://creativecommons.org/licenses/by-nc/3.0/us/ or send a letter to
% Creative Commons, 171 Second Street, Suite 300, San Francisco,
% California, 94105, USA.
%
% THE SOFTWARE IS PROVIDED "AS IS", WITHOUT WARRANTY OF ANY KIND, EXPRESS
% OR IMPLIED, INCLUDING BUT NOT LIMITED TO THE WARRANTIES OF
% MERCHANTABILITY, FITNESS FOR A PARTICULAR PURPOSE AND NONINFRINGEMENT.
% IN NO EVENT SHALL THE AUTHORS OR COPYRIGHT HOLDERS BE LIABLE FOR ANY
% CLAIM, DAMAGES OR OTHER LIABILITY, WHETHER IN AN ACTION OF CONTRACT,
% TORT OR OTHERWISE, ARISING FROM, OUT OF OR IN CONNECTION WITH THE
% SOFTWARE OR THE USE OR OTHER DEALINGS IN THE SOFTWARE.
%----------------------------------------------------------------------%
