\documentclass[12 pt]{article}
% Change "article" to "report" to get rid of page number on title page
\usepackage{amsmath,amsfonts,amsthm,amssymb}
\usepackage{setspace}
\usepackage{Tabbing}
\usepackage{fancyhdr}
\usepackage{lastpage}
\usepackage{extramarks}
\usepackage{chngpage}
\usepackage{indentfirst}
\usepackage{soul,color}
\usepackage{graphicx,float,wrapfig}
\usepackage{gauss}
\usepackage{dcolumn}
\newcolumntype{2}{D{.}{}{2.0}}
\usepackage{multicol}
\usepackage{Tabbing}
\usepackage{fancyhdr}
\usepackage{lastpage}
\usepackage{extramarks}
\usepackage{enumerate}
\usepackage{mathtools}

\makeatletter
\renewcommand\section{\@startsection{section}{1}{\z@}%
                                  {-3.5ex \@plus -1ex \@minus -.2ex}%
                                  {2.3ex \@plus.2ex}%
                                  {\normalfont\bfseries}
                                }
\makeatother

\usepackage{etoolbox}
\makeatletter
\patchcmd\g@matrix
 {\vbox\bgroup}
 {\vbox\bgroup\normalbaselines}% restore the standard baselineskip
 {}{}
\makeatother

% In case you need to adjust margins:
\topmargin=-0.45in      %
\evensidemargin=0in     %
\oddsidemargin=0in      %
\textwidth=6.5in        %
\textheight=9.0in       %
\headsep=0.25in         %

% Homework Specific Information
\newcommand{\hmwkTitle}{Homework 3,\ Functions}
\newcommand{\hmwkDueDate}{Wednesday,\ September\ 16,\ 2015}
\newcommand{\hmwkClass}{Math\ 620}
\newcommand{\hmwkClassTime}{10:00}
\newcommand{\hmwkClassInstructor}{Boynton}
\newcommand{\hmwkAuthorName}{Kailee\ Gray}

\newtheorem{theorem}{Theorem}[section]
\newtheorem{lemma}[theorem]{Lemma}
\newtheorem{proposition}[theorem]{Proposition}
\newtheorem{corollary}[theorem]{Corollary}


\newenvironment{definition}[1][Definition]{\begin{trivlist}
\item[\hskip \labelsep {\bfseries #1}]}{\end{trivlist}}
\newenvironment{example}[1][Example]{\begin{trivlist}
\item[\hskip \labelsep {\bfseries #1}]}{\end{trivlist}}
\newenvironment{remark}[1][Remark]{\begin{trivlist}
\item[\hskip \labelsep {\bfseries #1}]}{\end{trivlist}}




% Setup the header and footer
\pagestyle{plain}                                                       %
\lhead{\hmwkAuthorName}                                                 %
\chead{\hmwkClass\ (\hmwkClassInstructor\ \hmwkClassTime): \hmwkTitle}  %
\rhead{\firstxmark}                                                     %
\lfoot{\lastxmark}                                                      %
\cfoot{}                                                                %
\rfoot{Page\ \thepage\ of\ \pageref{LastPage}}                          %
\renewcommand\headrulewidth{0.4pt}                                      %
\renewcommand\footrulewidth{0.4pt}                                      %

% This is used to trace down (pin point) problems
% in latexing a document:
%\tracingall

%%%%%%%%%%%%%%%%%%%%%%%%%%%%%%%%%%%%%%%%%%%%%%%%%%%%%%%%%%%%%
% Some tools
\newcommand{\enterProblemHeader}[1]{\nobreak\extramarks{#1}{#1 continued on next page\ldots}\nobreak%
                                    \nobreak\extramarks{#1 (continued)}{#1 continued on next page\ldots}\nobreak}%
\newcommand{\exitProblemHeader}[1]{\nobreak\extramarks{#1 (continued)}{#1 continued on next page\ldots}\nobreak%
                                   \nobreak\extramarks{#1}{}\nobreak}%

\newlength{\labelLength}
\newcommand{\labelAnswer}[2]
  {\settowidth{\labelLength}{#1}%
   \addtolength{\labelLength}{0in}%
   \changetext{}{-\labelLength}{}{}{}%
   \noindent\fbox{\begin{minipage}[c]{\columnwidth}#2\end{minipage}}%
   \marginpar{\fbox{#1}}%

   % We put the blank space above in order to make sure this
   % \marginpar gets correctly placed.
   \changetext{}{+\labelLength}{}{}{}}%

\setcounter{secnumdepth}{0}
\newcommand{\homeworkProblemName}{}%
\newcounter{homeworkProblemCounter}%
\newenvironment{homeworkProblem}[1][\arabic{homeworkProblemCounter}]%
  {\stepcounter{homeworkProblemCounter}%
   \renewcommand{\homeworkProblemName}{#1}%
   \section{\homeworkProblemName}%
   \noindent 
   
   \enterProblemHeader{\homeworkProblemName}}%
  {\exitProblemHeader{\homeworkProblemName}}%

\newcommand{\problemAnswer}[1]
  {\noindent\begin{minipage}[c]{\columnwidth}#1\end{minipage}}%

\newcommand{\problemLAnswer}[1]
    {\noindent\begin{minipage}[c]{\columnwidth}#1\end{minipage}}%

\newcommand{\homeworkSectionName}{}%
\newlength{\homeworkSectionLabelLength}{}%
\newenvironment{homeworkSection}[1]%
  {% We put this space here to make sure we're not connected to the above.
   % Otherwise the changetext can do funny things to the other margin

   \renewcommand{\homeworkSectionName}{#1}%
   \settowidth{\homeworkSectionLabelLength}{\homeworkSectionName}%
   \addtolength{\homeworkSectionLabelLength}{0 in}%
   \changetext{}{-\homeworkSectionLabelLength}{}{}{}%
   \subsection{\homeworkSectionName}%
   \enterProblemHeader{\homeworkProblemName\ [\homeworkSectionName]}}%
  {\enterProblemHeader{\homeworkProblemName}%

   % We put the blank space above in order to make sure this margin
   % change doesn't happen too soon (otherwise \sectionAnswer's can
   % get ugly about their \marginpar placement.
   \changetext{}{+\homeworkSectionLabelLength}{}{}{}}%

\newcommand{\sectionAnswer}[1]
  {\noindent\begin{minipage}[c]{\columnwidth}#1\end{minipage}}%
   \enterProblemHeader{\homeworkProblemName}\exitProblemHeader{\homeworkProblemName}%
  
 \newcommand{\R}{{\mathbb R}}
          \newcommand{\nil}{\varnothing}
          \newcommand{\N}{{\mathbb N}}
          \newcommand{\Z}{{\mathbb Z}}
        \newcommand{\MOD}{{ \ (\text{mod} \ }}

 \newcommand{\C}{{\mathbb C}}
 
%%%%%%%%%%%%%%%%%%%%%%%%%%%%%%%%%%%%%%%%%%%%%%%%%%%%%%%%%%%%%
% Make title
\title{\vspace{2in}\textmd{\textbf{\hmwkClass:\ \hmwkTitle}}\\\normalsize\vspace{0.1in}\small{Due\ on\ \hmwkDueDate}\\\vspace{0.1in}\large{\textit{\hmwkClassInstructor\ \hmwkClassTime}}\vspace{3in}}
\date{}
\author{\textbf{\hmwkAuthorName}}
%%%%%%%%%%%%%%%%%%%%%%%%%%%%%%%%%%%%%%%%%%%%%%%%%%%%%%%%%%%%%

\begin{document}
\begin{spacing}{1.5}
\maketitle
\newpage
% Uncomment the \tableofcontents and \newpage lines to get a Contents page
% Uncomment the \setcounter line as well if you do NOT want subsections
%       listed in Contents
%\setcounter{tocdepth}{1}
%\tableofcontents
%\newpage

% When problems are long, it may be desirable to put a \newpage or a
% \clearpage before each homeworkProblem environment

\clearpage

\begin{homeworkProblem}[Exercise 2.1.1: Determine whether the given function is one-to-one and whether it is onto.]
\noindent \textbf{(a) $f: \R \rightarrow \R; f(x)=x+3$\\} 
\\
\problemAnswer{
\textbf{(1-1)} This function is one-to-one: If $f(x_1)=f(x_2)$, then $x_1+3=x_2+3$ implies $x_1=x_2$.
\\
\textbf{(onto)} This function is onto: for any $y \in \R$, $y-3 \in \R$ will map to $y$; $f(y-3)=y-3+3=y$.\\
}
\\
\\
\noindent \textbf{(b) $f: \C \rightarrow \C; f(x)=x^2+2x+1$\\} \\
\problemAnswer{
\textbf{(1-1)} This function is not one-to-one: notice $f(0)=1=f(-2)$, but $0 \neq -2$.
\\
\textbf{(onto)} This function is onto: for any $y \in \C$, $ \sqrt{y} \in \C$ so $-1 \pm \sqrt{y}  \in \C$ will map to $y$; $f\left( -1 \pm \sqrt{y}\right)=\left( -1 \pm \sqrt{y}\right)^2 + 2\left( -1 \pm \sqrt{y}\right)+1=1 \pm 2\sqrt{y}+y-2 \pm 2 \sqrt{y}+1=y$.\\
}
\\
\\
\noindent \textbf{(c) $f: \Z_n \rightarrow \Z_n; f([x]_n)=[mx+b]_n$ where $m, b \in \Z$\\} \\
\problemAnswer{
\textbf{(1-1)} This function is not one-to-one: Let $n=8, m=4, b=1$. Then, $f([0]_8)=[4\cdot 0 + 1]=1=[4\cdot 2 + 1]=f([2]_8$ but $[0]_8 \neq [2]_8$.\\
However, note if $\gcd(m,n)=1$, then $f$ is 1-1. if $f([x_1]_n)=1=f([x_2]_n)$, then $[mx_1+b]_n=[mx_2+b]_n$ so $mx_1+b \equiv mx_2 + b \MOD n)$ which, since $\gcd(m,n)=1$, implies $x_1 \equiv x_2 \MOD n)$.
\\
\textbf{(onto)} This function is not onto: using $n=8, m=4, b=1$, there is no element in $\Z_8$ that maps to $2$. If there were, there would be a solution to $4x + 1 \equiv 2 \MOD 8)$ which would imply $4x \equiv 7 \MOD 8)$. If $x$ is even, $4x \equiv 0 \MOD 8)$ and if $x$ is odd, $4x \equiv 4 \MOD 8)$ so no such $x$ exists in $\Z_8$.

However, as above, if $\gcd(m,n)=1$, then $m$ has a unique multiplicative inverse, $m^{-1}$, mod $n$ and so $f$ is onto: for any $y\in \Z_8$, $m^{-1}(y-b)$ maps to $y$.\\
}
\\
\\
continued on page 3...

\newpage
\noindent \textbf{2.1.1}
\noindent \textbf{(d) $f: \R^+ \rightarrow \R; f(x)=$}ln\textbf{$(x)$}\\  \\
\problemAnswer{
\textbf{(1-1)} This function is one-to-one: If $f(x_1)=f(x_2)$, then $\ln x_1=\ln x_2+3$, so $e^{\ln x_1}=e^{\ln x_2}$, which implies $x_1=x_2$.
\\
\textbf{(onto)} This function is onto: for any $y \in \R$, $e^y \in \R^+$ and $f(e^y)=\ln (e^y) = y$.\\
}

\end{homeworkProblem}

\begin{homeworkProblem}[Exercise 2.1.3: For each one-to-one and onto function in exercise 2, find the inverse of the function.]
\noindent \textbf{(a) $f: \R \rightarrow \R; f(x)=x+3$} \qquad \qquad \qquad \qquad\problemAnswer{$f^{-1}(x)=x-3$}
\textbf{(b)} $f: \C \rightarrow \C; f(x)=x^2+2x+1$ \qquad \qquad \qquad \qquad \problemAnswer{$f^{-1}(x)=-1 \pm \sqrt{x}$}
\\
\noindent \textbf{(c) $f: \Z_n \rightarrow \Z_n; f([x]_n)=[mx+b]_n$ where} $m, b \in \Z$ \\
\problemAnswer{
\vspace{.3 cm}
If $\gcd(m,n)=1$, the multiplicative inverse of $m$ exists mod $n$. So let $m^{-1}$ be the multiplicative inverse of $m$. Then, $f^{-1}([x]_n)=[m^{-1}\cdot (x-b)]_n$.}\\
\\
\noindent \textbf{(d) $f: \R^+ \rightarrow \R; f(x)=$}ln\textbf{$(x)$} \qquad \qquad \qquad \qquad
\problemAnswer{$f^{-1}(x)=e^x$}

\end{homeworkProblem}

\begin{homeworkProblem}[Exercise 2.1.9: Show that the following formula yields a well-defined function.]
\noindent \textbf{(a) $f: \Z_8 \rightarrow \Z_8; f([x]_8)=[mx]_8$, for any $m \in \Z$} 
\\
\problemAnswer{
\vspace{.3 cm}
\textbf{(WD1)} For every $\lbrack x \rbrack_{8} \in \Z_{8}$ we have $ \lbrack x\rbrack_8 \in  \Z_8$ so this condition is satisfied.
\\
\textbf{(WD2)} Assume $\lbrack x_1 \rbrack_{8}=\lbrack x_2 \rbrack_{8}$. Then, $x_1 \equiv x_2 \MOD 8)$. Multiply both sides of this congruence by $m$ to obtain $mx_1 \equiv mx_2 \MOD 8)$. This implies $f(\lbrack x_1 \rbrack_{8})=f(\lbrack x_2 \rbrack_{8})$.
\\
}

\end{homeworkProblem}

\begin{homeworkProblem}[Exercise 2.1.10: Give an example to show that the formula does not define a function.]
\noindent \textbf{(d) $p: \Z_{12} \rightarrow \Z_5;\  p([x]_{12})=[2x]_5$} 
\\
\problemAnswer{
\vspace{.3 cm}
Notice $[0]_{12}=[12]_{12}$ but $f([0]_{12})=[0]_5$ and $f([12]_{12})=[4]_5$. Since $f([0]_{12})\neq f([12]_{12})$, we have the same input mapping to two different outputs, so $f$ is not a function. 
}

\end{homeworkProblem}

\begin{homeworkProblem}[Exercise 2.1.11: Let $k$ and $n$ be positive integers. For a fixed $m \in \Z$ define $f: \Z_n \rightarrow \Z_k$ by $f(\lbrack x \rbrack_n)=(\lbrack mx\rbrack_k)$ for $x  \in \Z$. Show that $f$ defines a function if and only if $k \ | \ mn$.]
 \problemLAnswer{
\begin{proof}
	$(\Rightarrow)$ \hskip0.5 cm  Assume $f$ is a function. Then, $[0]_n$, which is equivalent to $[n]_n$, must map to exactly one element in $\Z_k$. Thus $f([0]_n)=f([n]_n)$ and so $[m\cdot 0]_k=[m\cdot n]_k$. Thus $0 \equiv mn \MOD k)$ which implies $k \ | \ mn$.
\\
$(\Leftarrow)$ \hskip0.25 cm Next, suppose $k \ | \ mn$. \\
\textbf{(WD1)} For every $\lbrack x \rbrack_{n} \in \Z_{n}$ we have $\lbrack mx\rbrack_k \in  \Z_k$ so this condition is satisfied.
\\
\textbf{(WD2)} 
We must show that when $[x_1]_n=[x_2]_n$, $f([x_1]_n)=f([x_2]_n)$. Let $x_1, x_2 \in \Z_n$ such that $[x_1]_n=[x_2]_n$. Then $x_1 \equiv x_2 \MOD n)$ and so $x_1=x_2+bn$ for some $b \in \Z$. Multiply both sides of $x_1=x_2+bn$ by $m$ to obtain $mx_1=mx_2+mbn$. Equivalently, $mx_1=mx_2+b(mn)$. Notice if $k \ | \ mn$, $mn = ka$ for some $a \in \Z$. So we can substitute $mn = ka$ into $mx_1=mx_2+b(mn)$ to obtain $mx_1=mx_2+b(ka)$. Thus, $mx_1 \equiv mx_2 \MOD k)$ and $f([x_1]_n)=f([x_2]_n)$.
\end{proof}
}
\end{homeworkProblem}

\begin{homeworkProblem}[Exercise 2.1.12: Let $k$ and $n$ be positive integers such that $k \ | \ mn$. Show that $f$ defined by: $m \in \Z$, $f: \Z_n \rightarrow \Z_k$, and $f(\lbrack x \rbrack_n)=(\lbrack mx\rbrack_k)$ is a one-to-one correspondence if and only if $k=n$ and $\gcd(m,n)=1$.]
 \problemLAnswer{
\begin{proof}
	$(\Rightarrow)$ \hskip0.5 cm  Assume $f$ defined by $m \in \Z$, $f: \Z_n \rightarrow \Z_k$, and $f(\lbrack x \rbrack_n)=(\lbrack mx\rbrack_k)$ is a one-to-one correspondence. Then, $f$ is one-to-one and onto. \\
	\\
	If $f$ is one-to-one, the Pigeon-hole Principle as presented in Boyton implies $n<k$ or $n=k$. If $n<k$, $|\Z_n|<|\Z_k|$. Since $f$ is one-to-one, there exist a maximum of $n$ elements mapped to in $|\Z_k|$ by $f$. So there would exist some $b \in \Z_k$ such that there was no $x \in \Z_n$ with $f([x]_n)=[b]_k$. However, $f$ is onto so such a $b \in \Z_n$ is a contradiction. Thus, $k=n$. 
	\\
	\\
	Because $f$ is onto, there exists $x \in \Z_n$ such that $f([x]_n)=[1]_k$; $f([x]_n)=[mx]_k$, so there must exist $x \in \Z_n$ such that $mx \equiv 1 \MOD k)$. Then, $mx = 1 + bk$ for some $b \in \Z$. Equivalently, $1 = mx + (-b)k$. We showed above that $k=n$, so we can substitute to obtain $1 = mx + (-b)n$. Since we can write $1$ as a linear combination of $m, n$, $\gcd (m,n)=1$. 
\\
\\
\\
$(\Leftarrow)$ \hskip0.25 cm Assume $k=n$ and $\gcd(m,n)=1$. We will show $f$ is one-to-one and then use theorem 14 in Boynton to show $f$ is a one-to-one correspondence. Suppose $x_1, x_2 \in \Z_n$ such that $f([x_1]_n)=f([x_2]_n)$. Since $f([x_1]_n)=[mx_1]_k$ and $f([x_2]_n)=[mx_2]_k$, we have $mx_1 \equiv mx_2 \MOD k)$. Since $k=n$, the previous equivalence can be written $mx_1 \equiv mx_2 \MOD n)$. \\
\\
If $\gcd(m,n)=1$, $m$ has a multiplicative inverse mod $n$, so multiplying both sides of $mx_1 \equiv mx_2 \MOD n)$ by this multiplicative inverse gives us $x_1 \equiv x_2 \MOD n)$. Thus $f([x_1]_n)=f([x_2]_n)$ implies $[x_1]_n=[x_2]_n$; so $f$ is one-to-one.If $k=n$, $\Z_k=\Z_n$ and $|\Z_k|=|\Z_n|$. Thus, because $f$ has domain and co-domain with the same cardinality and because $f$ is one-to-one, theorem 14 in Boynton implies $f$ is a one-to-one correspondence.

\end{proof}
}
\end{homeworkProblem}

\begin{homeworkProblem}[Exercise 2.1.14: Let $f: A \rightarrow B$ and $g: B \rightarrow C$ be one-to-one and onto. Show that $(g \circ f)^{-1}$ exists and $(g \circ f)^{-1}=f^{-1} \circ g^{-1}$. ]
 \problemLAnswer{
\begin{proof}
Let $f: A \rightarrow B$ and $g: B \rightarrow C$ be one-to-one and onto.\\
	First we will show that $(g \circ f)^{-1}$ exists. By proposition 2.1.5 (Beachy), if $g, f$ are one-to-one and onto, $g \circ f$ is one-to-one and onto. So, by proposition 2.1.7 (Beachy), $(g \circ f)^{-1}$ exists and is unique. Next, we will prove $(g \circ f)^{-1}=f^{-1} \circ g^{-1}$. By definition 2.1.6 (Beachy), if $(g \circ f)^{-1}$ is the inverse of $g \circ f$, $(g \circ f)\circ (g \circ f)^{-1} =1_C$ and $(g \circ f)^{-1} \circ (g \circ f)=1_A$. So, we will evaluate $(g \circ f)\circ (f^{-1} \circ g^{-1})$ and $(f^{-1} \circ g^{-1}) \circ (g \circ f)$: 
\begin{equation*}
 \begin{multlined}
(g \circ f)\circ (f^{-1} \circ g^{-1}) = g \circ (f\circ f^{-1}) \circ g^{-1} \qquad \text{by associative property of } \circ\\
= g \circ (1_B) \circ g^{-1} \qquad \text{by definition 2.1.6 (Beachy) } \\
= g \circ (1_B \circ g^{-1}) \qquad \text{by associative property of } \circ\\
= g \circ g^{-1} \qquad \text{by definition 2.1.6 (Beachy) } \\
= 1_C \qquad \text{by definition 2.1.6 (Beachy.)}\\
\end{multlined}
\end{equation*}
\begin{equation*}
 \begin{multlined}
(f^{-1} \circ g^{-1})\circ (g \circ f) = f^{-1} \circ (g^{-1}\circ g) \circ f \qquad \text{by associative property of } \circ\\
= f^{-1} \circ (1_B) \circ f \qquad \text{by definition 2.1.6 (Beachy) } \\
= f^{-1} \circ (1_B \circ f)  \qquad \text{by associative property of } \circ\\
= f^{-1} \circ f \qquad \text{by definition 2.1.6 (Beachy) } \\
= 1_A \qquad \text{by definition 2.1.6 (Beachy.)} \qquad \qquad \qquad \qquad 
\end{multlined}
\end{equation*}
Thus $f^{-1} \circ g^{-1}$ is an inverse of $g \circ f$. Additionally, $(g \circ f)^{-1}$ is unique so $(g \circ f)^{-1}=f^{-1} \circ g^{-1}$.
\end{proof}
}
\end{homeworkProblem}


\begin{homeworkProblem}[Exercise 2.1.15, part 1: Let $f: A \rightarrow B$ and $g: B \rightarrow C$ be functions. Prove that if $g \circ f$ is one-to-one, then $f$ is one-to-one ]
 \problemLAnswer{
\begin{proof}
Assume $g \circ f$ is one-to-one. Let $x_1, x_2 \in A$ such that $f(x_1)=f(x_2)$. Then, since $g$ is a function, $g(f(x_1))=g(f(x_2))$ and so $(g\circ f) (x_1)=(g\circ f)(x_2)$. Since $g \circ f$ is one-to-one, $(g\circ f) (x_1)=(g\circ f)(x_2)$ implies $x_1=x_2$. Thus, $f$ is one-to-one.
\end{proof}
}
\end{homeworkProblem}

\begin{homeworkProblem}[Exercise 2.1.15, part 2: Let $f: A \rightarrow B$ and $g: B \rightarrow C$ be functions. Prove that if  $g \circ f$ is onto, then $g$ is onto. ]
 \problemLAnswer{
\begin{proof}
If $g \circ f$ is onto for any $c \in C$, there exists $a \in A$ such that $(g \circ f)(a)=c$. This implies $g(f((a))=c$.  $f$ is a function so for any $a \in A$, there exists $b \in B$ with $f(a)=b$.  Thus, for any $c \in C$, there exists $b \in B$, $f(a)=b$ such that $g(b)=g(f((a))=c$. Thus $g$ is onto.  
\end{proof}
}
\end{homeworkProblem}


\begin{homeworkProblem}[Exercise 2.1.17: Let $f: A \rightarrow B$ be a function. Prove that $f$ is onto if and only if $h \circ f = k \circ f$ implies $h=k$, for every set $C$ and all choice of functions $h: B \rightarrow C$ and $k: B \rightarrow C$. ]
 \problemLAnswer{
\begin{proof}
	$(\Rightarrow)$ \hskip0.5 cm  Assume $f$ is onto. Then, for any $b \in B$, there exists $a \in A$ such that $f(a)=b$. Next, assume $h \circ f = k \circ f$. Then, by definition of function equality, for all $a \in A$, $(h \circ f)(a) = (k \circ f)(a)$. Equivalently, $h(f(a)) = k(f(a))$. Again, since $f$ is onto and because $h, k$ are functions, for any $b \in B$ there exists an $a \in A$ such that $h(b)=h(f(a))$ and $k(b)=k(f(a))$. Thus $h(f(a)) = k(f(a))$, $h(b)=h(f(a))$ and $k(b)=k(f(a))$ imply $h(b) = k(b)$ for any $b \in B$. Therefore, $h=k$.
\\
$(\Leftarrow)$ \hskip0.25 cm Assume $h \circ f = k \circ f$ implies $h=k$, for every set $C$ and all choice of functions $h: B \rightarrow C$ and $k: B \rightarrow C$.  Since $h \circ f = k \circ f$, $h, k$ agree on the image on $f$. Suppose $f$ is not onto. Then, there exists $b \in B$ such that there are no $a \in A$ with $f(a)=b$. We will define $h, k$ such that $h, k$ agree on the image of $f$ but not on all of $B$. Let $h(x)=1$ if $x = b$, $k(x)=2$ if $x=b$, and let $h(x)=0=k(x)$ when $x \neq b$. Then, $(h \circ f)(x) = (k \circ f)(x)$ for all $x$ but $h(x)\neq k(x)$ when $x=b$. This is a contradiction, so $f$ must be onto. 

\end{proof}
}
\end{homeworkProblem}

\begin{homeworkProblem}[Exercise 2.1.20: Define $f \colon \Z_{mn}\rightarrow \Z_m \times \Z_n$ by $f(\lbrack x \rbrack_{mn})=(\lbrack x\rbrack_m, \lbrack x \rbrack_n)$. Show that $f$ is a function and that $f$ is onto if and only if $\gcd(m,n)=1$.]
\problemLAnswer{
\begin{proof}
$(\Rightarrow)$ \hskip0.5 cm  Assume $f \colon \Z_{mn} \rightarrow \Z_m \times \Z_n$ defined by $f(\lbrack x \rbrack_{mn})=(\lbrack x\rbrack_m, \lbrack x \rbrack_n)$ is a function and is onto. Since $f$ is onto, there must exist some $[x]_{mn} \in \Z_{mn}$ such that $f([x]_{mn})=([0]_m, [1]_n)$. Then, $x \equiv 0 \MOD m) $ and $x \equiv 1 \MOD n)$ so $x=mk$ and $x = 1 + ns$ for some $k, s \in \Z$. Thus, $mk = 1 + ns$ and $1 = ns + (-k)m$. Since we can write $1$ as a linear combination of $m$ and $n$, $\gcd(m,n)=1$.
\\
\\
$(\Leftarrow)$ \hskip0.25 cm 
Assume $\gcd(m,n,)=1$. 
\\
First, we will show $f \colon \Z_{mn} \rightarrow \Z_m \times \Z_n$ defined by $f(\lbrack x \rbrack_{mn})=(\lbrack x\rbrack_m, \lbrack x \rbrack_n)$ is a function. 
\\
\textbf{(WD1)} For every $\lbrack x \rbrack_{mn} \in \Z_{mn}$ we have $\left( \lbrack x\rbrack_m, \lbrack x \rbrack_n \right) \in  \Z_m \times \Z_n$ so this condition is satisfied.
\\
\textbf{(WD2)} Assume $\lbrack x_1 \rbrack_{mn}=\lbrack x_2 \rbrack_{mn}$. Then, $x_1 \equiv x_2 \MOD mn)$ so $x_1 = x_2 + mn(k)$ for some $k \in \Z$. Thus, $x_1 \equiv x_2 \MOD m)$ and $x_1 \equiv x_2 \MOD n)$ and $(\lbrack x_1 \rbrack_m, \lbrack x_1 \rbrack_n)=(\lbrack x_2 \rbrack_m, \lbrack x_2 \rbrack_n)$. This implies $f(\lbrack x_1 \rbrack_{mn})=f(\lbrack x_2 \rbrack_{mn})$.
\\
Now we will show $f$ is onto. Consider any element in $([b]_m, [a]_n) \in \Z_m \times \Z_n$.   If $f([x]_{mn})=([b]_m, [a]_n)$, then $x \equiv b \MOD m) $ and $x \equiv a \MOD n)$. Because $\gcd(m,n)=1$, the Chinese Remainder Theorem implies there exists a unique solution, mod $mn$ to the previous system of congruences. Also, since $\gcd(m,n)=1$ there exists $r,s \in \Z$ such that $rm + sn = 1$. Following the construction in the proof of the Chinese Remainder Theorem, let $x = arm + bsn$, so that $f([arm + bsn]_{mn})=([arm + bsn]_m,[arm + bsn]_n)=([bsn]_m,[arm]_n)$. Since $sn \equiv 1 \MOD m)$ and $rm \equiv 1 \MOD n)$, we have $f([arm + bsn]_{mn})=([b]_m,[a]_n)$.  \end{proof}
}
\end{homeworkProblem}

\end{spacing}
\end{document}

%%%%%%%%%%%%%%%%%%%%%%%%%%%%%%%%%%%%%%%%%%%%%%%%%%%%%%%%%%%%%

%----------------------------------------------------------------------%
% The following is copyright and licensing information for
% redistribution of this LaTeX source code; it also includes a liability
% statement. If this source code is not being redistributed to others,
% it may be omitted. It has no effect on the function of the above code.
%----------------------------------------------------------------------%
% Copyright (c) 2007, 2008, 2009, 2010, 2011 by Theodore P. Pavlic
%
% Unless otherwise expressly stated, this work is licensed under the
% Creative Commons Attribution-Noncommercial 3.0 United States License. To
% view a copy of this license, visit
% http://creativecommons.org/licenses/by-nc/3.0/us/ or send a letter to
% Creative Commons, 171 Second Street, Suite 300, San Francisco,
% California, 94105, USA.
%
% THE SOFTWARE IS PROVIDED "AS IS", WITHOUT WARRANTY OF ANY KIND, EXPRESS
% OR IMPLIED, INCLUDING BUT NOT LIMITED TO THE WARRANTIES OF
% MERCHANTABILITY, FITNESS FOR A PARTICULAR PURPOSE AND NONINFRINGEMENT.
% IN NO EVENT SHALL THE AUTHORS OR COPYRIGHT HOLDERS BE LIABLE FOR ANY
% CLAIM, DAMAGES OR OTHER LIABILITY, WHETHER IN AN ACTION OF CONTRACT,
% TORT OR OTHERWISE, ARISING FROM, OUT OF OR IN CONNECTION WITH THE
% SOFTWARE OR THE USE OR OTHER DEALINGS IN THE SOFTWARE.
%----------------------------------------------------------------------%
