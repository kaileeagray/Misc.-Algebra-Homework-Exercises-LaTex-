\documentclass[12 pt]{article}
% Change "article" to "report" to get rid of page number on title page
\usepackage{amsmath,amsfonts,amsthm,amssymb}
\usepackage{setspace}
\usepackage{Tabbing}
\usepackage{fancyhdr}
\usepackage{lastpage}
\usepackage{extramarks}
\usepackage{chngpage}
\usepackage{indentfirst}
\usepackage{soul,color}
\usepackage{graphicx,float,wrapfig}
\usepackage{gauss}
\usepackage{dcolumn}
\newcolumntype{2}{D{.}{}{2.0}}
\usepackage{multicol}
\usepackage{Tabbing}
\usepackage{fancyhdr}
\usepackage{lastpage}
\usepackage{extramarks}
\usepackage{enumerate}
\usepackage{mathtools}
\usepackage{tikz}
\usetikzlibrary{positioning}
\usepackage{float}
\usepackage{wrapfig}



\graphicspath{ {/home/user/Documents/} }

\makeatletter
\renewcommand\section{\@startsection{section}{1}{\z@}%
                                  {-3.5ex \@plus -1ex \@minus -.2ex}%
                                  {2.3ex \@plus.2ex}%
                                  {\normalfont\bfseries}
                                }
\makeatother

\usepackage{etoolbox}
\makeatletter
\patchcmd\g@matrix
 {\vbox\bgroup}
 {\vbox\bgroup\normalbaselines}% restore the standard baselineskip
 {}{}
\makeatother

% In case you need to adjust margins:
\topmargin=-0.75in      %
\evensidemargin=0in     %
\oddsidemargin=0in      %
\textwidth=6.5in        %
\textheight=9.0in       %
\headsep=0.25in         %

% Homework Specific Information
\newcommand{\hmwkTitle}{$\S 4.1$ Fields; Roots of Polynomials}
\newcommand{\hmwkDueDate}{Friday,\ November\ 20,\ 2015}
\newcommand{\hmwkClass}{Math\ 620}
\newcommand{\hmwkClassTime}{10:00}
\newcommand{\hmwkClassInstructor}{Boynton}
\newcommand{\hmwkAuthorName}{Kailee\ Gray}

\newtheorem{theorem}{Theorem}[section]
\newtheorem{lemma}[theorem]{Lemma}
\newtheorem{proposition}[theorem]{Proposition}
\newtheorem{corollary}[theorem]{Corollary}


\newenvironment{definition}[1][Definition]{\begin{trivlist}
\item[\hskip \labelsep {\bfseries #1}]}{\end{trivlist}}
\newenvironment{example}[1][Example]{\begin{trivlist}
\item[\hskip \labelsep {\bfseries #1}]}{\end{trivlist}}
\newenvironment{remark}[1][Remark]{\begin{trivlist}
\item[\hskip \labelsep {\bfseries #1}]}{\end{trivlist}}




% Setup the header and footer
\pagestyle{plain}                                                       %
\lhead{\hmwkAuthorName}                                                 %
\chead{\hmwkClass\ (\hmwkClassInstructor\ \hmwkClassTime): \hmwkTitle}  %
\rhead{\firstxmark}                                                     %
\lfoot{\lastxmark}                                                      %
\cfoot{}                                                                %
\rfoot{Page\ \thepage\ of\ \pageref{LastPage}}                          %
\renewcommand\headrulewidth{0.4pt}                                      %
\renewcommand\footrulewidth{0.4pt}                                      %

% This is used to trace down (pin point) problems
% in latexing a document:
%\tracingall

%%%%%%%%%%%%%%%%%%%%%%%%%%%%%%%%%%%%%%%%%%%%%%%%%%%%%%%%%%%%%
% Some tools
\newcommand{\enterProblemHeader}[1]{\nobreak\extramarks{#1}{#1 continued on next page\ldots}\nobreak%
                                    \nobreak\extramarks{#1 (continued)}{#1 continued on next page\ldots}\nobreak}%
\newcommand{\exitProblemHeader}[1]{\nobreak\extramarks{#1 (continued)}{#1 continued on next page\ldots}\nobreak%
                                   \nobreak\extramarks{#1}{}\nobreak}%

\newlength{\labelLength}
\newcommand{\labelAnswer}[2]
  {\settowidth{\labelLength}{#1}%
   \addtolength{\labelLength}{0in}%
   \changetext{}{-\labelLength}{}{}{}%
   \noindent\fbox{\begin{minipage}[c]{\columnwidth}#2\end{minipage}}%
   \marginpar{\fbox{#1}}%

   % We put the blank space above in order to make sure this
   % \marginpar gets correctly placed.
   \changetext{}{+\labelLength}{}{}{}}%

\setcounter{secnumdepth}{0}
\newcommand{\homeworkProblemName}{}%
\newcounter{homeworkProblemCounter}%
\newenvironment{homeworkProblem}[1][\arabic{homeworkProblemCounter}]%
  {\stepcounter{homeworkProblemCounter}%
   \renewcommand{\homeworkProblemName}{#1}%
   \section{\homeworkProblemName}%
   \noindent 
   
   \enterProblemHeader{\homeworkProblemName}}%
  {\exitProblemHeader{\homeworkProblemName}}%

\newcommand{\problemAnswer}[1]
  {\noindent\begin{minipage}[c]{\columnwidth}#1\end{minipage}}%

\newcommand{\problemLAnswer}[1]
    {\noindent\begin{minipage}[c]{\columnwidth}#1\end{minipage}}%

\newcommand{\homeworkSectionName}{}%
\newlength{\homeworkSectionLabelLength}{}%
\newenvironment{homeworkSection}[1]%
  {% We put this space here to make sure we're not connected to the above.
   % Otherwise the changetext can do funny things to the other margin

   \renewcommand{\homeworkSectionName}{#1}%
   \settowidth{\homeworkSectionLabelLength}{\homeworkSectionName}%
   \addtolength{\homeworkSectionLabelLength}{0 in}%
   \changetext{}{-\homeworkSectionLabelLength}{}{}{}%
   \subsection{\homeworkSectionName}%
   \enterProblemHeader{\homeworkProblemName\ [\homeworkSectionName]}}%
  {\enterProblemHeader{\homeworkProblemName}%

   % We put the blank space above in order to make sure this margin
   % change doesn't happen too soon (otherwise \sectionAnswer's can
   % get ugly about their \marginpar placement.
   \changetext{}{+\homeworkSectionLabelLength}{}{}{}}%

\newcommand{\sectionAnswer}[1]
  {\noindent\begin{minipage}[c]{\columnwidth}#1\end{minipage}}%
   \enterProblemHeader{\homeworkProblemName}\exitProblemHeader{\homeworkProblemName}%
  
 \newcommand{\R}{{\mathbb R}}
          \newcommand{\nil}{\varnothing}
          \newcommand{\N}{{\mathbb N}}
          \newcommand{\Z}{{\mathbb Z}}
        \newcommand{\MOD}{{ \ (\text{mod} \ }}

 \newcommand{\C}{{\mathbb C}}
  \newcommand{\Q}{{\mathbb Q}}
  
%%%%%%%%%%%%%%%%%%%%%%%%%%%%%%%%%%%%%%%%%%%%%%%%%%%%%%%%%%%%%
% Make title
\title{\vspace{2in}\textmd{\textbf{\hmwkClass:\ \hmwkTitle}}\\\normalsize\vspace{0.1in}\small{Due\ on\ \hmwkDueDate}\\\vspace{0.1in}\large{\textit{\hmwkClassInstructor\ \hmwkClassTime}}\vspace{3in}}
\date{}
\author{\textbf{\hmwkAuthorName}}
%%%%%%%%%%%%%%%%%%%%%%%%%%%%%%%%%%%%%%%%%%%%%%%%%%%%%%%%%%%%%

\begin{document}
\begin{spacing}{1.75}
\maketitle

 
% Uncomment the \tableofcontents and   lines to get a Contents page
% Uncomment the \setcounter line as well if you do NOT want subsections
%       listed in Contents
%\setcounter{tocdepth}{1}
%\tableofcontents
% 

% When problems are long, it may be desirable to put a   or a
% \clearpage before each homeworkProblem environment
4.1: 2,6,9,11,13,17,18

\begin{flushleft}

\begin{homeworkProblem} [Exercise 4.1.2: Let $p$ be a prime number and let $n$ be a positive integer. How many polynomials are there of degree $n$ over $\Z_p$? ]
\problemAnswer{ 
Consider some polynomial over $\Z_p$ of degree $n$: $a_nx^n + a_{n-1}x^{n-1}+ \dots + a_1x + a_0$. Then, $a_n, a_{n-1}, \dots, a_1, a_0 \in \Z_p$ so there are $p$ choices for each of the $n+1$ $a_i$'s with the exception of $a_n$.  Since $a_nx^n + a_{n-1}x^{n-1}+ \dots + a_1x + a_0$ has degree $n$, the leading coefficient must be non-zero; thus there are $p-1$ choices for $a_n$. Hence, there are $p^n(p-1)$ possible polynomials of degree $n$ over $\Z_p$.  
}\end{homeworkProblem}
\begin{homeworkProblem}[Exercise 4.1.6: Let $p$ be a prime number. Find all roots of $x^{p-1}-1$ in $\Z_p$. ]
\problemAnswer{
By corollary 1.4.12 (Fermat), since $p$ is prime, $x^p \equiv x \MOD p)$ for any integer $x$. We are looking for roots in $\Z_p$ so consider $x \MOD p)$. Then, $x \in \Z_p$ and so $\gcd(x,p)=1$ which allows us, as long as $x \neq 0$, to divide both sides of $x^p \equiv x \MOD p)$ by $x$ to obtain $x^{p-1} \equiv 1 \MOD p)$. This is equivalent to $x^{p-1}-1 \equiv 0 \MOD p)$. Thus, any $x \in \Z_p$, except $x=0$, is a root of $x^{p-1}-1$ in $\Z_p$.
}
\end{homeworkProblem}

\begin{homeworkProblem}[Exercise 4.1.9: Let $a$ be a nonzero element of a field $F$. Show that $(a^{-1})^{-1}=a$ and $(-a)^{-1}=-a^{-1}$. ]
\problemAnswer{
\begin{proof}
 Let $a$ be a nonzero element of a field $F$. Since $a$ is nonzero, $a^{-1}$ and $(a^{-1})^{-1}$ exist;
 \begin{eqnarray*}
 	(a^{-1})^{-1} & = & (a^{-1})^{-1}  \\
 	(a^{-1})^{-1}a^{-1} & = & (a^{-1})^{-1}a^{-1} \\
 	1 & = & (a^{-1})^{-1}a^{-1} \\
	1a & = & (a^{-1})^{-1}a^{-1}a \\
	a & = & (a^{-1})^{-1}.
 \end{eqnarray*}
Since $a$ is nonzero, $-a$ is nonzero and $(-a)^{-1}$ exists such that $(-a)^{-1}(-a)=1$;
\begin{eqnarray*}
(-a)^{-1}(-a) & = & 1 \\ 	
-(-a)^{-1}a & = & 1 \\ 
-(-a)^{-1}a(-a^{-1}) & = & 1(-a^{-1}) \\ 
-(-(-a)^{-1}a(a^{-1})) & = & -a^{-1} \\ 
(-a)^{-1} & = & -a^{-1}
\end{eqnarray*}
 \end{proof} }
\end{homeworkProblem}
 
\begin{homeworkProblem}[Exercise 4.1.11: Show that the set $\Q(\sqrt{3})=\{ a+b\sqrt{3} \ | \ a, b \in \Q \}$ is closed under addition, subtraction, multiplication, and division.]
\problemAnswer{
\begin{proof}
Consider $a+b\sqrt{3}, c + d\sqrt{3} \in \Q(\sqrt{3})$. \\
\textbf{(addition)} Then, $a+b\sqrt{3}+ c + d\sqrt{3}=a+c + (b+d)\sqrt{3}$. Since addition in $\Q$ is closed, $a + c \in \Q$ and $b + d \in \Q$. Thus, $\Q(\sqrt{3})$ is closed under addition.\\
\textbf{(subtraction)} Then, $(a+b\sqrt{3}) - (c + d\sqrt{3})=a-c + (b-d)\sqrt{3}=a+(-c) + (b+(-d))\sqrt{3}$. $c, d \in \Q$ implies $-c, -d \in \Q$. Addition in $\Q$ is closed so $a + (-c) \in \Q$ and $b + (-d) \in \Q$. Thus, $\Q(\sqrt{3})$ is closed under subtraction.\\
\textbf{(multiplication)} Note $(a+b\sqrt{3})(c + d\sqrt{3})=ac + (bc)\sqrt{3}+(ad)\sqrt{3}+(bd)3=ac+3bd + (bc+ad)\sqrt{3}$. Multiplication and addition in $\Q$ is closed so $a,b,c, d \in \Q$ implies $ac+3bd, bc+ad \in \Q$. Therefore, $ac+3bd + (bc+ad)\sqrt{3} \in \Q(\sqrt{3})$. Thus, $\Q(\sqrt{3})$ is closed under multiplication.\\
\textbf{(division)} Note $c+ d \sqrt{3}\neq 0$ and 
\[
\frac{a+b\sqrt{3}}{c + d\sqrt{3}}=\frac{a+b\sqrt{3}}{c + d\sqrt{3}}\cdot \frac{c - d\sqrt{3}}{c - d\sqrt{3}}=\frac{ac-3bd +(bc-ad)\sqrt{3}}{c^2 - 3d^2}=\frac{ac-3bd}{c^2 - 3d^2} +\frac{bc-ad}{c^2 - 3d^2}\sqrt{3}\]
 Multiplication, addition, and subtraction is closed in $\Q$ and division is closed in $\Q - \{0\}$  so $a,b,c, d \in \Q$ implies $\frac{ac-3bd}{c^2 - 3d^2} \in \Q$ and $\frac{bc-ad}{c^2 - 3d^2} \in \Q$ as long as $c^2-3d^2\neq0$. Suppose $c^2-3d^2  =  0. \text{ Then, }
	c^2  =  3d^2$ and $c  = \pm \sqrt{3}d. $
Notice $\sqrt{3}$ is an irrational number so $\sqrt{3}d$ is irrational as long as $d\neq 0$. Since $c=\pm \sqrt{3}d$, if $d=0$, $c=0$, but $c+d\sqrt{3}\neq 0$. Thus, $c^2$ is irrational which implies $c$ is irrational. However, $c+d\sqrt{3} \in \Q(\sqrt{3})$ implies $c,d \in \Q$. Hence, $c^2-3d^2\neq0$. Thus, $\Q(\sqrt{3})$ is closed under division.
\end{proof}}
\end{homeworkProblem}
\newpage
\begin{homeworkProblem}[]
\textbf{Exercise 4.1.13: Show Let $F = \left\{ \text{ all matrices of the form } \begin{bmatrix}
   a & b \\
    -b  & a\\
\end{bmatrix}, \ a, b \in \R \right\}$ is a field under the operations of matrix addition and multiplication. }
\begin{proof}
 Consider $\begin{bmatrix}
   a & b \\
    -b  & a\\
\end{bmatrix}, \begin{bmatrix}
   c & d \\
    -d  & c\\ \end{bmatrix}, \begin{bmatrix}
   e & f \\
    -f  & e\\ \end{bmatrix} \in F$ with $a, b, c, d, e, f \in \R.$
\\ \textbf{(closure, $+$)}
\[\begin{bmatrix}
   a & b \\
    -b  & a\\
\end{bmatrix} + \begin{bmatrix}
   c & d \\
    -d  & c\\ \end{bmatrix}
    =\begin{bmatrix}
   a+c & b+d \\
    -(b+d)  & a+c\\ \end{bmatrix} \in F \text{ since } a+c, b+d \in \R \]
\\  \textbf{(closure, $\cdot$)}
\[\begin{bmatrix}
   a & b \\
    -b  & a\\
\end{bmatrix} \cdot \begin{bmatrix}
   c & d \\
    -d  & c\\ \end{bmatrix}
    =\begin{bmatrix}
   ac-bd & ad+bc \\
    -(ad+bc)  & ac-bd\\ \end{bmatrix} \in F \text{ since } ac-bd, ad+bc \in \R \]
     \textbf{(associative, $+$)}
\[\left( \begin{bmatrix}
   a & b \\
    -b  & a\\
\end{bmatrix} + \begin{bmatrix}
   c & d \\
    -d  & c\\ \end{bmatrix} \right)
    + \begin{bmatrix}
   e & f \\
    -f  & e\\ \end{bmatrix} =
    \begin{bmatrix}
   a+c & b+d \\
    -(b+d)  & a+c\\ \end{bmatrix} 
    + \begin{bmatrix}
   e & f \\
    -f  & e\\ \end{bmatrix} = 
    \begin{bmatrix}
   (a+c)+e & (b+d)+f \\
    -(b+d)-f  & (a+c)+e\\ \end{bmatrix} 
    \]
    \[
     =\begin{bmatrix}
   a+(c+e) & b+(d+f) \\
    -b-(d+f)  & a+(c+e)\\ \end{bmatrix} =
    \begin{bmatrix}
   a & b \\
    -b  & a\\
\end{bmatrix} +
\begin{bmatrix}
   c+e & d+f \\
    -(d+f) & c+e\\ \end{bmatrix}=
    \begin{bmatrix}
   a & b \\
    -b  & a\\
\end{bmatrix} + \left(
\begin{bmatrix}
   c & d \\
    -d  & c\\ \end{bmatrix}
    + \begin{bmatrix}
   e & f \\
    -f  & e\\ \end{bmatrix}
\right)
    \]
      \textbf{(associative, $\cdot$)}
      \[\left( \begin{bmatrix}
   a & b \\
    -b  & a\\
\end{bmatrix} \cdot \begin{bmatrix}
   c & d \\
    -d  & c\\ \end{bmatrix} \right)
    \cdot \begin{bmatrix}
   e & f \\
    -f  & e\\ \end{bmatrix} =
    \begin{bmatrix}
   ac-bd & ad+bc \\
    -(ad+bc)  & ac-bd\\ \end{bmatrix} 
    \cdot \begin{bmatrix}
   e & f \\
    -f  & e\\ \end{bmatrix} = \]
    \[
    \begin{bmatrix}
   e(ac-bd)-f(ad+bc) & e(ad+bc)+f(ac-bd) \\
    -e(ad+bc)-f(ac-bd)  & e(ac-bd)-f(ad+bc)\\ \end{bmatrix} =
    \]
    \[
        \begin{bmatrix}
   eac-ebd-fad-fbc & ead+ebc+fac-fbd \\
    -ead-ebc-fac+fbd  & eac-ebd-fad-fbc\\ \end{bmatrix}=
    \]
    \[
    \begin{bmatrix}
  a (ec-fd)-b(ed+fc) & a(ed+fc)+b(ec-fd) \\
    -a(ed+fc)-b(ec-fd)  & a(ec-fd)-b(ed+fc)\\ \end{bmatrix}=\begin{bmatrix}
   a & b \\
    -b  & a\\
\end{bmatrix} \cdot
     \begin{bmatrix}
   ec-fd & ed+fc \\
    -(ed+fc)  & ec-fd\\ \end{bmatrix} =\]
    \[
    \begin{bmatrix}
   a & b \\
    -b  & a\\
\end{bmatrix} \cdot \left(
\begin{bmatrix}
   c & d \\
    -d  & c\\ \end{bmatrix}
    \cdot \begin{bmatrix}
   e & f \\
    -f  & e\\ \end{bmatrix}
\right)
    \]
  \textbf{(commutative, $+$)}
\[\begin{bmatrix}
   a & b \\
    -b  & a\\
\end{bmatrix} + \begin{bmatrix}
   c & d \\
    -d  & c\\ \end{bmatrix}
    =\begin{bmatrix}
   a+c & b+d \\
    -(b+d)  & a+c\\ \end{bmatrix} =\begin{bmatrix}
   c+a & d+b \\
    -(d+b)  & c+a\\ \end{bmatrix}= \begin{bmatrix}
   c & d \\
    -d  & c\\ \end{bmatrix}+\begin{bmatrix}
   a & b \\
    -b  & a\\
\end{bmatrix} \]  
  \textbf{(commutative, $\cdot$)}
  \[\begin{bmatrix}
   a & b \\
    -b  & a\\
\end{bmatrix} \cdot \begin{bmatrix}
   c & d \\
    -d  & c\\ \end{bmatrix}
    =\begin{bmatrix}
   ac-bd & ad+bc \\
    -(ad+bc)  & ac-bd\\ \end{bmatrix}=\begin{bmatrix}
   ca-db & cb+da \\
    -(da+cb)  & -bd+ca\\ \end{bmatrix}=\begin{bmatrix}
   c & d \\
    -d  & c\\ \end{bmatrix} \cdot
    \begin{bmatrix}
   a & b \\
    -b  & a\\
\end{bmatrix} 
    \]
  \textbf{(distributive from the right)}
  \[
  \left( \begin{bmatrix}
   a & b \\
    -b  & a\\
\end{bmatrix} + \begin{bmatrix}
   c & d \\
    -d  & c\\ \end{bmatrix} \right)
    \cdot \begin{bmatrix}
   e & f \\
    -f  & e\\ \end{bmatrix} =
    \begin{bmatrix}
   a+c & b+d \\
    -(b+d)  & a+c\\ \end{bmatrix} 
    \cdot \begin{bmatrix}
   e & f \\
    -f  & e\\ \end{bmatrix} =
  \]
  \[
    \begin{bmatrix}
   ae+ce-fb-fd & af+cf+be+de \\
    -be-de-af-cf  &  -bf-df+ae+ce \\ \end{bmatrix} =
    \begin{bmatrix}
   ae-bf & af+be \\
    -be-af  &  -bf+ae \\ \end{bmatrix}+
    \begin{bmatrix}
   ce-df & cf+de \\
    -de-cf  &  -df+ce \\ \end{bmatrix}=
  \]
  \[
   \begin{bmatrix}
   a & b \\
    -b  & a\\
\end{bmatrix} \cdot \begin{bmatrix}
   e & f \\
    -f  & e\\ \end{bmatrix} + \begin{bmatrix}
   c & d \\
    -d  & c\\ \end{bmatrix} 
    \cdot \begin{bmatrix}
   e & f \\
    -f  & e\\ \end{bmatrix}
  \]
    \textbf{(distributive from the left)}
    \[
 \begin{bmatrix}
   e & f \\
    -f  & e\\ \end{bmatrix} \cdot \left( \begin{bmatrix}
   a & b \\
    -b  & a\\
\end{bmatrix} + \begin{bmatrix}
   c & d \\
    -d  & c\\ \end{bmatrix} \right)
    =
   \begin{bmatrix}
   e & f \\
    -f  & e\\ \end{bmatrix}
    \cdot  \begin{bmatrix}
   a+c & b+d \\
    -(b+d)  & a+c\\ \end{bmatrix} 
 =
  \]
  \[
    \begin{bmatrix}
   ea+ec-fb-fd & eb+ed+fa+fc \\
    -fa-fc-eb-ed  &  -fb-fd+ea+ec \\ \end{bmatrix} =
    \begin{bmatrix}
   ea-fb & eb+fa \\
    -fa-eb  &  -fb+ea \\ \end{bmatrix}+
    \begin{bmatrix}
   ec-fd & ed+fc \\
    -fc-ed  &  -fd+ec \\ \end{bmatrix}=
  \]
  \[
   \begin{bmatrix}
   e & f \\
    -f  & e\\ \end{bmatrix}
    \cdot 
    \begin{bmatrix}
   a & b \\
    -b  & a\\
\end{bmatrix} + \begin{bmatrix}
   e & f \\
    -f  & e\\
    \end{bmatrix}
    \cdot \begin{bmatrix}
   c & d \\
    -d  & c\\ \end{bmatrix} 
    \]
    \textbf{(identity, $+$)} For any element in $F$,
    \[
     \begin{bmatrix}
   a & b \\
    -b  & a\\
\end{bmatrix} + \begin{bmatrix}
   0 & 0 \\
    0  & 0\\
    \end{bmatrix}=\begin{bmatrix}
   a+0 & b+0\\
    -b+0  & a+0\\
\end{bmatrix}=
\begin{bmatrix}
   a & b \\
    -b  & a\\
\end{bmatrix} =\begin{bmatrix}
   0 & 0 \\
    0  & 0\\
\end{bmatrix}
+
\begin{bmatrix}
   a & b \\
    -b  & a\\
\end{bmatrix}
    \]  
    
   \textbf{(identity, $\cdot$)} For any element in $F$,
    \[
     \begin{bmatrix}
   a & b \\
    -b  & a\\
\end{bmatrix} \cdot \begin{bmatrix}
   1 & 0 \\
    0  & 1\\
    \end{bmatrix}=\begin{bmatrix}
   a & b\\
    -b  & a\\
\end{bmatrix}. \text{ and }
\begin{bmatrix}
   1 & 0 \\
    0  & 1\\
    \end{bmatrix}
    \cdot 
    \begin{bmatrix}
   a & b \\
    -b  & a\\
\end{bmatrix} =\begin{bmatrix}
   a & b \\
    -b  & a\\
\end{bmatrix}
    \]  
   \textbf{(inverse, $+$)} Notice $-1 \cdot \begin{bmatrix}
    a & b \\
    -b  & a\\
    \end{bmatrix} = -\begin{bmatrix}
    -a & -b \\
    b  & -a\\
    \end{bmatrix}$. Then, 
      \[
     \begin{bmatrix}
   a & b \\
    -b  & a\\
\end{bmatrix} + -1 \cdot \begin{bmatrix}
    a & b \\
    -b  & a\\
    \end{bmatrix}=\begin{bmatrix}
   0 & 0\\
     0 & 0\\
\end{bmatrix}. \text{ and }
-1 \cdot \begin{bmatrix}
    a & b \\
    -b  & a\\
    \end{bmatrix}
    +
    \begin{bmatrix}
   a & b \\
    -b  & a\\
\end{bmatrix} =\begin{bmatrix}
   0 & 0 \\
   0  & 0\\
\end{bmatrix}
    \]  
   \textbf{(inverse, $\cdot$)} We will show for any $A \in F$, $A \neq \begin{bmatrix}
   0 & 0 \\
   0  & 0\\
\end{bmatrix}$, $A^{-1}=\frac{1}{a^2+b^2} \cdot \begin{bmatrix}
    a & -b \\
    b  & a\\
    \end{bmatrix}$. Note $b,a$ are not both $0$, so $a^2 + b^2 \neq 0$. Thus, $A^{-1} \in F$ and: 
    \[
    \frac{1}{a^2+b^2} \begin{bmatrix}
    a & b \\
    -b  & a\\
    \end{bmatrix}\cdot \begin{bmatrix}
    a & -b \\
    b  & a\\
    \end{bmatrix}=
    \frac{1}{a^2+b^2} \begin{bmatrix}
    a^2+b^2 & -ab+ba \\
    -ba+ab  & b^2+a^2\\
    \end{bmatrix}
    =
    \frac{1}{a^2+b^2} \begin{bmatrix}
    a^2+b^2 & 0 \\
    0  & a^2+b^2\\
    \end{bmatrix}=
    \begin{bmatrix}
    1 & 0 \\
    0  & 1\\
    \end{bmatrix}
    \]
 Since $(F, \cdot)$ is commutative, 
 \[
    \begin{bmatrix}
    a & -b \\
    b  & a\\
    \end{bmatrix}\cdot
    \frac{1}{a^2+b^2} \begin{bmatrix}
    a & b \\
    -b  & a\\
    \end{bmatrix}
    =
    \begin{bmatrix}
    1 & 0 \\
    0  & 1\\
    \end{bmatrix}
    \]
Thus, $(F, +, \cdot)$ is a field. 
 \end{proof}
\end{homeworkProblem}

\begin{homeworkProblem}[]
\textbf{Exercise 4.1.17:  Let $(x_0, y_0), (x_1, y_1), (x_2, y_2)$ be points in the Euclidean plane $\R^2$ such that $x_0, x_1, x_2$ are distinct. Show that
\[
f(x) =  \frac{y_0(x-x_1)(x-x_2)}{(x_0-x_1)(x_0-x_2)}+ \frac{y_1(x-x_0)(x-x_2)}{(x_1-x_0)(x_1-x_2)}+ \frac{y_2(x-x_0)(x-x_1)}{(x_2-x_0)(x_2-x_1)}
\]
defines a polynomial $f(x)$ such that $f(x_0)=y_0$, $f(x_1)=y_1$, and $f(x_2)=y_2$.
}
\begin{proof}
Because multiplication and addition are closed in $\R$ and because $x_0, x_1, x_2$ are distinct and division is closed in $\R - \{0\}$, we can see that the expansion of $f(x)$ will yield a degree 2 polynomial with coefficients in $\R$. Next, compute $f(x_0), f(x_1)$, and $f(x_2)$:
\[
f(x_0) =  \frac{y_0(x_0-x_1)(x_0-x_2)}{(x_0-x_1)(x_0-x_2)}+ \frac{y_1(x_0-x_0)(x_0-x_2)}{(x_1-x_0)(x_1-x_2)}+ \frac{y_2(x_0-x_0)(x_0-x_1)}{(x_2-x_0)(x_2-x_1)}=\frac{y_0(x_0-x_1)(x_0-x_2)}{(x_0-x_1)(x_0-x_2)}=y_0
\]
\[
f(x_1) =  \frac{y_0(x_1-x_1)(x_1-x_2)}{(x_0-x_1)(x_0-x_2)}+ \frac{y_1(x_1-x_0)(x_1-x_2)}{(x_1-x_0)(x_1-x_2)}+ \frac{y_2(x_1-x_0)(x_1-x_1)}{(x_2-x_0)(x_2-x_1)}= \frac{y_1(x_1-x_0)(x_1-x_2)}{(x_1-x_0)(x_1-x_2)} =y_1
\]
\[
f(x_2) =  \frac{y_0(x_2-x_1)(x_2-x_2)}{(x_0-x_1)(x_0-x_2)}+ \frac{y_1(x_2-x_0)(x_2-x_2)}{(x_1-x_0)(x_1-x_2)}+ \frac{y_2(x_2-x_0)(x_2-x_1)}{(x_2-x_0)(x_2-x_1)}= \frac{y_2(x_2-x_0)(x_2-x_1)}{(x_2-x_0)(x_2-x_1)} =y_2.
\]
Thus, $f(x)$ satisfies the given conditions.
\end{proof}

\end{homeworkProblem}

\begin{homeworkProblem}[Exercise 4.1.18: Use Lagrange's interpolation formula to find a polynomial $f(x)$ such that $f(1)=0$, $f(2)=1$, and $f(3)=4$.]
\problemAnswer{ Then, $x_0=1, y_0=0, x_1=2, y_1=1, x_2=3, y_2=4$, so applying Lagrange's interpolation formula, we have
\[
f(x) = \frac{0(x-2)(x-3)}{(1-2)(1-3)}+\frac{1(x-1)(x-3)}{(2-1)(2-3)}+\frac{4(x-1)(x-2)}{(3-1)(3-2)}=-(x-1)(x-3)+2(x-2)(x-2)
\]
\[
=(x-1)(-x+3+2x-4)=(x-1)(x-1)=x^2-2x+1.
\]
By inspection, we can see that $f(x)=x^2-2x+1$ satisfies the given conditions.
}
\end{homeworkProblem}

\end{flushleft}
\end{spacing}
\end{document}

%%%%%%%%%%%%%%%%%%%%%%%%%%%%%%%%%%%%%%%%%%%%%%%%%%%%%%%%%%%%%

%----------------------------------------------------------------------%
% The following is copyright and licensing information for
% redistribution of this LaTeX source code; it also includes a liability
% statement. If this source code is not being redistributed to others,
% it may be omitted. It has no effect on the function of the above code.
%----------------------------------------------------------------------%
% Copyright (c) 2007, 2008, 2009, 2010, 2011 by Theodore P. Pavlic
%
% Unless otherwise expressly stated, this work is licensed under the
% Creative Commons Attribution-Noncommercial 3.0 United States License. To
% view a copy of this license, visit
% http://creativecommons.org/licenses/by-nc/3.0/us/ or send a letter to
% Creative Commons, 171 Second Street, Suite 300, San Francisco,
% California, 94105, USA.
%
% THE SOFTWARE IS PROVIDED "AS IS", WITHOUT WARRANTY OF ANY KIND, EXPRESS
% OR IMPLIED, INCLUDING BUT NOT LIMITED TO THE WARRANTIES OF
% MERCHANTABILITY, FITNESS FOR A PARTICULAR PURPOSE AND NONINFRINGEMENT.
% IN NO EVENT SHALL THE AUTHORS OR COPYRIGHT HOLDERS BE LIABLE FOR ANY
% CLAIM, DAMAGES OR OTHER LIABILITY, WHETHER IN AN ACTION OF CONTRACT,
% TORT OR OTHERWISE, ARISING FROM, OUT OF OR IN CONNECTION WITH THE
% SOFTWARE OR THE USE OR OTHER DEALINGS IN THE SOFTWARE.
%----------------------------------------------------------------------%
