\documentclass[12 pt]{article}
% Change "article" to "report" to get rid of page number on title page
\usepackage{amsmath,amsfonts,amsthm,amssymb}
\usepackage{setspace}
\usepackage{Tabbing}
\usepackage{fancyhdr}
\usepackage{lastpage}
\usepackage{extramarks}
\usepackage{chngpage}
\usepackage{indentfirst}
\usepackage{soul,color}
\usepackage{graphicx,float,wrapfig}
\usepackage{gauss}
\usepackage{dcolumn}
\newcolumntype{2}{D{.}{}{2.0}}
\usepackage{multicol}
\usepackage{Tabbing}
\usepackage{fancyhdr}
\usepackage{lastpage}
\usepackage{extramarks}
\usepackage{enumerate}
\usepackage{mathtools}

\graphicspath{ {/home/user/Documents/} }

\makeatletter
\renewcommand\section{\@startsection{section}{1}{\z@}%
                                  {-3.5ex \@plus -1ex \@minus -.2ex}%
                                  {2.3ex \@plus.2ex}%
                                  {\normalfont\bfseries}
                                }
\makeatother

\usepackage{etoolbox}
\makeatletter
\patchcmd\g@matrix
 {\vbox\bgroup}
 {\vbox\bgroup\normalbaselines}% restore the standard baselineskip
 {}{}
\makeatother

% In case you need to adjust margins:
\topmargin=-0.45in      %
\evensidemargin=0in     %
\oddsidemargin=0in      %
\textwidth=6.5in        %
\textheight=9.0in       %
\headsep=0.25in         %

% Homework Specific Information
\newcommand{\hmwkTitle}{Subgroup HW}
\newcommand{\hmwkDueDate}{Monday,\ October\ 12,\ 2015}
\newcommand{\hmwkClass}{Math\ 620}
\newcommand{\hmwkClassTime}{10:00}
\newcommand{\hmwkClassInstructor}{Boynton}
\newcommand{\hmwkAuthorName}{Kailee\ Gray}

\newtheorem{theorem}{Theorem}[section]
\newtheorem{lemma}[theorem]{Lemma}
\newtheorem{proposition}[theorem]{Proposition}
\newtheorem{corollary}[theorem]{Corollary}


\newenvironment{definition}[1][Definition]{\begin{trivlist}
\item[\hskip \labelsep {\bfseries #1}]}{\end{trivlist}}
\newenvironment{example}[1][Example]{\begin{trivlist}
\item[\hskip \labelsep {\bfseries #1}]}{\end{trivlist}}
\newenvironment{remark}[1][Remark]{\begin{trivlist}
\item[\hskip \labelsep {\bfseries #1}]}{\end{trivlist}}




% Setup the header and footer
\pagestyle{plain}                                                       %
\lhead{\hmwkAuthorName}                                                 %
\chead{\hmwkClass\ (\hmwkClassInstructor\ \hmwkClassTime): \hmwkTitle}  %
\rhead{\firstxmark}                                                     %
\lfoot{\lastxmark}                                                      %
\cfoot{}                                                                %
\rfoot{Page\ \thepage\ of\ \pageref{LastPage}}                          %
\renewcommand\headrulewidth{0.4pt}                                      %
\renewcommand\footrulewidth{0.4pt}                                      %

% This is used to trace down (pin point) problems
% in latexing a document:
%\tracingall

%%%%%%%%%%%%%%%%%%%%%%%%%%%%%%%%%%%%%%%%%%%%%%%%%%%%%%%%%%%%%
% Some tools
\newcommand{\enterProblemHeader}[1]{\nobreak\extramarks{#1}{#1 continued on next page\ldots}\nobreak%
                                    \nobreak\extramarks{#1 (continued)}{#1 continued on next page\ldots}\nobreak}%
\newcommand{\exitProblemHeader}[1]{\nobreak\extramarks{#1 (continued)}{#1 continued on next page\ldots}\nobreak%
                                   \nobreak\extramarks{#1}{}\nobreak}%

\newlength{\labelLength}
\newcommand{\labelAnswer}[2]
  {\settowidth{\labelLength}{#1}%
   \addtolength{\labelLength}{0in}%
   \changetext{}{-\labelLength}{}{}{}%
   \noindent\fbox{\begin{minipage}[c]{\columnwidth}#2\end{minipage}}%
   \marginpar{\fbox{#1}}%

   % We put the blank space above in order to make sure this
   % \marginpar gets correctly placed.
   \changetext{}{+\labelLength}{}{}{}}%

\setcounter{secnumdepth}{0}
\newcommand{\homeworkProblemName}{}%
\newcounter{homeworkProblemCounter}%
\newenvironment{homeworkProblem}[1][\arabic{homeworkProblemCounter}]%
  {\stepcounter{homeworkProblemCounter}%
   \renewcommand{\homeworkProblemName}{#1}%
   \section{\homeworkProblemName}%
   \noindent 
   
   \enterProblemHeader{\homeworkProblemName}}%
  {\exitProblemHeader{\homeworkProblemName}}%

\newcommand{\problemAnswer}[1]
  {\noindent\begin{minipage}[c]{\columnwidth}#1\end{minipage}}%

\newcommand{\problemLAnswer}[1]
    {\noindent\begin{minipage}[c]{\columnwidth}#1\end{minipage}}%

\newcommand{\homeworkSectionName}{}%
\newlength{\homeworkSectionLabelLength}{}%
\newenvironment{homeworkSection}[1]%
  {% We put this space here to make sure we're not connected to the above.
   % Otherwise the changetext can do funny things to the other margin

   \renewcommand{\homeworkSectionName}{#1}%
   \settowidth{\homeworkSectionLabelLength}{\homeworkSectionName}%
   \addtolength{\homeworkSectionLabelLength}{0 in}%
   \changetext{}{-\homeworkSectionLabelLength}{}{}{}%
   \subsection{\homeworkSectionName}%
   \enterProblemHeader{\homeworkProblemName\ [\homeworkSectionName]}}%
  {\enterProblemHeader{\homeworkProblemName}%

   % We put the blank space above in order to make sure this margin
   % change doesn't happen too soon (otherwise \sectionAnswer's can
   % get ugly about their \marginpar placement.
   \changetext{}{+\homeworkSectionLabelLength}{}{}{}}%

\newcommand{\sectionAnswer}[1]
  {\noindent\begin{minipage}[c]{\columnwidth}#1\end{minipage}}%
   \enterProblemHeader{\homeworkProblemName}\exitProblemHeader{\homeworkProblemName}%
  
 \newcommand{\R}{{\mathbb R}}
          \newcommand{\nil}{\varnothing}
          \newcommand{\N}{{\mathbb N}}
          \newcommand{\Z}{{\mathbb Z}}
        \newcommand{\MOD}{{ \ (\text{mod} \ }}

 \newcommand{\C}{{\mathbb C}}
  \newcommand{\Q}{{\mathbb Q}}
  
%%%%%%%%%%%%%%%%%%%%%%%%%%%%%%%%%%%%%%%%%%%%%%%%%%%%%%%%%%%%%
% Make title
\title{\vspace{2in}\textmd{\textbf{\hmwkClass:\ \hmwkTitle}}\\\normalsize\vspace{0.1in}\small{Due\ on\ \hmwkDueDate}\\\vspace{0.1in}\large{\textit{\hmwkClassInstructor\ \hmwkClassTime}}\vspace{3in}}
\date{}
\author{\textbf{\hmwkAuthorName}}
%%%%%%%%%%%%%%%%%%%%%%%%%%%%%%%%%%%%%%%%%%%%%%%%%%%%%%%%%%%%%

\begin{document}
\begin{spacing}{1.75}
\maketitle
\newpage
% Uncomment the \tableofcontents and \newpage lines to get a Contents page
% Uncomment the \setcounter line as well if you do NOT want subsections
%       listed in Contents
%\setcounter{tocdepth}{1}
%\tableofcontents
%\newpage

% When problems are long, it may be desirable to put a \newpage or a
% \clearpage before each homeworkProblem environment

\clearpage

\begin{homeworkProblem} [Exercise 3.2.4: Show that $\{ (1), (12)(34), (13)(24), (14)(23) \}$ is a subgroup of $S_4$. ]

\problemAnswer{ 
Let $G = \{ (1), (12)(34), (13)(24), (14)(23) \}$. Notice that $G$ and $S_4$ are finite sets. $G$ contains only permutations of $\{1,2,3,4\}$ and $S_4$ contains all possible permutations of $\{1, 2, 3, 4\}$ so $G$ is a subset of $S_4$. From proposition 3.1.6 we know $S_4$ is a group under the operation of composition of functions. Thus, by corollary 3.2.4, it suffices to show for any $a, b \in G$, $a \circ b \in G$. Since $|G|=4$, we will verify this property holds by calculating every possible composition in $G$:

\[
\begin{array}{l || l | l | l | l}
\circ & (1)& (12)(34)& (13)(24) & (14)(23) \\
\hline\hline
(1) & (1)& (12)(34)& (13)(24) & (14)(23)\\
\hline
(12)(34)&(12)(34)& (1)& (14)(23) & (13)(24)\\ 
\hline
(13)(24) & (13)(24) & (14)(23)& (1) & (12)(34)\\ 
\hline
(14)(23) & (14)(23)& (13)(24)& (12)(34) & (1)\\ 
\end{array}
	\]
The table above shows the composition of any two elements in $G$ is contained in $G$. Thus, $G$ is a subgroup of $S_4$.
}
\end{homeworkProblem}

\begin{homeworkProblem}[Exercise 3.2.11: Let $S$ be a set, and let $a$ be a fixed element of $S$. Show that $\{ \sigma \in \text{Sym} (S) | \ \sigma(a)=a\}$ is a subgroup of Sym$(S)$. ]
\problemAnswer{
Let $H=\{ \sigma \in \text{Sym} (S) | \ \sigma(a)=a\}$. Then, by definition of $H$, for any $\sigma \in H$, $\sigma \in $ Sym$(S)$, so $H \subseteq $ Sym$(S)$. 
Consider any $\sigma, \tau \in H$. Then, $\sigma(a)=a$ and $\tau(a)=a$. 
By definition 2.3.1, a function $\sigma$ is a permutation of $S$ if $\sigma$ is one-to-one and onto. Therefore, $\sigma$ and $\tau$ are well-defined, one-to-one, and onto. Then, $\sigma\tau (a)=\sigma(\tau (a)) = \sigma (a) = a$ implies $\sigma \tau \in H$ for any $\sigma, \tau \in H$. Also if $\tau(a)=a$,  $\tau^{-1}(a)=a$ which implies $\tau^{-1} \in H$. We know $\sigma \tau \in H$ for any $\sigma, \tau \in H$; so since $\sigma, \tau^{-1} \in H$, $\sigma \tau^{-1} \in H$. \\
Notice $\delta = (1) \in $ Sym$(S)$. Also, since $\delta(a)=a$, $\delta \in H$. So, $H \neq \O$. 
Thus, for any $\sigma, \tau \in H$, $\sigma \tau^{-1} \in H$. By corollary 3.2.3, $H$ is a subgroup of Sym$(S)$. 
}
\end{homeworkProblem}

\begin{homeworkProblem}[Exercise 3.2.15: Prove that any cyclic group is abelian.]
\problemAnswer{
 \begin{proof}
Let $G$ be any cyclic group. By definition 3.2.5, there exists some $a \in G$ such that $\left< a \right>=G$. So, for any $x, y \in G$, $x = a^{n_1}$ and $y = a^{n_2}$ for some $n_1, n_2 \in \Z$. Then, $xy = a^{n_1}a^{n_2}$. By definition 3.1.4', $a^{n_1}a^{n_2}=a^{n_1+n_2}$. Then, addition in $\Z$ is commutative, so $n_1+n_2 = n_2 + n_1$. Therefore, $xy = a^{n_2 + n_1}=a^{n_2}a^{n_1}= yx$. For any $x, y \in G$ we have $xy= yx$; hence $G$ is abelian. 
 \end{proof} }
\end{homeworkProblem}

\begin{homeworkProblem}[Exercise 3.2.17: Prove that the intersection of any collection of subgroups of a group is again a subgroup.]
\problemAnswer{
\begin{proof}
Let $G$ be a group with identity element $e$. Consider some collection of subgroups of $G$, indexed in no particular order by $k \in K$. Then consider $L =\bigcap_K H_k$. $H_k \subseteq G$ for all $k$ so $\bigcap_K H_k \subseteq G$.  Notice $L\neq\O$ since all subsets of $G$ must contain $e$.  If $L = \{ e\}$, then $L$ is trivially a subgroup, so consider $a,b \in L$. Then $a, b \in H_k$ for all $k$. Since all $H_k \leq G$, $ab \in H_k$ for all $k$ which implies $ab \in L$. Also, if $a \in H_k$ for all $k$, $H_k$ are groups, so $a^{-1} \in H_k$ for all $k$. Thus, $a^{-1} \in L$. Thus, by proposition 3.2.2, $L$ is a subgroup of $G$. 
\end{proof}

}
\end{homeworkProblem}

\begin{homeworkProblem}[Exercise 3.2.19: Let $G$ be a group and let $a \in G$. The set $C(a)=\{ x \in G \ | \ xa=ax\}$ of all elements of $G$ that commute with $a$ is called the \textbf{centralizer} of $a$.  \\ \textbf{(a) Show that $C(a)$ is a subgroup of $G$.}]
\problemAnswer{
\begin{proof}
	Let $G$ be a group and let $a \in G$. Define the set $C(a)=\{ x \in G \ | \ xa=ax\}$. Notice for all $x \in C(a)$, $x \in G$, so $C(a) \subseteq G$. Also, $G$ is a group so $G$ contains an identity element $e$, and for any $a \in G$, $ea=a=ea$. Thus, $e \in C(a)$. Consider any $x \in C(a)$. Then, $xa=ax$. Since $x \in C(a)$, $x \in G$, so there exists $x^{-1} \in G$ such that $x^{-1}x = e$. $xa=ax$ implies $x^{-1}xax^{-1}=x^{-1}axx^{-1}$. Thus, $eax^{-1}=x^{-1}ae$, so $ax^{-1}=x^{-1}a$ implies $x^{-1} \in C(a)$. Finally, consider any $x, y \in C(a)$. Then, $xa = ax$ and $ya=ay$. If $ya=ay$, then $xya=xay$. But, $xa=ax$, so $xya=axy$. Thus, $x, y \in C(a)$. By proposition 3.2.2, $C(a)\leq G$.
	 \\
\end{proof}}
\noindent \textbf{(b)Show that $\left< a \right> \subseteq C(a)$. }
\begin{proof}
	Consider some $x \in \left<a\right>$. Then, $x = a^n$ for some $n \in \Z$. Notice $ax = aa^n=a^{1+n}=a^{n+1}=a^n a = xa$. Thus, $x \in C(a)$. \end{proof}

\noindent \textbf{(c)Compute $C(a)$ if $G=S_3$ and $a=(123)$. }
 Note, $S_3 = \{ (1), (123),(132),(23),(13),(12) \}$. Since $(1)$ is the identity element of $S_3$, $(1)(123)=(123)(1)$ implies $(1)\in C(a)$. Also, any element commutes with itself, so $(123) \in C(a)$. We will compose all remaining elements of $S_3$ with $(123)$ to check for commutativity: 
 \[
 \begin{array}{l||l|l|l|l}
   & (132)& (23) & (13)& (12)\\
 \hline
 \textbf{permutation from top row }\circ(123)  & (1)& (12) & (23)& (13)\\	
 \hline
(123)\circ \textbf{permutation from top row }  & (1)& (13) & (12)& (23)\\	
 \end{array}
 \]
 Thus, $C((123))=\{(1), (123), (132)\}=\left< (123) \right>$. \\
 \noindent \textbf{(d)Compute $C(a)$ if $G=S_3$ and $a=(12)$. }
 Since $(1)$ is the identity element of $S_3$, $(1)(12)=(12)(1)$ implies $(1)\in C(a)$. Also, any element commutes with itself, so $(12) \in C(a)$. We will compose all remaining elements of $S_3$ with $(12)$ to check for commutativity: 
 \[
 \begin{array}{l||l|l|l|l}
   & (132)& (23) & (13)& (123)\\
 \hline
 \textbf{permutation from top row }\circ(12)  & (13)& (123) & (132)& (23)\\	
 \hline
(12)\circ \textbf{permutation from top row }  & (23)& (132) & (123)& (13)\\	
 \end{array}
 \]
 Thus, $C((12))=\{(1), (12)\}.$ \\
\end{homeworkProblem}
\newpage
\begin{homeworkProblem}[Exercise 3.2.21: Let $G$ be a group. The set $Z(G)=\{ x \in G \ | \ xg = gx \text{ for all } g \in G\}$ is called the \textbf{center} of $G$.  \textbf{(a) Show that $Z(G)$ is a subgroup of $G$.}]
\begin{proof}
	Note $e \in Z(G)$ since $eg=ge$ for all $g \in G$. Also, all $x \in Z(G) \in G$ by defintion of $Z(G)$ so $Z(G) \subseteq G$. Consider any $x, y \in Z(G)$. Then, for all $g \in G$, $xg = gx$ and $yg=gy$. If $yg=gy$, $xyg=xgy$. Since $xg=gx$ we have $xyg=gxy$. Thus, $xy \in Z(G)$. Since $x \in G$, $x^{-1} \in G$ such that $x^{-1}x=e=xx^{-1}$. Then, if $x \in Z(G)$, $xg = gx$ for all $g \in G$. Equivalently, $x^{-1}xgx^{-1}=x^{-1}gxx^{-1}$ and $egx^{-1}=x^{-1}ge$. So, $gx^{-1}=x^{-1}g$ implies $x^{-1} \in Z(G)$ for any $x \in Z(G)$. By proposition 3.2.2, $Z(G)$ is a subgroup of $G$.  
\end{proof}
\noindent \textbf{(b)Show that $Z(G)=\bigcap_{a \in G}C(a)$. }
\begin{proof}
	\textbf{(show $Z(G)\subseteq \bigcap_{a \in G}C(a)$)} Consider any $x \in Z(G)$. Then, for all $g \in G$, $xg=gx$. Equivalently for all $a \in G$, $xa=ax$. Thus, $x \in C(a)$ for all $a \in G$ so $x \in \bigcap_{a \in G}C(a)$.\\	
	\textbf{(show $Z(G)\supseteq \bigcap_{a \in G}C(a)$)} Consider any $x \in \bigcap_{a \in G}C(a)$. Then, for all $a \in G$, $x \in C(a)$. So, for all $a \in G$, $xa=ax$. Equivalently, for all $g \in G$, $xg=gx$ so $x \in Z(G)$. 	
\end{proof}

\noindent \textbf{(c) Compute the center of $S_3$. }
Consider the multiplication table of $S_3$:
  \[
 \begin{array}{c||c|c|c|c|c|c}
   \circ & (1) & (12) & (13)& (23) & (123)& (132)\\
 \hline
  (1) & (1) & (12) & (13)& (23) & (123)& (132)\\
 \hline 
  (12) & (12) & (1) & (132)& (123) & (23)& (13)\\
 \hline
   (13) & (13) & (123) & (1)& (132) & (12)& (23)\\
 \hline
   (23) & (23) & (132) & (123)& (1) & (13)& (12)\\
 \hline
   (123) & (123) & (13) & (23)& (12) & (132)& (1)\\
 \hline
   (132) & (132) & (23) & (12)& (123) & (1)& (123)\\
 \end{array}
 \]
 By inspection, $Z(G)=\{(1)\}$. Also note that $C(123) \cap C(12)=\{(1) \}$ so $Z(G)=\{(1)\}$ also follows from exercise 3.2.19 parts c, d.
\end{homeworkProblem}

\begin{homeworkProblem}[Exercise 3.2.23: Let $G$ be a cyclic group, and let $a, b$ be elements of $G$ such that neither $a=x^2$ nor $b=x^2$ has a solution in $G$. Show that $ab=x^2$ does have a solution in $G$. ]
\begin{proof}
	Let $G$ be a cyclic group, and let $a, b$ be elements of $G$ such that neither $a=x^2$ nor $b=x^2$ has a solution in $G$. Since $G$ is cyclic, $G=\left< g \right>$ for some $g \in G$ and so $a, b \in G$ imply $a = g^n$ and $b = g^m$ for some $n,m \in \Z$. If $m$ or $n$ are even, then $m = 2k$ or $n = 2l$ for $k, l \in \Z$. Then, $a=g^{2k}=(g^k)^2$ and $b = g^{2l}=(g^l)^2$. But $a \neq x^2$ and $b \neq x^2$ for any $x \in G$, so $m, n$ must be odd. Also, $G$ is a group so we can write $ab=g^{m+n}$. Then, since $m, n$ are odd, $m + n$ is even so $\frac{m+n}{2} \in \Z$ which implies $g^\frac{m+n}{2} \in G$. We can write $ab=(g^{\frac{m+n}{2}})^2$. Thus, $ab = x^2$ has a solution in $G$, namely $g^{\frac{m+n}{2}}$.
\end{proof}

\end{homeworkProblem}

\begin{homeworkProblem}[Exercise 3.2.24: Let $G$ be a group with $a, b \in G$.  \textbf{(a) Show that $o(a^{-1})=o(a)$. }]
\begin{proof}
	Let $G$ be a group with $a \in G$. Consider $o(a) < \infty$ and suppose $o(a)=n$ for $n \in \Z^+$. Then, by definition 3.2.7, $a^n=e$ and $n$ is the smallest such positive integer. Since $G$ is a group, $a^{-1} \in G$ and $(a^{-1})^n=a^{-n} \in G$. If $a^n=e$, then $a^na^{-n}=ea^{-n}$. By exponential laws of groups, $a^na^{-n}=a^{n-n}=a^0=e$. Thus, $ea^{-n}=a^{-n}=e$. Again, by the exponential laws of groups $a^{-n}=(a^{-1})^n$. Hence, $(a^{-1})^n=e$. Now, suppose there is $m \in \Z^+, m<n$ such that $(a^{-1})^m=e$. Then, following a similar argument as above, $(a^{-1})^m=e$ implies $a^{-m}a^{m}=ea^m$ which implies $e=a^m$. The order of $a$ is $n$, so the existence of such an $m<n$ contradicts the definition of order of $a$. Thus, when $o(a) < \infty$, $o(a^{-1})=n=o(a)$. 
	Next, suppose $o(a)= \infty$. Then, there does not exists a $k \in \Z^+$ such that $a^k = e$. If $o(a^{-1}) \neq \infty$, then there exists some $k \in \Z^+$ such that $(a^{-1})^k = e$ where $k$ is the smallest such integer. But from the first part of the proof we know this implies $a^k =e$ which contradicts our assumption that $o(a) = \infty$. Thus, if $o(a)=\infty$, $o(a^{-1}) = \infty$. 
\end{proof}\newpage
\noindent \textbf{(b) Show that $o(ab)=o(ba)$. }
\begin{proof}
		Let $G$ be a group with $a, b \in G$. First, consider $o(ab) < \infty$. Then, suppose $o(ab)=n$ for $n \in \Z^+$. Then, by definition 3.2.7, $(ab)^n=e$ and $n$ is the smallest such positive integer. By part (a), $o(ab)=n$ implies $o((ab)^{-1})=n$. So, $(ab)^{-n} =e$. We will manipulate $(ab)^n=e$ using the listed properties of groups to obtain our desired result:
	\[
		\begin{array}{c  c  c}
			(ab)^n & =e & \\
			(ab)(ab)\cdots (ab) &=e & \text{ by exponential laws of groups }\\
			a(ba)(ba)\cdots (ba)b & = e & \text{ by associative property }\\
			a(ba)^{n-1}b &=e &\\
			a(ba)^{n-1}ba &=ea & \text{ multiply on the right by a}\\
			a^{-1} a(ba)^{n-1}ba &=a^{-1}ea & \text{ multiply on the left by } a^{-1}\\
			(a^{-1} a)(ba)^{n-1}(ba) &=(a^{-1}e)a & \text{ by associative property } \\
			(e)(ba)^{n-1}(ba) &=(a^{-1})a & \text{ definition of identity and inverse elements }\\
			(e(ba)^{n-1})(ba) &=e & \text{ definition of inverse elements, associativity }\\
			(ba)^{n-1}(ba) &=e & \text{ definition of identity element }\\
			(ba)^{n}&=e & \text{ definition of identity element }. 
		\end{array}
\]
The above implies that for any $n \in \Z^+$, if $(ab)^n = e$, then $(ba)^n = e$. So, if there exists some $k \in \Z^+$ with $k \leq n$ and $(ba)^k = e$, then $(ab)^k=e$. This contradicts our assumption that $o(ab)=n$, since $n$ must be the smallest positive integer with $(ab)^n=e$. Thus, $n$ is the smallest positive integer such that $(ba)^n=e$ so $o(ba)=n$.

If $o(ab)=\infty$, then there is no integer $n \in \Z^+$ such that $(ab)^n = e$. If $o(ba) \neq \infty$, there there exists some integer $k \in \Z^+$ such that $(ba)^k = e$. From above, if $(ba)^k = e$, then $(ab)^k = e$ which contradicts our assumption that $o(ab)=\infty$. Thus, if $o(ab)=\infty$, then $o(ba)=\infty$.
Hence, $o(ab)=o(ba)$.
		\end{proof}
\noindent \textbf{(c) Show that $o(aba^{-1})=o(b)$. }
\begin{proof}
		Let $G$ be a group with $a, b \in G$. Then, $a^{-1}, e \in G$ such that $a^{-1}a=e$ and $aba^{-1} \in G$. By associative property we can write, $aba^{-1}=(ab)a^{-1}$. If two elements of $G$ are equal, their orders must be equal so, $o(aba^{-1})=o((ab)a^{-1})$. Then, $ab, a^{-1} \in G$, so by part (b), $o((ab)a^{-1}))= o(a^{-1}(ab))$. By associativity, $a^{-1}(ab)= (a^{-1}a)b=eb=b$. Since $a^{-1}(ab)=b$ and  $o((ab)a^{-1}))= o(a^{-1}(ab))$, $o((ab)a^{-1}))= o(b)$. Thus, $o(aba^{-1})=o(b)$ for any $a, b \in G$. 
	
\end{proof}
\end{homeworkProblem}

\begin{homeworkProblem}[Exercise 3.2.26: Let $G$ be a  group with $a, b \in G$. Assume that $o(a)$ and $o(b)$ are finite and relatively prime, and that $ab=ba$. Show that $o(ab)=o(a)o(b)$.]
\problemAnswer{
\begin{proof}

Let $G$ be a  group with $a, b \in G$. Without loss of generality, assume $a, b \neq e$.  Assume that $o(a)$ and $o(b)$ are finite and relatively prime, and that $ab=ba$. Then, $G$ is abelian. If $o(a)$ and $o(b)$ are finite and relatively prime, $o(a)=p$ and $o(b)=q$ for some $p, q \in \Z^{+}$ such that $\gcd(p,q)=1$. Then, $p, q$ are the smallest positive integers such that $a^p=e$ and $b^q=e$. Since $G$ is abelian, by exercise 17 in section 3.1, $(ab)^{pq}=a^{pq}b^{pq}$. Applying the exponential laws of groups, we obtain $a^{pq}b^{pq}=(a^p)^q (b^q)^p=e^q e^p=e$. Thus, $(ab)^{pq}=e$. Suppose $o(ab)\neq pq$ and $o(ab)=k$ where $k \in \Z^+$, $k < pq$ and $(ab)^k=e$. By proposition 3.2.8 (b), $(ab)^{pq}=e$ implies $o(ab) \ | \ pq$ and so $k \ | \ pq$. Because $p$ and $q$ are relatively prime, the only divisors of $pq$ are $\pm 1, \pm p, \pm q, and \pm pq$. Since $k \in \Z^+$, $k >0$, $k\neq pq$, so $k = 1, p, q$. We will consider all these cases:\\
\textbf{($k=1$)} If $k =1$, $(ab)^1=ab = e$. Then, $a^{-1}=b$ and $a=b^{-1}$. From part (a) of exercise 3.2.24, $o(a^{-1})=o(a)$. Since $a^{-1}=b$, $o(a^{-1})=o(b)$ which implies $o(a)=o(b)$. So, $p=q$. Then, $\gcd(p, q)=p = q \neq 1$ as assumed. Hence, $k \neq 1$. \\
\textbf{($k=p$)} If $k =p$, $(ab)^p = e$. By exercise 3.1.17, $(ab)^p = a^p b^p=e$. Since $o(a)=p$, $a^p = e$, so $a^p b^p=e$ implies $a^p b^p=a^p$. Thus, $b^p=e$. If $p>q$, by proposition 3.2.8(b), $q \ | \ p$. But, $p$ and $q$ are relatively prime, $q \not| \ p$ and so $p<q$. But, $p\not<q$ since $q$ is the smallest integer such that $b^q=e$.   Hence, $k \neq p$. \\
\textbf{($k=q$)} If $k =q$, $(ab)^q = e$. By exercise 3.1.17, $(ab)^q = a^q b^q=e$. Since $o(b)=q$, $b^q = e$, so $a^q b^q=e$ implies $a^q b^q=b^q$. Thus, $a^q=e$. If $q>p$, by proposition 3.2.8(b), $p \ | \ q$. But, $p$ and $q$ are relatively prime, $p \not| \ q$ and so $q<p$. But, $p\not<q$ since $q$ is the smallest integer such that $b^q=e$.   Hence, $k \neq p$. \\Thus, $pq$ is the smallest integer such that $(ab)^{pq}=e$ and $o(ab)=pq = o(a)o(b)$.  

	
\end{proof}}
\end{homeworkProblem}

\end{spacing}
\end{document}

%%%%%%%%%%%%%%%%%%%%%%%%%%%%%%%%%%%%%%%%%%%%%%%%%%%%%%%%%%%%%

%----------------------------------------------------------------------%
% The following is copyright and licensing information for
% redistribution of this LaTeX source code; it also includes a liability
% statement. If this source code is not being redistributed to others,
% it may be omitted. It has no effect on the function of the above code.
%----------------------------------------------------------------------%
% Copyright (c) 2007, 2008, 2009, 2010, 2011 by Theodore P. Pavlic
%
% Unless otherwise expressly stated, this work is licensed under the
% Creative Commons Attribution-Noncommercial 3.0 United States License. To
% view a copy of this license, visit
% http://creativecommons.org/licenses/by-nc/3.0/us/ or send a letter to
% Creative Commons, 171 Second Street, Suite 300, San Francisco,
% California, 94105, USA.
%
% THE SOFTWARE IS PROVIDED "AS IS", WITHOUT WARRANTY OF ANY KIND, EXPRESS
% OR IMPLIED, INCLUDING BUT NOT LIMITED TO THE WARRANTIES OF
% MERCHANTABILITY, FITNESS FOR A PARTICULAR PURPOSE AND NONINFRINGEMENT.
% IN NO EVENT SHALL THE AUTHORS OR COPYRIGHT HOLDERS BE LIABLE FOR ANY
% CLAIM, DAMAGES OR OTHER LIABILITY, WHETHER IN AN ACTION OF CONTRACT,
% TORT OR OTHERWISE, ARISING FROM, OUT OF OR IN CONNECTION WITH THE
% SOFTWARE OR THE USE OR OTHER DEALINGS IN THE SOFTWARE.
%----------------------------------------------------------------------%
