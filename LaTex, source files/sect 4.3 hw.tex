\documentclass[12 pt]{article}
% Change "article" to "report" to get rid of page number on title page
\usepackage{amsmath,amsfonts,amsthm,amssymb}
\usepackage{setspace}
\usepackage{Tabbing}
\usepackage{fancyhdr}
\usepackage{lastpage}
\usepackage{extramarks}
\usepackage{chngpage}
\usepackage{indentfirst}
\usepackage{soul,color}
\usepackage{graphicx,float,wrapfig}
\usepackage{gauss}
\usepackage{dcolumn}
\newcolumntype{2}{D{.}{}{2.0}}
\usepackage{multicol}
\usepackage{Tabbing}
\usepackage{fancyhdr}
\usepackage{lastpage}
\usepackage{extramarks}
\usepackage{enumerate}
\usepackage{enumitem}

\usepackage{mathtools}
\usepackage{tikz}
\usetikzlibrary{positioning}
\usepackage{float}
\usepackage{wrapfig}



\graphicspath{ {/home/user/Documents/} }

\makeatletter
\renewcommand\section{\@startsection{section}{1}{\z@}%
                                  {-3.5ex \@plus -1ex \@minus -.2ex}%
                                  {2.3ex \@plus.2ex}%
                                  {\normalfont\bfseries}
                                }
\makeatother

\usepackage{etoolbox}
\makeatletter
\patchcmd\g@matrix
 {\vbox\bgroup}
 {\vbox\bgroup\normalbaselines}% restore the standard baselineskip
 {}{}
\makeatother

% In case you need to adjust margins:
\topmargin=-0.75in      %
\evensidemargin=0in     %
\oddsidemargin=0in      %
\textwidth=6.5in        %
\textheight=9.0in       %
\headsep=0.25in         %

%\raggedbottom


% Homework Specific Information
\newcommand{\hmwkTitle}{$\S 4.3$ Existence of Roots}
\newcommand{\hmwkDueDate}{Monday,\ December \ 7,\ 2015}
\newcommand{\hmwkClass}{Math\ 620}
\newcommand{\hmwkClassTime}{10:00}
\newcommand{\hmwkClassInstructor}{Boynton}
\newcommand{\hmwkAuthorName}{Kailee\ Gray}

\newtheorem{theorem}{Theorem}[section]
\newtheorem{lemma}[theorem]{Lemma}
\newtheorem{proposition}[theorem]{Proposition}
\newtheorem{corollary}[theorem]{Corollary}


\newenvironment{definition}[1][Definition]{\begin{trivlist}
\item[\hskip \labelsep {\bfseries #1}]}{\end{trivlist}}
\newenvironment{example}[1][Example]{\begin{trivlist}
\item[\hskip \labelsep {\bfseries #1}]}{\end{trivlist}}
\newenvironment{remark}[1][Remark]{\begin{trivlist}
\item[\hskip \labelsep {\bfseries #1}]}{\end{trivlist}}




% Setup the header and footer
\pagestyle{plain}                                                       %
\lhead{\hmwkAuthorName}                                                 %
\chead{\hmwkClass\ (\hmwkClassInstructor\ \hmwkClassTime): \hmwkTitle}  %
\rhead{\firstxmark}                                                     %
\lfoot{\lastxmark}                                                      %
\cfoot{}                                                                %
\rfoot{Page\ \thepage\ of\ \pageref{LastPage}}                          %
\renewcommand\headrulewidth{0.4pt}                                      %
\renewcommand\footrulewidth{0.4pt}                                      %

% This is used to trace down (pin point) problems
% in latexing a document:
%\tracingall

%%%%%%%%%%%%%%%%%%%%%%%%%%%%%%%%%%%%%%%%%%%%%%%%%%%%%%%%%%%%%
% Some tools
\newcommand{\enterProblemHeader}[1]{\nobreak\extramarks{#1}{#1 continued on next page\ldots}\nobreak%
                                    \nobreak\extramarks{#1 (continued)}{#1 continued on next page\ldots}\nobreak}%
\newcommand{\exitProblemHeader}[1]{\nobreak\extramarks{#1 (continued)}{#1 continued on next page\ldots}\nobreak%
                                   \nobreak\extramarks{#1}{}\nobreak}%

\newlength{\labelLength}
\newcommand{\labelAnswer}[2]
  {\settowidth{\labelLength}{#1}%
   \addtolength{\labelLength}{0in}%
   \changetext{}{-\labelLength}{}{}{}%
   \noindent\fbox{\begin{minipage}[c]{\columnwidth}#2\end{minipage}}%
   \marginpar{\fbox{#1}}%

   % We put the blank space above in order to make sure this
   % \marginpar gets correctly placed.
   \changetext{}{+\labelLength}{}{}{}}%

\setcounter{secnumdepth}{0}
\newcommand{\homeworkProblemName}{}%
\newcounter{homeworkProblemCounter}%
\newenvironment{homeworkProblem}[1][\arabic{homeworkProblemCounter}]%
  {\stepcounter{homeworkProblemCounter}%
   \renewcommand{\homeworkProblemName}{#1}%
   \section{\homeworkProblemName}%
   \noindent 
   
   \enterProblemHeader{\homeworkProblemName}}%
  {\exitProblemHeader{\homeworkProblemName}}%

\newcommand{\problemAnswer}[1]
  {\noindent\begin{minipage}[c]{\columnwidth}#1\end{minipage}}%

\newcommand{\problemLAnswer}[1]
    {\noindent\begin{minipage}[c]{\columnwidth}#1\end{minipage}}%

\newcommand{\homeworkSectionName}{}%
\newlength{\homeworkSectionLabelLength}{}%
\newenvironment{homeworkSection}[1]%
  {% We put this space here to make sure we're not connected to the above.
   % Otherwise the changetext can do funny things to the other margin

   \renewcommand{\homeworkSectionName}{#1}%
   \settowidth{\homeworkSectionLabelLength}{\homeworkSectionName}%
   \addtolength{\homeworkSectionLabelLength}{0 in}%
   \changetext{}{-\homeworkSectionLabelLength}{}{}{}%
   \subsection{\homeworkSectionName}%
   \enterProblemHeader{\homeworkProblemName\ [\homeworkSectionName]}}%
  {\enterProblemHeader{\homeworkProblemName}%

   % We put the blank space above in order to make sure this margin
   % change doesn't happen too soon (otherwise \sectionAnswer's can
   % get ugly about their \marginpar placement.
   \changetext{}{+\homeworkSectionLabelLength}{}{}{}}%

\newcommand{\sectionAnswer}[1]
  {\noindent\begin{minipage}[c]{\columnwidth}#1\end{minipage}}%
   \enterProblemHeader{\homeworkProblemName}\exitProblemHeader{\homeworkProblemName}%
  
 \newcommand{\R}{{\mathbb R}}
          \newcommand{\nil}{\varnothing}
          \newcommand{\N}{{\mathbb N}}
          \newcommand{\Z}{{\mathbb Z}}
        \newcommand{\MOD}{{ \ (\text{mod} \ }}

 \newcommand{\C}{{\mathbb C}}
  \newcommand{\Q}{{\mathbb Q}}
  
%%%%%%%%%%%%%%%%%%%%%%%%%%%%%%%%%%%%%%%%%%%%%%%%%%%%%%%%%%%%%
% Make title
\title{\vspace{2in}\textmd{\textbf{\hmwkClass:\ \hmwkTitle}}\\\normalsize\vspace{0.1in}\small{Due\ on\ \hmwkDueDate}\\\vspace{0.1in}\large{\textit{\hmwkClassInstructor\ \hmwkClassTime}}\vspace{3in}}
\date{}
\author{\textbf{\hmwkAuthorName}}
%%%%%%%%%%%%%%%%%%%%%%%%%%%%%%%%%%%%%%%%%%%%%%%%%%%%%%%%%%%%%

\begin{document}
\begin{spacing}{1.75}
\maketitle

 
% Uncomment the \tableofcontents and   lines to get a Contents page
% Uncomment the \setcounter line as well if you do NOT want subsections
%       listed in Contents
%\setcounter{tocdepth}{1}
%\tableofcontents
% 

% When problems are long, it may be desirable to put a   or a
% \clearpage before each homeworkProblem environment

\begin{flushleft}
\begin{homeworkProblem}[Exercise 4.3.5: Let $\phi: F_1 \rightarrow F_2$ be an isomorphism of fields. Prove that $\phi(1)=1$. That is, prove that $\phi$ must map the multiplicative identity of $F_1$ to the multiplicative identity of $F_2$.]
\problemAnswer{
\begin{proof}
	Let $\phi: F_1 \rightarrow F_2$ be an isomorphism of fields. Since $F_1, F_2$ are fields they contain a multiplicative and additive idenity. Let $1_1$, $1_2$ denote the multiplicative identities in $F_1, F_2$, respectively; let $0_1$ and $0_2$ be the additive identities of $F_1$ and $F_2$, respectively. First, to show $\phi(1_1)$ has a multiplicative inverse in $F_2$ we will show that $\phi(1_1) \neq 0_2$:\\
	 Since $F_1, F_2$ are fields, there exists an additive inverse of every element in $F_1, F_2$.  Note that $\phi$ is an isomorphism so it preserves addition; also, $F_1, F_2$ are fields so addition and multiplication are associative. Then:
	 \begin{eqnarray*}
	 \phi(0_1) & = & \phi(0_1) + 0_2 \\
	 & = & \phi(0_1) + \phi(0_1) - \phi(0_1) \\
	 & = & \phi(0_1 + 0_1) - \phi(0_1) \\
	 & = & \phi(0_1) - \phi(0_1) \\
	 & = & 0_2.
	 \end{eqnarray*}
Since $\phi$ is a bijection, if $\phi(0_1)=0_2$, $\phi(1_1) \neq 0_2$ so $\phi(1_1)$ has a multiplicative inverse in $F_2$, denote by $\phi(1_1)^{-1}$. Following a similar process and justifications as above, we have: 
 \begin{eqnarray*}
	 \phi(1_1) & = & \phi(1_1) \cdot 1_2 \\
	 & = & \phi(1_1) \cdot (\phi(1_1)\cdot \phi(1_1)^{-1}) \\
	 & = & (\phi(1_1) \cdot \phi(1_1))\cdot \phi(1_1)^{-1}\\
	 & = & \phi(1_1 \cdot 1_1)\cdot \phi(1_1)^{-1} \\
	 & = & \phi(1_1 )\cdot \phi(1_1)^{-1} \\
	  & = & 1_2 .
	 \end{eqnarray*}
\end{proof} }
\end{homeworkProblem}
\begin{homeworkProblem}[Exercise 4.3.6: Let $F$ be a field, let $p(x)$ be an irreducible polynomial in $F \lbrack x \rbrack$, and let $E=\{ \lbrack a \rbrack \in F\lbrack x \rbrack \slash \left< p(x) \right> | a \in F \}$. Show that $E$ is a subfield of $F\lbrack x \rbrack \slash \left< p(x) \right>$. ]
\begin{proof}[\textbf{(prove $E$ is a subfield)}] First, since $p(x)$ is irreducible, note that $F\lbrack x \rbrack \slash \left< p(x) \right>$ is a field by theorem 4.3.6. Given the definition of $E$, it is clear $E$ is a subset of $F\lbrack x \rbrack \slash \left< p(x) \right>$; also $F \neq \O$ so $E \neq \O$. Thus, by exercise 4.3.4, it suffices to show $E$ is closed under addition, subtraction, multiplication, and division of $E$. \\
\textbf{(addition, subtraction)} Consider any $[a], [b] \in E$. Then, using the definition of $\boxplus$ given in theorem 5 of Boynton, 
$
[a] \boxplus [b] = [a+b] \in E \text{ since } F \text{ is closed under } +.
$
 If $[a], [b] \in E$, then $a, b \in F$ so $a, b $ have additive inverses in $F$, denote $-a, -b$. Thus
$
[a]-[b]=[a] \boxplus [-b] = [a-b]=[a + (-b)] \in E \text{ since } F \text{ is closed under } +$ and $ 
[b]-[a]=[b] \boxplus [-a] = [b-a]=[b + (-a)] \in E \text{ since } F \text{ is closed under } +.$
\\
\textbf{(multiplication, division)} Consider any $[a], [b] \in E$, $[a], [b] \neq 0$. Then, using the definition of $\boxdot$ given in theorem 5 of Boynton, 
$
[a] \boxdot [b] = [a\cdot b] \in E \text{ since } F \text{ is closed under } \cdot.
$
If $[a], [b] \in E$, then $a, b \in F$ and $[a], [b] \neq 0$ so $a, b $ have multiplicative inverses in $F$, denote $a^{-1}, b^{-1}$. Note
$
[a]\div[b]=[a] \boxdot [b]^{-1} = [a] \boxdot [b^{-1}]=[a \cdot b^{-1}] \in E \text{ since } F \text{ is closed under } \cdot.$ Also, $[b]\div[a]=[b] \boxdot [a]^{-1} = [b] \boxdot [a^{-1}]=[b \cdot a^{-1}] \in E \text{ since } F \text{ is closed under } \cdot.$\\
Hence, $E$ is a subfield of $F$.
\end{proof}
\textbf{ Prove $\phi: F \rightarrow E$ defined by $\phi(a)=[a]$ is an isomorphism of fields. }
\begin{proof}
\textbf{(well-defined)} Consider $a_1, a_2 \in F$ with $a_1 = a_2$. Then, $a_1 - a_2 = 0$ so $p(x) | (a_1 - a_2)$. Hence $[a_1]=[a_2]$.
\\
\textbf{(one to one)} Suppose $\phi(a_1) = \phi(a_2)$ for some $a_1, a_2 \in F$. Then, $[a_1]=[a_2]$ and $[a_1], [a_2] \in F[x] \slash \left< p(x) \right>$ imply $p(x) \ | \ a_1 - a_2$. So there exists some $q(x) \in F[x] \slash \left< p(x) \right>$ such that $a_1 - a_2 = p(x)q(x)$. Since $\deg(a_1 - a_2)=0$, $\deg(p(x) q(x))=0$ and so $\deg(p(x))+\deg(q(x))=0$. But $\deg(p(x))\geq 1$ so $\deg(q(x))<0$ and $q(x) = 0$,  $a_1  = a_2$. 
\textbf{(onto)} Consider any $[a] \in E$. Then, $a \in F$ and so $\phi(a)=[a]$.\\
\textbf{(preserves $+$)}   $\text{Let $a_1, a_2 \in F$. Then, }  \text{Let $a_1, a_2 \in F$. Then, } \phi(a_1 + a_2) = [a_1 + a_2]=[a_1] + [a_2]=\phi(a_1)+\phi(a_2).$\\
\textbf{(preserves $\cdot$)} $ \text{ Let } a_1, a_2 \in F. \text{ Then, } \phi(a_1 \cdot a_2) = [a_1 \cdot a_2]=[a_1] \cdot [a_2]= \phi(a_1)\cdot \phi(a_2)$.
\\
Thus, by definition 4.3.7, $\phi$ is an isomorphism of fields.
 \end{proof} 
\end{homeworkProblem}
 
\begin{homeworkProblem}[Exercise 4.3.8: Prove that $\R \lbrack x \rbrack \slash \left< x^2 + 2 \right>$ is isomorphic to $\C$. ]
\begin{proof}
By proposition 4.3.3 in Beachy, all elements in $\R \lbrack x \rbrack \slash \left< x^2 + 2 \right>$ are of the form $[a + bx]$. Define $\phi: \R \lbrack x \rbrack \slash \left< x^2 + 2 \right> \rightarrow \C$ by $\phi([a + bx])=a + bi \sqrt{2}$. We will show $\phi$ is an isomorphism.\\
\textbf{(well-defined)} Consider $[a_1+b_1x], [a_2+b_2 x] \in \R \lbrack x \rbrack \slash \left< x^2 + 2 \right>$ with $[a_1+b_1x] = [a_2+b_2 x]$. Then, by definition 4.3.2, $(x^2+2) \ | \ (a_1+b_1x - (a_2+b_2x))$ and so $(x^2+2)  \ | \ (a_1 - a_2+(b_1-b_2)x)$. Thus, there exists $q(x)$ such that $a_1 - a_2+(b_1-b_2)x = (x^2 + 2) q(x)$. Since $\deg(a_1 - a_2+(b_1-b_2)x)=1$ and $\deg(x^2 + 2)=2$, $\deg(q(x))<0$. Thus, $q(x)=0$ and so $a_1 - a_2 + (b_1-b_2)x=0$ which implies $a_1 - a_2 = 0$ and $b_1 - b_2 = 0$. Thus, $a_1+b_1 i \sqrt{2}=a_2+ b_2 i \sqrt{2}$ which implies $\phi ([a_1+b_1x]) = \phi ( [a_2+b_2 x] )$. 
\\
\textbf{(one to one)} Suppose $\phi([a_1+b_1x]) = \phi([a_2+b_2 x])$ for some $[a_1+b_1x], [a_2+b_2 x] \in \R \lbrack x \rbrack \slash \left< x^2 + 2 \right>$. Then, $a_1 + b_1 i \sqrt{2} = a_2 + b_2 i \sqrt{2}$; equivalently $a_1 = a_2 $ and $b_1 = b_2$. Thus, $a_1 + b_1 x = a_2 + b_2x$ and so $[a_1 + b_1 x] = [a_2 + b_2x]$. \\
\textbf{(onto)} Consider any $a+ bi \in \C$. Then, $a, b \in \R$ and so $$
\phi\left(\left[a+\frac{b}{\sqrt{2}} \ x\right]\right)=a + \frac{b}{\sqrt{2}}i \sqrt{2} = a + b i.
$$
\textbf{(preserves addition, multiplication)} Let $[a_1+b_1x], [a_2+b_2 x] \in \R \lbrack x \rbrack \slash \left< x^2 + 2 \right>$. Then, \\$
\phi([a_1+b_1x] + [a_2+b_2 x]) = \phi[a_1 + b_1 x+ a_2+b_2 x]=\phi[a_1 + a_2 + (b_1 + b_2)x] =a_1 + a_2 + (b_1 + b_2)i \sqrt{2} = a_1 + b_1i \sqrt{2} + a_2 + b_2 i \sqrt{2} =\phi([a_1+b_1x]) + \phi([a_2+b_2 x]).
$ Next, notice
$\phi([a_1+b_1x] \cdot [a_2+b_2 x]) = \phi( [(a_1+b_1x)\cdot (a_2+b_2 x)] )= \phi(a_1a_2 + (a_2b_1 + a_1b_2)x + b_1b_2 x^2)$. In $\R \lbrack x \rbrack \slash \left< x^2 + 2 \right>$, $[x^2 + 2] = 0$ so $[x]^2 = -[2]$. Thus, $ b_1b_2 x^2=-2b_1b_2$ so we have 
\begin{eqnarray*}
	\phi(a_1a_2 + (a_2b_1 + a_1b_2)x + -2b_1b_2 ) & = & \phi((a_1a_2 - 2b_1b_2)+(a_2b_1+a_1b_2)x) \\
	&=& (a_1a_2 - 2b_1b_2) + (a_2b_1+a_1b_2)i\sqrt{2} \\
	& = & (a_1 + b_1i  \sqrt{2})(a_2 + b_2 i \sqrt{2}) \\
	& = & \phi([a_1+b_1x] ) \cdot\phi( [a_2+b_2 x])
\end{eqnarray*}
By definition 4.3.7, $\phi$ is an isomorphism; hence $\R \lbrack x \rbrack \slash \left< x^2 + 2 \right>$ is isomorphic to $\C$.
\end{proof}
\end{homeworkProblem}

\begin{homeworkProblem}[Exercise 4.3.9: Prove that $\R\lbrack x \rbrack \slash \left<  x^2 + x + 1 \right>$ is isomorphic to $\C$.]

\begin{proof}
By proposition 4.3.3 in Beachy, all elements in $\R \lbrack x \rbrack \slash \left<  x^2 + x + 1 \right>$ are of the form $[a + bx]$. Define $\phi: \R \lbrack x \rbrack \slash \left< x^2 + x + 1 \right> \rightarrow \C$ by $\phi([a+bx])= a -\frac{b}{2} + \frac{b\sqrt{3}}{2} i$.
 We will show $\phi$ is an isomorphism.\\
\textbf{(well-defined)} Consider $[a_1+b_1x], [a_2+b_2 x] \in \R \lbrack x \rbrack \slash \left<  x^2 + x + 1 \right>$ with $[a_1+b_1x] = [a_2+b_2 x]$. Then, by definition 4.3.2, $(x^2 + x + 1) \ | \ (a_1+b_1x - (a_2+b_2x))$ and so $(x^2 + x + 1)  \ | \ (a_1 - a_2+(b_1-b_2)x)$. Thus, there exists $q(x)$ such that $a_1 - a_2+(b_1-b_2)x = (x^2 + x + 1) q(x)$. Since $\deg(a_1 - a_2+(b_1-b_2)x)=1$ and $\deg(x^2 + x + 1)=2$, $\deg(q(x))<0$. Thus, $q(x)=0$ and so $a_1 - a_2 + (b_1-b_2)x=0$ which implies $a_1 - a_2 = 0$ and $b_1 - b_2 = 0$. Thus, $a_1+b_1 \left( -\frac{1}{2} + \frac{\sqrt{3}}{2} i\right)=a_2+ b_2 \left( -\frac{1}{2} + \frac{\sqrt{3}}{2} i\right)$ which implies $\phi ([a_1+b_1x]) = \phi ( [a_2+b_2 x] )$. 
\\
\textbf{(one to one)} Suppose $\phi([a_1+b_1x]) = \phi([a_2+b_2 x])$ for some $[a_1+b_1x], [a_2+b_2 x] \in \R \lbrack x \rbrack \slash \left< x^2 + x+1 \right>$. Then, $$a_1 -\frac{b_1}{2} + \frac{b_1\sqrt{3}}{2} i=a_2 -\frac{b_2}{2} + \frac{b_2\sqrt{3}}{2} i \text{ and }a_1 - a_2 +\frac{b_2-b_1}{2}+ (b_1- b_2)\frac{\sqrt{3}}{2} i = 0. $$
 Therefore, $b_1 - b_2 = 0$ so $b_1 = b_2$ and $b_2 - b_1 = 0$. Also, $ a_1 - a_2 +\frac{b_2-b_1}{2}  = 0 \text{ so } a_1 - a_2 = 0. $ $\text{ Thus, } a_1 = a_2.
 $
 Therefore, $a_1 + b_1 x = a_2 + b_2x$ and so $[a_1 + b_1 x] = [a_2 + b_2x]$. \\
\textbf{(onto)} Consider any $a+ bi \in \C$. Then, $a, b \in \R$ so $a+\frac{b}{\sqrt{3}}, \frac{2b}{\sqrt{3}} \in \R$ and  $$
\phi\left(\left[a+\frac{b}{\sqrt{3}}+ \frac{2b}{\sqrt{3}} \ x\right]\right)=a+\frac{b}{\sqrt{3}} - \frac{1}{2}\frac{2b}{\sqrt{3}} + \frac{2b}{\sqrt{3}}\frac{\sqrt{3}}{2}i  = a + b i.
$$
\textbf{(preserves addition, multiplication)} Let $[a_1+b_1x], [a_2+b_2 x] \in \R \lbrack x \rbrack \slash \left<  x^2 + x + 1 \right>$. 
 \begin{eqnarray*}
 	\text{Then, } \phi([a_1+b_1x] + [a_2+b_2 x]) &= & \phi([a_1 + a_2 + (b_1 + b_2)x]) \\
 	& = & a_1 + a_2 + -\frac{b_1 + b_2}{2}+ \frac{ (b_1 + b_2)\sqrt{3}}{2} \\
 	&=& a_1 -\frac{b_1}{2} + \frac{b_1\sqrt{3}}{2} i+a_2 -\frac{b_2}{2} + \frac{b_2\sqrt{3}}{2} i  \\
 	& =&\phi([a_1+b_1x]) + \phi([a_2+b_2 x]). \text{ Next, notice }
\end{eqnarray*}
$\phi([a_1+b_1x] \cdot [a_2+b_2 x]) = \phi( [(a_1+b_1x)\cdot (a_2+b_2 x)] )= \phi(a_1a_2 + (a_2b_1 + a_1b_2)x + b_1b_2 x^2)$.\\ In $\R \lbrack x \rbrack \slash \left<  x^2 + x + 1\right>$, $[x^2 +x +1] = 0$ so $[x]^2 = -[x]-[1]$. Thus, $ x^2 b_1b_2 =-(x+1)b_1b_2$ so we have $ \phi((a_1a_2 - b_1b_2)+(a_2b_1+a_1b_2-b_1b_2)x)=$
\begin{eqnarray*}
	&=& (a_1a_2 - b_1b_2) + \frac{a_2b_1+a_1b_2-b_1b_2}{2} + \frac{\sqrt{3}}{2}(a_2b_1+a_1b_2-b_1b_2)i \\
	& = & \left(a_1 + \frac{b_1}{2} + b_1i\frac{  \sqrt{3}}{2}\right)  \left(a_2 + \frac{b_2}{2} + b_2i\frac{  \sqrt{3}}{2} \right) =  \phi([a_1+b_1x] ) \cdot\phi( [a_2+b_2 x])
\end{eqnarray*}
By definition 4.3.7, $\phi$ is an isomorphism; hence $\R \lbrack x \rbrack \slash \left< x^2 + x+1 \right>$ is isomorphic to $\C$.
 \end{proof} 
\end{homeworkProblem}

\begin{homeworkProblem}[Exercise 4.3.10: Is $\Q \lbrack x \rbrack \slash \left< x^2 +2 \right>$ isomorphic to $\Q \lbrack x \rbrack \slash \left< x^2 + 1 \right>$? ]

\begin{proof}
No. By proof similar to 4.3.8, 4.3.9, 4.3.13 and by Jason's approval to use, $\Q \lbrack x \rbrack \slash \left< x^2 +2 \right>$ is isomorphic to $\Q(i\sqrt{2})$ and  $\Q \lbrack x \rbrack \slash \left< x^2 + 1 \right>$ is isomorphic to $\Q(i)$. Thus $\Q \lbrack x \rbrack \slash \left< x^2 +2 \right>$ isomorphic to $\Q \lbrack x \rbrack \slash \left< x^2 + 1 \right>$ if and only if $\Q(i\sqrt{2})$ is isomorphic to $\Q(i)$. So, suppose $\Q(i\sqrt{2})$ is isomorphic to $\Q(i)$. 
First, note that the polynomial $x^2+2$ has a root in $\Q(i\sqrt{2})$, $i\sqrt{2} \in \Q(i \sqrt{2})$: $(i\sqrt{2})^2 + 2 = -2 + 2 = 0$. \\
 Suppose the polynomial $x^2+2$ has a root in $\Q(i)$. Then, there would be some $a, b \in \Q$ such that $(a+bi)^2 + 2 = 0$. Equivalently, $a^2 - b^2 + 2 = -2abi$. But, $a^2 - b^2 + 2 \in \R, \not \in \C$ and $-2abi \in \C$ so this is a contradiction. Thus, $x^2+2$ does not have a root in $\Q(i)$. \\
Now, we assumed that $\Q(i\sqrt{2})$ is isomorphic to $\Q(i)$ so there is an isomorphism $\phi : \Q(i\sqrt{2}) \rightarrow \Q(i)$. Let $x \in \Q(i \sqrt{2})$ be a root of $x^2 + 2$. Then, $x^2 + 2 = 0$ and so $\phi(x^2 + 2) = \phi(0)$. $\phi(0)=0$ and $\phi $ preserves multiplication and addition so $\phi(x)^2 + \phi(2) = 0$. Also, $\phi(2) = \phi(1 + 1) = \phi(1) + \phi(1) = 1 + 1 = 2$ which implies there is some $\phi(x) \in \Q(i)$ with $\phi(x)^2 + 2 = 0$ and so $\phi(x)$ is a root of $x^2 +2$. This is a contradiction since we proved $\Q(i)$ does not contain a root of $x^2 +2$. 
\end{proof}
\end{homeworkProblem}

\begin{homeworkProblem}[Exercise 4.3.13: Prove that $\Q \lbrack x \rbrack \slash \left<  x^2 - 3 \right> $ is isomorphic to $\Q(\sqrt{3})$.]
\problemAnswer{ 
\begin{proof}
Define 	
By proposition 4.3.3 in Beachy, all elements in $\Q \lbrack x \rbrack \slash \left< x^2 - 3 \right>$ are of the form $[a + bx]$. Define $\phi: \Q \lbrack x \rbrack \slash \left< x^2 -3 \right> \rightarrow \Q(\sqrt{3})$ by $\varphi([a+bx])=a+b\sqrt{3}$. We will show $\phi$ is an isomorphism.\\
\textbf{(well-defined)} Consider $[a_1+b_1x], [a_2+b_2 x] \in \Q \lbrack x \rbrack \slash \left< x^2 -3 \right>$ with $[a_1+b_1x] = [a_2+b_2 x]$. Then, by definition 4.3.2, $(x^2-3)  \ | \ (a_1 - a_2+(b_1-b_2)x)$. Thus, there exists $q(x)$ such that $a_1 - a_2+(b_1-b_2)x = (x^2 - 3) q(x)$. Since $\deg(a_1 - a_2+(b_1-b_2)x)=1$ and $\deg(x^2 - 3)=2$, $\deg(q(x))<0$. Thus, $q(x)=0$ and so $a_1 - a_2 + (b_1-b_2)x=0$ which implies $a_1 - a_2 = 0$ and $b_1 - b_2 = 0$. Thus, $a_1+b_1  \sqrt{3}=a_2+ b_2  \sqrt{3}$ which implies $\phi ([a_1+b_1x]) = \phi ( [a_2+b_2 x] )$. 
\\
\textbf{(one to one)} Suppose $\phi([a_1+b_1x]) = \phi([a_2+b_2 x])$ for some $[a_1+b_1x], [a_2+b_2 x] \in \Q \lbrack x \rbrack \slash \left< x^2 -3 \right>$. Then, $a_1 + b_1  \sqrt{3} = a_2 + b_2  \sqrt{3}$; because $a_1,a_2, b_1, b_1 \in \Q$ this is equivalent to $a_1 = a_2 $ and $b_1 = b_2$. Thus, $a_1 + b_1 x = a_2 + b_2x$ and so $[a_1 + b_1 x] = [a_2 + b_2x]$. \\
\textbf{(onto)} Consider any $a+ b\sqrt{3} \in \Q(\sqrt{3})$. Then, $a, b \in \Q$ and so $
\phi\left(\left[a+ b x\right]\right)= a + b \sqrt{3} .
$
\textbf{(preserves addition, multiplication)} Let $[a_1+b_1x], [a_2+b_2 x] \in \Q \lbrack x \rbrack \slash \left< x^2 -3 \right>$. Then, \\$
\phi([a_1+b_1x] + [a_2+b_2 x]) = \phi[a_1 + b_1 x+ a_2+b_2 x]=\phi[a_1 + a_2 + (b_1 + b_2)x] =a_1 + a_2 + (b_1 + b_2) \sqrt{3} = a_1 + b_1 \sqrt{3} + a_2 + b_2  \sqrt{3} =\phi([a_1+b_1x]) + \phi([a_2+b_2 x]).
$ Next, notice
$\phi([a_1+b_1x] \cdot [a_2+b_2 x]) = \phi( [(a_1+b_1x)\cdot (a_2+b_2 x)] )= \phi(a_1a_2 + (a_2b_1 + a_1b_2)x + b_1b_2 x^2)$. In $\Q \lbrack x \rbrack \slash \left< x^2 - 3 \right>$, $[x^2 - 3] = 0$ so $[x]^2 = [3]$. Thus, $ b_1b_2 x^2=3b_1b_2$ so we have 
\begin{eqnarray*}
	\phi(a_1a_2 + (a_2b_1 + a_1b_2)x + 3b_1b_2 ) & = & \phi((a_1a_2 +3 b_1b_2)+(a_2b_1+a_1b_2)x) \\
	&=& (a_1a_2 +3b_1b_2) + (a_2b_1+a_1b_2)\sqrt{3} \\
	& = & (a_1 + b_1  \sqrt{3})(a_2 + b_2  \sqrt{3}) \\
	& = & \phi([a_1+b_1x] ) \cdot\phi( [a_2+b_2 x])
\end{eqnarray*}
By definition 4.3.7, $\phi$ is an isomorphism; hence $\Q \lbrack x \rbrack \slash \left< x^2 -3 \right>$ is isomorphic to $\Q(\sqrt{3})$.

\end{proof}
}
\end{homeworkProblem}
\begin{homeworkProblem}[Exercise 4.3.14: Show that the polynomial $x^2 - 3$ has a root in $\Q(\sqrt{3})$ but not in $\Q(\sqrt{2})$. Explain why this implies $\Q(\sqrt{3})$ is not isomorphic to $\Q(\sqrt{2})$. ]
\begin{proof}
	First, show $x^2 - 3$ has a root in $\Q(\sqrt{3})$ but not in $\Q(\sqrt{2})$. Notice $\sqrt{3} \in \Q(\sqrt{3})$ and $\sqrt{3}^2 - 3 = 0$ so $x^2 - 3$ has a root in $\Q(\sqrt{3})$. Suppose $x^2 - 3$ has a root in $\Q(\sqrt{2})$. Then, there exists some $a + b \sqrt{2}$, with $a, b \in \Q$, such that $(a+b \sqrt{2})^2 - 3 = 0$. Simplifying,
	\[
	(a+b \sqrt{2})^2 - 3  = a^2 + 2b^2 + 2ab \sqrt{2} - 3 = (a^2 + 2b^2 - 3) + 2ab \sqrt{2}. 
	\]
	\[
	\text{ If } (a^2 + 2b^2 - 3) + 2ab \sqrt{2} = 0, \  a^2 + 2b^2 - 3= 0 \text{ and } 2ab = 0. \text{ Thus, } a=0 \text{ or } b = 0.
	\]
	Suppose $b=0$. Then, $a^2 + 2\cdot 0^2 - 3 = 0$ implies $a^2 = 3$ and so $a = \pm \sqrt{3} \not\in \Q$. Thus, $b \neq 0$. Suppose $a = 0$, then $0^2 + 2\cdot b^2 - 3 = 0$ implies $2b^2 = 3$ and so $b = \pm \sqrt{\frac{3}{2}} \not\in \Q$. Thus, $a \neq 0$. Hence, $x^2 - 3$ does not have a root in $\Q(\sqrt{2})$. \\
	Next, show $\Q(\sqrt{2})$ is not isomorphic to $\Q(\sqrt{3})$. Suppose $\Q(\sqrt{2})$ is isomorphic to $\Q(\sqrt{3})$. Then, there exists a bijective map $\phi: \Q(\sqrt{3}) \rightarrow \Q(\sqrt{2})$ that preserves multiplication and addition. From above we know there is $x \in \Q(\sqrt{3})$  with $x^2 - 3 = 0$, then $\phi(x^2 - 3) = \phi(0)$. $\phi$ is an isomorphism so $\phi(0)=0$. Also,
	\[
	\phi(x^2 - 3) = \phi(x^2) + \phi(-3) = \phi(x)\phi(x) - \phi(3) = \phi(x)^2 - \phi(1+ 1+ 1) =\phi(x)^2 - (\phi(1)+ \phi(1)+ \phi(1)) .
	\]
	$
	 \text{ Hence there must be an element in } \Q(\sqrt{2}), \ \phi(x) \ \text{ with } \phi(x)^2 - 3 =0.
	$
	From above, no such element exists in $\Q(\sqrt{2})$. Thus, $\Q(\sqrt{2})$ is not isomorphic to $\Q(\sqrt{3})$.
\end{proof}
\end{homeworkProblem}
\begin{homeworkProblem}[Exercise 4.3.17: Find an irreducible polynomial $p(x)$ of degree $3$ over $\Z_2$, and list all elements of $\Z_2 \lbrack x \rbrack \slash \left<  p(x) \right>$. Give the identities necessary to multiply elements. ]
\begin{proof}
	From section 4.2 homework, number 12, the polynomial $x^3 + x + 1$ is irreducible over $\Z_2$. Then, by theorem 9 of Boynton, all elements in $\Z_2 \lbrack x \rbrack \slash \left<  x^3 + x + 1 \right>$ can be represented by some polynomial of the form $ax^2+bx+ c$ for some $a,b,c \in \Z_2$. By statement on page 209 of Beachy, $\Z_2 \lbrack x \rbrack \slash \left<  x^3 + x + 1 \right>$ has $2^3 =8$ polynomials of degree less than $3$. There are $8$ polynomials of degree 2 or less in $\Z_2 \lbrack x \rbrack \slash \left<  x^3 + x + 1 \right>$, so we have the following elements:
	\[
	[0], [1], [x], [x+1], [x^2], [x^2+1], [x^2 + x], [x^2+x+1]
	\]
	The identities necessary to multiply elements are $[x]^3 = [-x-1]$ and $[x]^4=[-x^2-x]$.
\end{proof}
\end{homeworkProblem}

\begin{homeworkProblem}[Exercise 4.3.18: Give addition and multiplication tables for the field $\Z_3 \lbrack x \rbrack \slash \left< x^2+x+2 \right>$. ]
Let $p(x) = x^2+x+2$ with $p(x) \in \Z_3$. Then, $p(0)=2, p(1) = 1, p(2) = 2$ so by proposition 4.2.7, $p(x)$ is irreducible in $\Z_3$. By theorem 9 of Boynton, all elements in $\Z_3 \lbrack x \rbrack \slash \left<  x^2+x+2 \right>$ can be represented by a polynomial of the form $a+bx$ with $a, b \in \Z_3$. Then, by page 209 of Beachy, $\Z_3 \lbrack x \rbrack \slash \left<  x^2+x+2 \right>$ has $3^2=9$ elements. To simplify the table, brackets have been omitted in listing the congruence classes.\\
We will use the identity $[x]^2=[2x+1]$ to multiply these elements.
\\
\[
\begin{array}{c||c|c|c|c|c|c|c|c|c}
+ & 0 & 1 & 2 & x & 2x & x + 1 & x+ 2 & 2x + 1 & 2x + 2 \\
	\hline
	0&0 & 1 & 2 & x & 2x & x + 1 & x+ 2 & 2x + 1 & 2x + 2\\
	\hline
	1 & 1 & 2 & 0 & x + 1 & 2x+ 1 & x + 2 & x & 2x+2 & 2x \\
		 \hline
	  2 &2& 0& 1 & x + 2 & 2x+ 2 & x & x+1 & 2x & 2x+1 \\
	\hline
	 x & x & x + 1 & x+ 2 & 2x  & 0 & 2x+1& 2x+2 & 1 & 2 \\
	\hline
	 2x & 2x  & 2x+ 1 & 2x + 2 & 0 & x & 1 & 2 & x+1 &x+2  \\
	\hline
 x + 1 & x+ 1 & x + 2 & x  & 2x+1 & 1 & 2x+2 & 2x & 2  & 0 \\
	\hline
  x+ 2 & x + 2 & x  & x+1 & 2x+2 & 2 & 2x &2x+1  & 0 &  1  \\
	\hline
2x + 1 & 2x + 1 & 2x+2 & 2x & 1 & x+1 & 2 & 0 & x + 2 & x  \\
	\hline
2x + 2 & 2x+2 & 2x & 2x+1 & 2 &x+2  & 0& 1 & x & x + 1   
\end{array}
\]

\[
\begin{array}{c||c|c|c|c|c|c|c|c|c}
\cdot & 0 & 1 & 2 & x & 2x & x + 1 & x+ 2 & 2x + 1 & 2x + 2 \\
	\hline
0 & 0 & 0 & 0 & 0 & 0 & 0 & 0 & 0   &  0 \\
	\hline
 1 & 0 & 1 & 2 & x & 2x & x + 1 & x+ 2 & 2x + 1 & 2x + 2 \\
		 \hline
 2 & 0 & 2 & 1 & 2x & x  & 2x + 2 & 2x+1 & x+2 & x+1 \\
	\hline
x & 0 & x  & 2x & 2x + 1 & x + 2 & 1 & x+1 & 2x+2 & 2 \\
	\hline
2x & 0 & 2x & x & x + 2 & 2x+1 & 2 & 2x+2 & x+1 & 1  \\
	\hline
x + 1 & 0 & x + 1 & 2x + 2 & 1 & 2 &  x+2 & 2x &x  & 2x+1 \\
	\hline
x+ 2 & 0 & x + 2 & 2x+1 & x+1 & 2x+2 & 2x &2   & 1 & x  \\
	\hline
2x + 1 & 0 & 2x+1 & x+2 & 2x+2 & x+1 &x  & 1 & 2 & 2x  \\
	\hline
2x + 2 & 0 & 2x+2 & x+1 & 2 &1  & 2x+1& x  & 2x & x+2  
\end{array}
\]

\end{homeworkProblem}

\begin{homeworkProblem}[Exercise 4.3.19: Find a polynomial of degree 3 irreducible over $\Z_3$, and use it to construct a field with 27 elements. List the elements of the field; give the identities necessary to multiply elements.]
From exercise 13 in section 4.2 homework, the polynomial $x^3 + 2x + 1$ is irreducible over $\Z_3$. By theorem 4.3.6 of Beachy, $\Z_3 \slash \left< x^3 + 2x+1\right>$ is a field. Also, by page 209 of Beachy, $\Z_3 \slash \left< x^3 + 2x+1\right>$ contains $3^3 = 27$ elements. Notice there are $27$ possible polynomials of degree $\leq $ 2 over $\Z_3$, so we have the following elements in $\Z_3 \slash \left< x^3 + 2x+1\right>$  \[
	[0], [1], [2], [x], [2x], [x+1], [x+2], [2x+1], [2x+2], [x^2], [2x^2], [x^2+1], [x^2+2], [2x^2+1], [2x^2+2],
\] 
\[
[x^2 + x], [x^2+x+1], [x^2+x+2], [x^2 + 2x], [x^2 + 2x+1], [x^2+2x+2]
\]
\[
[2x^2 + x], [2x^2+x+1], [2x^2+x+2], [2x^2 + 2x], [2x^2 + 2x+1], [2x^2+2x+2]
\]
We will need the identities $[x]^3 = [x+2]$ and $[x]^4 = [x^2+2x]$ to multiply these elements.
\end{homeworkProblem}

\begin{homeworkProblem}[Exercise 4.3.21: Find multiplicative inverses of the elements in the given fields. ]
\begin{enumerate}[label=(\alph*)]
\item $[a+bx]$ in $\R \lbrack x \rbrack \slash \left< x^2+1 \right>$ 
\\
If $a=0, b=0$, the element $[0]$ does not have a multiplicative inverse. \\
If $a\neq 0, b =0$, then the inverse of $[a]$ is just $\frac{1}{a}$. \\
Now suppose $b \neq 0$. Use the division algorithm to divide $x^2 + 1$ by $a+bx$:
\[
x^2 + 1 = (a+bx)\left( \frac{1}{b}x - \frac{a}{b^2} \right) + \frac{b^2 + a^2}{b^2}. 
\]
\[
\text{  Equivalently,  } \frac{b^2}{b^2 + a^2}(x^2 + 1) = \frac{b^2}{b^2 + a^2}(a+bx)\left( \frac{1}{b}x - \frac{a}{b^2} \right) + 1. \text{ Thus, in } \R \lbrack x \rbrack \slash \left< x^2+1 \right>,
\]
\[
\ \ 1 = \frac{-b^2}{b^2 + a^2}\left( \frac{1}{b}x - \frac{a}{b^2} \right)(a+bx) = \frac{\left( -bx +a \right)}{b^2 + a^2} (a+bx). \text{ Hence, } [a+bx]^{-1}= \frac{\left( -bx +a \right)}{b^2 + a^2}.
\]
\item $[a+bx]$ in $\Q \lbrack x \rbrack \slash \left< x^2-2 \right>$\\
If $a=0, b=0$, the element $[0]$ does not have a multiplicative inverse. \\
If $a\neq 0, b =0$, then the inverse of $[a]$ is $\frac{1}{a}$. \\
Now suppose $b \neq 0$. Use the division algorithm to divide $x^2 -2 $ by $a+bx$:
\[
x^2 -2 = (a+bx)\left( \frac{1}{b}x - \frac{a}{b^2} \right) + \frac{a^2 -2b^2}{b^2}. \text{ First, consider } a^2 - 2b^2 \neq 0.
\]
\[
\text{  Then,  } \frac{b^2}{a^2 - 2b^2}(x^2 -2) = \frac{b^2}{a^2 - 2b^2}(a+bx)\left( \frac{1}{b}x - \frac{a}{b^2} \right) + 1. \text{ Thus, in } \Q \lbrack x \rbrack \slash \left< x^2-2\right>,
\]
\[
\ \ 1 = \frac{b^2}{2b^2 - a^2}\left( \frac{1}{b}x - \frac{a}{b^2} \right)(a+bx) = \frac{\left( bx -a \right)}{2b^2 - a^2} (a+bx). \text{ Hence, } [a+bx]^{-1}= \frac{\left( bx -a \right)}{2b^2 - a^2}.
\]
Now, if $a^2 - 2b^2 =0$, we'll have $a^2 = 2b^2$ so
\[
\left( \frac{a}{b} \right)^2 = 2 \text{ therefore } \frac{a}{b} = \pm \sqrt{2}.
\]
However, $a, b \in \Q$, $b \neq 0$ implies $\frac{a}{b} \in \Q$ so $a^2 - 2b^2 \neq 0$.
\item $[x^2-2x+1]$ in $\Q \lbrack x \rbrack \slash \left< x^3-2 \right>$\\
The extended Euclidean algorithm yields  \[
1 = (-5x+6)(x^3 - 2) + (5x^2 + 4x + 13)(x^2-2x+1).
\]
However, after expanding $(5x^2 + 4x + 13)(x^2-2x+1)$ and reducing mod $ x^3 - 2$ using the identities $x^3 \equiv 2$ and $x^4 \equiv 2x$, we do not get $1$. So, knowing that the multiplicative inverse is likely of the form $ax^2 + bx+c$, solve for $a,b,c$:
$
1 = (ax^2 + bx+ c)(x^2 -2x + 1)= ax^4 + (b-2a)x^3 + (a-2b + c)x^2 + (b-2c)x + c = (a-2b+c)x^2+(2a+b-2c)x + (2b-4a+c).$ Thus, $a,b,c$ must satisfy $a-2b+c=0$, $2a+b-2c=0$, and $2b-4a + c =1$. So, we will row reduce to solve this system of equations and obtain:
\[
\left[\begin{array}{@{}rrr|c@{}}
1&-2&1&0\\
2&1&-2&0\\
-4&2&1&1
\end{array}\right]
\quad \sim \quad
\left[\begin{array}{@{}rrr|c@{}}
1&0&0&3\\
0&1&0& 4 \\
0&0&1 & 5
\end{array}\right]
\]
So, $[x^2 - 2x + 1]^{-1} = 3x^2 + 4x + 5$. Now, check this inverse works:
\[
(3x^2 + 4x + 5)(x^2 - 2x + 1) = 3x^4 - 2x^3 - 6x + 5 \equiv 3\cdot 2x - 2 \cdot 2 - 6x + 5 = 1.
\]
Thus, $[x^2 - 2x + 1]^{-1} = 3x^2 + 4x + 5$. 

\item $[x^2-2x+1]$ in $\Z_3 \lbrack x \rbrack \slash \left< x^3+x^2+2x+1 \right>$\\
Use the Euclidean Algorithm to find $\gcd(x^2-2x+1, x^3+x^2+2x+1)$:
\begin{eqnarray*}
x^3+x^2+2x+1 & = & (x^2-2x + 1)(x) + x + 1 \\
	x^2-2x + 1 & = & (x+1)x + 1
\end{eqnarray*}
Next, back substitute and simplify to find the multiplicative inverse:
\begin{eqnarray*}
1 & = & (x^2-2x + 1) - (x+1)x \\
 & = &  (x^2-2x + 1) -x( x^3+x^2+2x+1 - (x^2-2x + 1)(x)) \\
 & = & (1 + x^2)(x^2-2x + 1) + (-x)( x^3+x^2+2x+1)
\end{eqnarray*}
Thus, $[x^2-2x+1]^{-1}=1 + x^2$.

\item $[x]$ in $\Z_5 \lbrack x \rbrack \slash \left< x^2+x+1 \right>$
Use the Division Algorithm to find $\gcd(x, x^2+x+1)$:
\[
x^2+x+1  =  (x)(x+1) + 1. \text{ Thus, } x^2+x+1  -  (x)(x+1) = 1
\]
In $\Z_5 , -x-1 \equiv 4x + 4$ so $[x]^{-1}=4x+4$.
\item $[x+4]$ in $\Z_5 \lbrack x \rbrack \slash \left< x^3+x+1 \right>$
\\
Use the division algorithm to find $\gcd(x+4, x^3 + x + 1)$:
\[
x^3 +  x + 1 = (x + 4)(x^2 + x + 2) + 3. \text{ Equivalently, } x^3 +  x + 1 - (x + 4)(x^2 + x + 2) = 3.
\]
In $\Z_5$, $3^{-1}=2$, so we can multiply the last equation by $2$ to obtain: 
\[
2(x^3 +  x + 1) - 2(x + 4)(x^2 + x + 2) = 1. \text{ Also, } - 2\mod 5 = 3, \text{  so   } 
\]
\[
[x+4]^{-1}=3(x^2+x+2) = 3x^2 + 3x + 6 = 3x^2 + 3x + 1.
\]

\end{enumerate}


\end{homeworkProblem}


\end{flushleft}
\end{spacing}
\end{document}

%%%%%%%%%%%%%%%%%%%%%%%%%%%%%%%%%%%%%%%%%%%%%%%%%%%%%%%%%%%%%

%----------------------------------------------------------------------%
% The following is copyright and licensing information for
% redistribution of this LaTeX source code; it also includes a liability
% statement. If this source code is not being redistributed to others,
% it may be omitted. It has no effect on the function of the above code.
%----------------------------------------------------------------------%
% Copyright (c) 2007, 2008, 2009, 2010, 2011 by Theodore P. Pavlic
%
% Unless otherwise expressly stated, this work is licensed under the
% Creative Commons Attribution-Noncommercial 3.0 United States License. To
% view a copy of this license, visit
% http://creativecommons.org/licenses/by-nc/3.0/us/ or send a letter to
% Creative Commons, 171 Second Street, Suite 300, San Francisco,
% California, 94105, USA.
%
% THE SOFTWARE IS PROVIDED "AS IS", WITHOUT WARRANTY OF ANY KIND, EXPRESS
% OR IMPLIED, INCLUDING BUT NOT LIMITED TO THE WARRANTIES OF
% MERCHANTABILITY, FITNESS FOR A PARTICULAR PURPOSE AND NONINFRINGEMENT.
% IN NO EVENT SHALL THE AUTHORS OR COPYRIGHT HOLDERS BE LIABLE FOR ANY
% CLAIM, DAMAGES OR OTHER LIABILITY, WHETHER IN AN ACTION OF CONTRACT,
% TORT OR OTHERWISE, ARISING FROM, OUT OF OR IN CONNECTION WITH THE
% SOFTWARE OR THE USE OR OTHER DEALINGS IN THE SOFTWARE.
%----------------------------------------------------------------------%
